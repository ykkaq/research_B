  \documentclass[a4paper,10pt,twocolumn]{jsarticle}
%DIF LATEXDIFF DIFFERENCE FILE
%DIF DEL old/old.tex   Thu Jan 16 11:02:28 2025
%DIF ADD Outline.tex   Thu Jan 16 11:02:35 2025
\usepackage[dvipdfmx]{graphicx}
\renewcommand{\baselinestretch}{0.8}
\usepackage{url}
\usepackage[top=25truemm,bottom=25truemm,left=20truemm,right=20truemm]{geometry}
\usepackage{float}

\usepackage{amssymb, amsmath}            % 数理環境用パッケージ
\usepackage{amsthm}                      % 定理環境用パッケージ
\usepackage{ascmac}                      % box環境用パッケージ
\usepackage{cite}                        % 参考文献用パッケージ

\usepackage{comment}
\usepackage{mathtools}
\usepackage{hyperref}
\usepackage{pxjahyper}
\usepackage{cleveref}
\usepackage{autonum}

\usepackage{SekineLabOutline}            % 関根研究室梗概用スタイルファイル


%\mathtoolsset{showonlyrefs}

\newcommand{\rad}{radii polynomial approach}
\newcommand{\nk}{Newton-Kantorovich}
\newcommand{\vdp}{van der Pol方程式}
\newcommand{\infg}{無限次元ガウスの消去法}
\newcommand{\fre}{Fr\'{e}chet}

%%%%%%%%%%%%%%%%%%%%%%%%%%%%%%%%
% タイトル
%%%%%%%%%%%%%%%%%%%%%%%%%%%%%%%%
\title{\vspace{-8mm}{\Large \gtfamily\mdseries\upshape 無限次元ガウスの消去法を用いた \rad{}改良 }\vspace{-3mm}}
\date{}
\author{(指導教員 関根 晃太 准教授) \\ 関根研究室 2131701 齋藤 悠希
\vspace{-5mm}}
\pagestyle{empty}
%DIF PREAMBLE EXTENSION ADDED BY LATEXDIFF
%DIF CFONT PREAMBLE %DIF PREAMBLE
\RequirePackage{color}\definecolor{RED}{rgb}{1,0,0}\definecolor{BLUE}{rgb}{0,0,1} %DIF PREAMBLE
\DeclareOldFontCommand{\sf}{\normalfont\sffamily}{\mathsf} %DIF PREAMBLE
\providecommand{\DIFaddtex}[1]{{\protect\color{blue} \sf #1}} %DIF PREAMBLE
\providecommand{\DIFdeltex}[1]{{\protect\color{red} \scriptsize #1}} %DIF PREAMBLE
%DIF SAFE PREAMBLE %DIF PREAMBLE
\providecommand{\DIFaddbegin}{} %DIF PREAMBLE
\providecommand{\DIFaddend}{} %DIF PREAMBLE
\providecommand{\DIFdelbegin}{} %DIF PREAMBLE
\providecommand{\DIFdelend}{} %DIF PREAMBLE
\providecommand{\DIFmodbegin}{} %DIF PREAMBLE
\providecommand{\DIFmodend}{} %DIF PREAMBLE
%DIF FLOATSAFE PREAMBLE %DIF PREAMBLE
\providecommand{\DIFaddFL}[1]{\DIFadd{#1}} %DIF PREAMBLE
\providecommand{\DIFdelFL}[1]{\DIFdel{#1}} %DIF PREAMBLE
\providecommand{\DIFaddbeginFL}{} %DIF PREAMBLE
\providecommand{\DIFaddendFL}{} %DIF PREAMBLE
\providecommand{\DIFdelbeginFL}{} %DIF PREAMBLE
\providecommand{\DIFdelendFL}{} %DIF PREAMBLE
%DIF HYPERREF PREAMBLE %DIF PREAMBLE
\providecommand{\DIFadd}[1]{\texorpdfstring{\DIFaddtex{#1}}{#1}} %DIF PREAMBLE
\providecommand{\DIFdel}[1]{\texorpdfstring{\DIFdeltex{#1}}{}} %DIF PREAMBLE
\newcommand{\DIFscaledelfig}{0.5}
%DIF HIGHLIGHTGRAPHICS PREAMBLE %DIF PREAMBLE
\RequirePackage{settobox} %DIF PREAMBLE
\RequirePackage{letltxmacro} %DIF PREAMBLE
\newsavebox{\DIFdelgraphicsbox} %DIF PREAMBLE
\newlength{\DIFdelgraphicswidth} %DIF PREAMBLE
\newlength{\DIFdelgraphicsheight} %DIF PREAMBLE
% store original definition of \includegraphics %DIF PREAMBLE
\LetLtxMacro{\DIFOincludegraphics}{\includegraphics} %DIF PREAMBLE
\newcommand{\DIFaddincludegraphics}[2][]{{\color{blue}\fbox{\DIFOincludegraphics[#1]{#2}}}} %DIF PREAMBLE
\newcommand{\DIFdelincludegraphics}[2][]{% %DIF PREAMBLE
\sbox{\DIFdelgraphicsbox}{\DIFOincludegraphics[#1]{#2}}% %DIF PREAMBLE
\settoboxwidth{\DIFdelgraphicswidth}{\DIFdelgraphicsbox} %DIF PREAMBLE
\settoboxtotalheight{\DIFdelgraphicsheight}{\DIFdelgraphicsbox} %DIF PREAMBLE
\scalebox{\DIFscaledelfig}{% %DIF PREAMBLE
\parbox[b]{\DIFdelgraphicswidth}{\usebox{\DIFdelgraphicsbox}\\[-\baselineskip] \rule{\DIFdelgraphicswidth}{0em}}\llap{\resizebox{\DIFdelgraphicswidth}{\DIFdelgraphicsheight}{% %DIF PREAMBLE
\setlength{\unitlength}{\DIFdelgraphicswidth}% %DIF PREAMBLE
\begin{picture}(1,1)% %DIF PREAMBLE
\thicklines\linethickness{2pt} %DIF PREAMBLE
{\color[rgb]{1,0,0}\put(0,0){\framebox(1,1){}}}% %DIF PREAMBLE
{\color[rgb]{1,0,0}\put(0,0){\line( 1,1){1}}}% %DIF PREAMBLE
{\color[rgb]{1,0,0}\put(0,1){\line(1,-1){1}}}% %DIF PREAMBLE
\end{picture}% %DIF PREAMBLE
}\hspace*{3pt}}} %DIF PREAMBLE
} %DIF PREAMBLE
\LetLtxMacro{\DIFOaddbegin}{\DIFaddbegin} %DIF PREAMBLE
\LetLtxMacro{\DIFOaddend}{\DIFaddend} %DIF PREAMBLE
\LetLtxMacro{\DIFOdelbegin}{\DIFdelbegin} %DIF PREAMBLE
\LetLtxMacro{\DIFOdelend}{\DIFdelend} %DIF PREAMBLE
\DeclareRobustCommand{\DIFaddbegin}{\DIFOaddbegin \let\includegraphics\DIFaddincludegraphics} %DIF PREAMBLE
\DeclareRobustCommand{\DIFaddend}{\DIFOaddend \let\includegraphics\DIFOincludegraphics} %DIF PREAMBLE
\DeclareRobustCommand{\DIFdelbegin}{\DIFOdelbegin \let\includegraphics\DIFdelincludegraphics} %DIF PREAMBLE
\DeclareRobustCommand{\DIFdelend}{\DIFOaddend \let\includegraphics\DIFOincludegraphics} %DIF PREAMBLE
\LetLtxMacro{\DIFOaddbeginFL}{\DIFaddbeginFL} %DIF PREAMBLE
\LetLtxMacro{\DIFOaddendFL}{\DIFaddendFL} %DIF PREAMBLE
\LetLtxMacro{\DIFOdelbeginFL}{\DIFdelbeginFL} %DIF PREAMBLE
\LetLtxMacro{\DIFOdelendFL}{\DIFdelendFL} %DIF PREAMBLE
\DeclareRobustCommand{\DIFaddbeginFL}{\DIFOaddbeginFL \let\includegraphics\DIFaddincludegraphics} %DIF PREAMBLE
\DeclareRobustCommand{\DIFaddendFL}{\DIFOaddendFL \let\includegraphics\DIFOincludegraphics} %DIF PREAMBLE
\DeclareRobustCommand{\DIFdelbeginFL}{\DIFOdelbeginFL \let\includegraphics\DIFdelincludegraphics} %DIF PREAMBLE
\DeclareRobustCommand{\DIFdelendFL}{\DIFOaddendFL \let\includegraphics\DIFOincludegraphics} %DIF PREAMBLE
%DIF COLORLISTINGS PREAMBLE %DIF PREAMBLE
\RequirePackage{listings} %DIF PREAMBLE
\RequirePackage{color} %DIF PREAMBLE
\lstdefinelanguage{DIFcode}{ %DIF PREAMBLE
%DIF DIFCODE_CFONT %DIF PREAMBLE
  moredelim=[il][\color{red}\scriptsize]{\%DIF\ <\ }, %DIF PREAMBLE
  moredelim=[il][\color{blue}\sffamily]{\%DIF\ >\ } %DIF PREAMBLE
} %DIF PREAMBLE
\lstdefinestyle{DIFverbatimstyle}{ %DIF PREAMBLE
	language=DIFcode, %DIF PREAMBLE
	basicstyle=\ttfamily, %DIF PREAMBLE
	columns=fullflexible, %DIF PREAMBLE
	keepspaces=true %DIF PREAMBLE
} %DIF PREAMBLE
\lstnewenvironment{DIFverbatim}{\lstset{style=DIFverbatimstyle}}{} %DIF PREAMBLE
\lstnewenvironment{DIFverbatim*}{\lstset{style=DIFverbatimstyle,showspaces=true}}{} %DIF PREAMBLE
\lstset{extendedchars=\true,inputencoding=utf8}

%DIF END PREAMBLE EXTENSION ADDED BY LATEXDIFF

\begin{document}

\maketitle
\vspace{-10mm}


%%%%%%%%%%%%%%%%%%%%%%%%%%%%%%%%
% 本文
%%%%%%%%%%%%%%%%%%%%%%%%%%%%%%%%

%%%%%%%%%%%%%%%%%%%%%%%%%%%%%%%%
\section{はじめに}
\vspace{-1mm}
精度保証付き数値計算に関する定理の一つに,\nk{}型の定理を利用した\rad{}がある.この定理は有限次元や無限次元を問わず, 非線形方程式や偏微分方程式など殆どの微分方程式に用いることができる.

従来の\rad{}では,ノルム空間の定義に重み付き$l_1$空間を用いている.重み付き$l_1$空間を用いると,$l_1$空間を用いた場合よりも,定理の適用できる問題が少ない.従来手法で重み付き$l_1$空間を用いている要因として,計算過程に計算困難な無限次元の問題が生じるためである.

本研究では,\vdp{}
\begin{equation}
  \frac{d^2x}{dt^2} - \mu (1-x^2)\frac{dx}{dt}+x=0
\end{equation}
を問題とし,無限次元ガウスの消去法を用いて,\rad{}におけるノルム空間を$l_1$空間で\DIFdelbegin \DIFdel{定義可能}\DIFdelend \DIFaddbegin \DIFadd{線形化作用素が全単射である}\DIFaddend か検証することを目的とする.


%%%%%%%%%%%%%%%%%%%%%%%%%%%%%%%%

\vspace{-1mm}
\section{\rad{} \cite{github}}
\vspace{-1mm}

\begin{Thm}
  \label{thm:radii}
  有界線形作用素$A^\dagger \in \mathcal{L}(X, Y), A \in \mathcal{L}(Y, X)$を考え,作用素$F:X \rightarrow Y$が$C^1\text{-\fre{}}$微分可能であるとする.また,$A$が単射\DIFdelbegin \DIFdel{で,$AF:X \rightarrow X$}\DIFdelend とする.いま,$\bar{x} \in X$に対して,
  \begin{align}
      \|AF(\bar{x})\|_X                             & \leq Y_0                                             \\
      \|I-AA^\dagger\|_{\mathcal{L}(X)}              & \leq Z_0                                             \\
      \|A(DF(\bar{x})-A^\dagger)\|_{\mathcal{L}(X)} & \leq Z_1                                             \\
      \begin{split}
        \|A(DF(b)-DF(\bar{x}))\|_{\mathcal{L}(X)} &\leq Z_2(r)r, \\ & \forall b \in \overline{B(\bar{x},r)}
      \end{split}
  \end{align}
  が成り立つとする.このとき
  \begin{equation}
    p(r) := Z_2(r)r^2 - (1-Z_1-Z_0)r+Y_0
  \end{equation}
  をradii polynomialといい,もし$p(r_0)<0$となる$r_0>0$が存在すれば,$F(\tilde{x})=0$をみたす$\tilde{x}$が$\overline{B(\tilde{x},r_0)}$内に一意存在する.
\end{Thm}

%%%%%%%%%%%%%%%%%%%%%%%%%%%%%%%%
\vspace{-1mm}
\section{提案手法}
\vspace{-1mm}

\rad{}で用いられる作用素$A_M,\ DF(x)$を以下のように定義する.
\begin{equation}
  \begin{split}
    DF(x) = \left[
      \begin{array}{c|ccc}
        0 & \cdots & 1 & \cdots \\ \hline
        \vdots & & \vdots &  \\
        \partial_w f_k & \cdots & \partial_{a_j}f_k & \cdots \\
        \vdots & & \vdots &
      \end{array}
    \right] %\in \mathbb{C}^{(2N+2M) \times (2N+2M)}
  \end{split}
\end{equation}
また,
\begin{equation}
  \bar{x} = (\omega,\ \underbrace{0,\cdots,0}_{M_2},\ a,\ \underbrace{0,\cdots,0}_{M_2}) \in \mathbb{C}^{2N+2M_2}
\end{equation}
と定めると,$A_M$の定義は,以下のように定義できる.
\begin{equation}
  A_M = \left(
  \begin{array}{c|ccc}
    DF(\bar{x})^{-1} & 0 & \cdots & \cdots \\ \hline
    0 & \lambda_N^{-1} &  & 0 \\
    \vdots &  & \lambda_{N+1}^{-1} &  \\
    \vdots & 0 &   & \ddots
  \end{array}
  \right)
\end{equation}
定義した作用素を用いて,\DIFdelbegin \DIFdel{$DF(\bar{x})^{-1}$}\DIFdelend \DIFaddbegin \DIFadd{$DF(\bar{x})$}\DIFaddend の全単射性を確かめる.$\Pi_N$を射影演算子とし,
\begin{equation}
  \begin{split}
    &\| I_{X_2} - S\|\\
    &= \| \left( \left( I-\Pi_N \right) ADF ( \bar{x} ) \right)|_{X_2}  \\
    &\quad -(\left( I-\Pi_N \right) ADF( \bar{x} ))|_{X_1} \left((\Pi_N ADF(\bar{x}))|_{X_1}\right)^{-1}\\
    &\qquad ((I-\Pi_N)ADF(\bar{x}))|_{X_2} \| \\
    & < 1
  \end{split}
\end{equation}
となれば,\DIFdelbegin \DIFdel{$DF(\bar{x})^{-1}$の}\DIFdelend \DIFaddbegin \DIFadd{$DF(\bar{x})$は}\DIFaddend 全単射\DIFdelbegin \DIFdel{性}\DIFdelend である.

%%%%%%%%%%%%%%%%%%%%%%%%%%%%%%%%
\vspace{-1mm}
\section{実験結果}
\vspace{-1mm}
%%%%%%%%%%%%%%%%%%%%%%%%%%%%%%%

実験では,フーリエ係数の次数を変更した際のノルム値を検証した.検証結果を\cref{tab:norm-num}に示す.結果より,ノルム値が1より小さいことがわかることから,\rad{}において\DIFaddbegin \DIFadd{,ノルム空間を$l_1$空間とし,}\DIFaddend 無限次元ガウスの消去法による計算手法が有効可能であることがわかる.
\DIFdelbegin %DIFDELCMD < 

%DIFDELCMD < %%%
\DIFdelend \begin{table}[htbp]
  \centering
  \caption{フーリエ係数の次数の変更とノルム値の比較}
  \label{tab:norm-num}
  \begin{tabular}{c||c}
    次数 & $\| I_{X_2}-S \|$ \\ \hline
    50 & 0.22815114629236252 \\
    100&0.11455533660051737\\
    150&0.07655718822651922\\
    200&0.05749210273025131
\end{tabular}
\end{table}

%%%%%%%%%%%%%%%%%%%%%%%%%%%%%%%%
\vspace{-1mm}
\section{おわりに}
\vspace{-1mm}
%%%%%%%%%%%%%%%%%%%%%%%%%%%%%%%%
本研究では,\DIFdelbegin \DIFdel{\rad{}の適}\DIFdelend \DIFaddbegin \DIFadd{線形化作}\DIFaddend 用\DIFaddbegin \DIFadd{素が全単射}\DIFaddend で\DIFdelbegin \DIFdel{き}\DIFdelend \DIFaddbegin \DIFadd{あ}\DIFaddend る\DIFdelbegin \DIFdel{問題}\DIFdelend \DIFaddbegin \DIFadd{こと}\DIFaddend を\DIFdelbegin \DIFdel{増や}\DIFdelend \DIFaddbegin \DIFadd{l1空間で証明}\DIFaddend す\DIFaddbegin \DIFadd{る}\DIFaddend ことを\DIFdelbegin \DIFdel{,}\DIFdelend 無限次元ガウスの消去法\DIFdelbegin \DIFdel{を用いること}\DIFdelend で\DIFdelbegin \DIFdel{可能になるかを}\DIFdelend 検証した.実験より,\rad{}の定理の一部の評価式に対し,期待した結果を得ることができ,無限次元ガウスの消去法が有効であることがわかった.今後の課題として,無限次元ガウスの消去法を他の評価式を対象に適応する手法を検討する.

%%%%%%%%%%%%%%%%%%%%%%%%%%%%%%%%
% 参考文献
%%%%%%%%%%%%%%%%%%%%%%%%%%%%%%%%
\vspace{-1mm}

{\footnotesize
\bibliography{reference}
}
\bibliographystyle{junsrt} %参考文献出力スタイル

\end{document}
