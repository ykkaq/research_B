\subsection*{_}
ここで,$S$について計算する.線形作用素の定義より,
\begin{equation}
  \begin{split}
    S &= E-CT^{-1}B \\
    &= \left( \left( I-\Pi_N \right) ADF ( \bar{x} ) \right) - (\left( I-\Pi_N \right) ADF( \bar{x} )) (\Pi_N ADF(\bar{x}))^{-1} ((I-\Pi_N)ADF(\bar{x}))
  \end{split}
  \label{eq:s-def}
\end{equation}

\subsubsection*{$ADF(\bar{x})$を計算する.}
式(\ref{eq:s-def})を分割して計算する.

以下の定義より,計算をする.
\begin{align}
  \bar{x} &= \left( \omega, a_{-N+1}, \cdots, a_{N-1}, 0, \cdots \right) \\
  (D_xf)_{ij} &= \partial_{x_i} f_j \\
  \begin{split}
    F(\bar{x}) &= \begin{bmatrix}
      \eta(a) \\
      (f_k(a))_k\in \mathbb{Z}
    \end{bmatrix}, \\
    \eta(a) &= \sum_{k\in\mathbb{Z}} a_k - \eta_0 \\
    f_k(a) &= -k^2\omega^2a_k - \mu i k \omega a_k + \frac{\mu}{3}(ik\omega)(a*a*a)_k + a_k\quad (\text{van der Pol})
  \end{split}
\end{align}

これらより,
\begin{equation*}
  \begin{split}
    DF(\bar{x}) &=
    \begin{bmatrix}
       & \multicolumn{1}{c|}{0} & 1  & \cdots & 1  & \\ \cline{2-5}
       & \multicolumn{1}{c|}{\vdots} & & \vdots & &  \\
       & \multicolumn{1}{c|}{\partial_w f_k} & \cdots & \partial_{a_j}f_k & \cdots & \\
       & \multicolumn{1}{c|}{\vdots} & & \vdots & &
    \end{bmatrix}\\
    \partial_\omega f_k &= (-2k^2\omega-\mu ik)a_k + \frac{\mu ik}{3} (a*a*a)_k \\
    \partial_{a_j} f_k &= (k^2\omega^2-\mu ik\omega +1)\delta_{kj} + \mu ik \omega (a*a)_{k-j}
  \end{split}
\end{equation*}

となる.なお,$\delta_{kj}$はクロネッカーのデルタである.

これより,

\begin{equation}
  \begin{split}
    &ADF(\bar{x}) \\
    &= \begin{bmatrix}
      A_N & 0  & \cdots & \cdots  \\
      0 & \lambda_{N+1}^{-1} & & \mathbf{0} \\
      \vdots & & \lambda_{N+2}^{-1} & \\
      \vdots & \mathbf{0} & & \ddots
    \end{bmatrix}
    \begin{bmatrix}
      0 & 1  & \cdots & \cdots \\
      \vdots & & \vdots &\\
      \partial_w f_k & \cdots & \partial_{a_j}f_k & \cdots \\
      \vdots & & \vdots &
   \end{bmatrix}\\
    &= \begin{bmatrix}
      & \multicolumn{1}{c|}{A_N} & 0 & \cdots & \cdots  & \\ \cline{2-5}
      & \multicolumn{1}{c|}{0} & \lambda_{N+1}^{-1} & & \mathbf{0} & \\
      & \multicolumn{1}{c|}{\vdots} & & \lambda_{N+2}^{-1} & & \\
      & \multicolumn{1}{c|}{\vdots} & \mathbf{0} & & \ddots &
    \end{bmatrix}
    \begin{bmatrix}
      & \multicolumn{1}{c|}{DF^{(N)}(\bar{x})^{-1}} & 1  & \cdots & \cdots &\\ \cline{2-5}
      & \multicolumn{1}{c|}{\vdots} & & \vdots & &\\
      & \multicolumn{1}{c|}{\partial_w f_k} & \cdots & \partial_{a_j}f_k & \cdots &\\
      & \multicolumn{1}{c|}{\vdots} & & \vdots & &
    \end{bmatrix}\\
    &= \begin{bmatrix}
      & \multicolumn{1}{c|}{T^{-1}} & \mathbf{0} &  \\ \cline{2-3}
      & \multicolumn{1}{c|}{\mathbf{0}} & \Lambda &
    \end{bmatrix}
    \begin{bmatrix}
      & \multicolumn{1}{c|}{T} & M_{01} &  \\ \cline{2-3}
      & \multicolumn{1}{c|}{M_{10}} & M_{11} &
    \end{bmatrix}\\
    &= \begin{bmatrix}
      T^{-1}T & T^{-1}M_{01} \\
      \Lambda M_{10} & \Lambda M_{11}
    \end{bmatrix}
  \end{split}
  \label{eq:ADF-0}
\end{equation}

式(\ref{eq:ADF-0})について,行列ブロックに分けて計算する.はじめに,
\begin{equation*}
  \begin{split}
    T^{-1}
    =\begin{bmatrix}
      T^{-1}_{0\ 0} & \cdots & T^{-1}_{0\ 2N} \\
      \vdots & \ddots & \vdots \\
      T^{-1}_{2N\ 0} & \cdots & T^{-1}_{2N\ 2N} \\
    \end{bmatrix}
  \end{split}
\end{equation*}
とおく.

\begin{equation}
  T^{-1}T = I \mid_{2N}
\end{equation}

% T^{-1}M_{01}q
\begin{equation}
  \begin{split}
    &T^{-1}M_{01} \\
    &= T^{-1}
    \begin{bmatrix}
      1 & \cdots & \cdots \\
      \partial_{a_N} f_{-N+1} & \partial_{a_{N+1}} f_{-N+1}  & \cdots \\
      \partial_{a_N} f_{-N} & \partial_{a_{N+1}} f_{-N}  & \cdots \\
      \vdots & \vdots & \vdots \\
      \partial_{a_N} f_{N-1} & \partial_{a_{N+1}} f_{N-1}  & \cdots
    \end{bmatrix}\\
    &= \begin{bmatrix}
      T^{-1}_{0\ 0} & \cdots & T^{-1}_{0\ 2N} \\
      \vdots & \ddots & \vdots \\
      T^{-1}_{2N\ 0} & \cdots & T^{-1}_{2N\ 2N} \\
    \end{bmatrix}
    \begin{bmatrix}
      1 & \cdots & \cdots \\
      \partial_{a_N} f_{-N+1} & \partial_{a_{N+1}} f_{-N+1}  & \cdots \\
      \vdots & \vdots & \vdots \\
      \partial_{a_N} f_{N-1} & \partial_{a_{N+1}} f_{N-1}  & \cdots
    \end{bmatrix}\\
    &= \begin{bmatrix}
      T^{-1}_{0\ 0} & \cdots & T^{-1}_{0\ 2N} \\
      \vdots & \ddots & \vdots \\
      T^{-1}_{2N\ 0} & \cdots & T^{-1}_{2N\ 2N} \\
    \end{bmatrix}
    \begin{bmatrix}
      1 & \cdots & \cdots  & \cdots\\
      0 & \cdots & \cdots  & \cdots\\
      \mu i (-N+1) \omega (a*a)_{-2N+2} & 0 & \cdots & \cdots\\
      \mu i (-N+2) \omega (a*a)_{-2N+1} & \mu i (-N+2) \omega (a*a)_{-2N+2} & 0 & \cdots \\
      \vdots & \ddots & \ddots & \ddots \\
      \mu i (N-1) \omega (a*a)_{-1} & \mu i (N-1) \omega (a*a)_{-2}  & \cdots & \cdots
    \end{bmatrix}\\
    &=\cdots やっていく
  \end{split}
\end{equation}

% lambda M10
\begin{equation}
  \begin{split}
    &\Lambda M_{10} \\
    &= \begin{bmatrix}
      \lambda^{-1}_N & & \mathbf{0} \\
      & \lambda^{-1}_{N+1} & \\
      \mathbf{0} &  & \ddots &
    \end{bmatrix}
    \begin{bmatrix}
      \partial_{\omega} f_{N} & \partial_{a_{-N+1}} f_{N} &  & \cdots \\
      \partial_{\omega} f_{N+1} & \partial_{a_{-N+1}} f_{N+1}  & \cdots \\
      \vdots & \vdots & \ddots \\
    \end{bmatrix}\\
    &= \begin{bmatrix}
      \lambda^{-1}_N & & \mathbf{0} \\
      & \lambda^{-1}_{N+1} & \\
      \mathbf{0} &  & \ddots &
    \end{bmatrix}\\
    & \begin{bmatrix}
      3^{-1}\mu i N (a*a*a)_{N} & 0 & \mu i N \omega (a*a)_{2N-2} & \mu i N \omega (a*a)_{2N-1} & \cdots & \mu i N \omega (a*a)_1 &\\
      3^{-1}\mu i (N+1) (a*a*a)_{N+1} & 0 & 0 & \mu i (N+1) \omega (a*a)_{2N-2} & \cdots & \mu i (N+1) \omega (a*a)_2 &\\
      \vdots & \vdots & \vdots & \ddots & \ddots & \vdots\\
      3^{-1}\mu i (3N-3) (a*a*a)_{3N-3} & 0 & 0 & 0 & \cdots & 0\\
      0 & 0 & 0 & 0 & \cdots & 0 \\
      \vdots & \vdots & \vdots & \vdots & \ddots & \vdots\\
    \end{bmatrix}
  \end{split}
\end{equation}


%lambda M11
\begin{equation}
  \tiny
  \begin{split}
    &\Lambda M_{11}\\
    &=\begin{bmatrix}
      \lambda^{-1}_N & & \mathbf{0} \\
      & \lambda^{-1}_{N+1} &\\
      \mathbf{0} &  & \ddots
    \end{bmatrix}
    \begin{bmatrix}
      \partial_{a_{N}} f_{N} & \partial_{a_{N+1}} f_{N}  & \cdots \\
      \partial_{a_{N}} f_{N+1} & \partial_{a_{N+1}} f_{N+1}  & \cdots \\
      \vdots & \vdots & \ddots \\
    \end{bmatrix}\\
    &=\begin{bmatrix}
      \lambda^{-1}_N &  & \mathbf{0} \\
      & \lambda^{-1}_{N+1} &\\
      \mathbf{0} &  & \ddots
    \end{bmatrix}\\
    &\begin{bmatrix}
      \lambda_{N} & \mu i N \omega (a*a)_{-1} & \cdots & \mu i N \omega (a*a)_{-2N+2} & 0 & \cdots \\
      \mu i (N+1) \omega (a*a)_{1} & \lambda_{N+1} & \cdots & \mu i (N+1) \omega (a*a)_{-2N+1} & \mu i (N+1) \omega (a*a)_{-2N+2} & \cdots \\
      \vdots & \ddots & \ddots & \ddots & \ddots & \ddots \\
      \mu i (3N-2) \omega (a*a)_{2N-2} & \mu i (3N-2) \omega (a*a)_{2N-1} & \ddots & \lambda_{3N-2} & \mu i (3N-2) \omega (a*a)_{-1} & \cdots \\
      0 & \mu i (3N-3) \omega (a*a)_{2N-2} & \ddots & \mu i (3N-3) \omega (a*a)_{1} & \ddots & \ddots \\
      \vdots & \vdots & \vdots & \vdots & \vdots & \ddots \\
    \end{bmatrix}
  \end{split}
\end{equation}

\subsection*{$M_{01}^T \stackrel{\mathrm{?}}{=} M_{10}$}
\begin{equation}
  \tiny
  \begin{split}
    &M_{01}
    =\begin{bmatrix}
      1 & 1 & 1 & \cdots\\
      0 & 0 & 0 & \cdots\\
      \mu i (-N+1) \omega (a*a)_{-2N+2} & 0 & 0 & \cdots\\
      \mu i (-N+2) \omega (a*a)_{-2N+1} & \mu i (-N+2) \omega (a*a)_{-2N+2} & 0 & \cdots \\
      \vdots & \ddots & \ddots & \ddots \\
      \mu i (N-1) \omega (a*a)_{-1} & \mu i (N-1) \omega (a*a)_{-2}  & \cdots & \cdots
    \end{bmatrix}\\
    &M_{10}
    = \begin{bmatrix}
      3^{-1}\mu i N (a*a*a)_{N} & 0 & \mu i N \omega (a*a)_{2N-2} & \mu i N \omega (a*a)_{2N-1} & \cdots & \mu i N \omega (a*a)_1 &\\
      3^{-1}\mu i (N+1) (a*a*a)_{N+1} & 0 & 0 & \mu i (N+1) \omega (a*a)_{2N-2} & \cdots & \mu i (N+1) \omega (a*a)_2 &\\
      \vdots & \vdots & \vdots & \ddots & \ddots & \vdots\\
      3^{-1}\mu i (3N-3) (a*a*a)_{3N-3}  & 0 & 0 & 0 & \cdots & 0\\
      0 & 0 & 0 & 0 & \cdots & 0 \\
      \vdots & \vdots & \vdots & \vdots & \ddots & \vdots\\
    \end{bmatrix}
  \end{split}
\end{equation}

$\mu i (-x) \omega (a*a)_{-x} \stackrel{\mathrm{?}}{=} \mu i (x) \omega (a*a)_{x}$について考える.

\begin{equation}
  \begin{split}
    &\mu i (-x) \omega (a*a)_{-x}\\
    &= -\mu ix \sum_{\substack{k_1+k_2=-x \\ k_i\in\mathbb{Z}}} a_{k_1}a_{k_2}\\
    &= -\mu ix (a_{-N+1}a_{N-1-x}+a_{-N+2}a_{N-2-x}+\cdots)
  \end{split}
\end{equation}

\begin{equation}
  \begin{split}
    &\mu i x \omega (a*a)_{x}\\
    &= \mu ix \sum_{\substack{k_1+k_2=x \\ k_i\in\mathbb{Z}}} a_{k_1}a_{k_2}\\
    &= \mu ix (a_{-N+1}a_{N-1+x}+a_{-N+2}a_{N-2+x}+\cdots)
  \end{split}
\end{equation}

より,

\begin{equation}
  \begin{split}
    &\mu i x \omega (a*a)_{x} - \mu i (-x) \omega (a*a)_{-x} \\
    &= \mu ix (a_{-N+1}a_{N-1+x}+a_{-N+2}a_{N-2+x}+\cdots) + \mu ix (a_{-N+1}a_{N-1-x}+a_{-N+2}a_{N-2-x}+\cdots)\\
    &= \mu ix (a_{-N+1}a_{N-1+x}+a_{-N+2}a_{N-2+x}+\cdots + a_{-N+1}a_{N-1-x}+a_{-N+2}a_{N-2-x}+\cdots)\\
    &= \mu ix (a_{-N+1}a_{N-1+x}+a_{-N+1+1}a_{N-1-1+x}+\cdots + a_{-N+1}a_{N-1-x}+a_{-N+1+1}a_{N-1-1-x}+\cdots)\\
    &= \mu ix (a_{-N+1}(a_{N-1+x}+a_{N-1-x})+a_{-N+1+1}(a_{N-1-1+x}+a_{N-1-1-x})+\cdots )\\
  \end{split}
\end{equation}

これらより,畳み込みの等号が成り立つ条件は,$j\in \mathbb{N}$とすると,
\begin{equation}
  a_{N-1-j+x} = a_{N-1-j-x}
\end{equation}

よって,$M_{01}^T \stackrel{\mathrm{?}}{=} M_{10}$が成り立つ条件は,$j,x,y\in \mathbb{N}$としとき,
\begin{equation}
  \begin{cases}
    a_{N-1+x-j} = a_{N-1-x-j}\\
    3^{-1}\mu i (a*a*a)_y = 1
  \end{cases}
\end{equation}
であると考えられる.


\subsection*{ゲルシュゴリンの定理}
[http://www.oishi.info.waseda.ac.jp/~oishi/lec2003/lec7.pdf] \\


\dfn[ゲルシュゴリンの定理]
行列$A$の固有値は複素$\lambda$平面の中の次の集合の中に存在する.
\begin{equation}
  K=\bigcup_{i=1}^{n} \left\{ \mu \in C \middle| |\mu - a_{ii}| \leq \sum_{k=1,k\neq i}^{n} |a_k| \right\}
\end{equation}
右辺の和集合の中の各集合は中心を$a_{ii}$とする半径
\begin{equation}
  r=\sum_{k=1,k\neq i}^{n} |a_{ik}|
\end{equation}
の複素平面内の閉円板である.

\begin{proof}
  $x$が行列$A$の固有値$\lambda$に対する固有ベクトルであるとする.$n \times n$行列$B$を任意に選び,$\lambda$は$B$の固有値でないとする.このとき,
  \begin{equation}
    (A-B)x = (\lambda I - B)x
  \end{equation}
  となる.$\lambda$が$B$の固有値でないことから$(\lambda I - B)^{-1}$が存在することがわかる.よって
  \begin{equation}
    x=(\lambda I - B)(A-B)x
  \end{equation}
  となる.これから,
  \begin{equation}
    || (\lambda I - B)(A-B) || \geq 1
  \end{equation}
  を得る.

\end{proof}

\subsection*{$I-S$}
行列の積の解を$\begin{bmatrix} A & B \\ C & D\end{bmatrix},\ t,r\in \mathbb{Z}$とする.

\begin{equation}
  \begin{split}
    A = I
  \end{split}
\end{equation}

\begin{equation}
    B_{jk} = T_{j,0}^{-1}+\sum_{\substack{k \leq t \leq 2N-2}} \, T_{j,k+2+t}^{-1} \mu i (- N + 1 + t) \omega (a*a)_{-2N+2+t-k}
\end{equation}

\begin{equation}
  \begin{split}
    &C_{jk} \\
    &=\begin{cases}
      \lambda_{N+j}^{-1} 3^{-1} \mu i (N+j) (a*a*a)_{N+j} & \text{if $N \leq N+j \leq 3N-3$} \\
      \lambda_{N+j}^{-1} \mu i (N+j) (a*a)_{2N-2 + j - (k-2)} & \text{if $ |j - k| \leq 2N-2$} \\
      0 & \text{if $3N-3 < N+j \lor |j - k| > 2N-2$}
    \end{cases}
  \end{split}
\end{equation}

\begin{equation}
  \begin{split}
    &D_{jk} \\
    &=\begin{cases}
      1 & \text{if $j = k$} \\
      \lambda_{N+j}^{-1} \mu i (N + j) \omega (a*a)_{j-k} & \text{if $|j-k| \leq 2N-2 \land j \neq k$} \\
      0 & \text{if otherwise}
    \end{cases}
  \end{split}
\end{equation}

ここで,$D-CB$を考える.

\begin{equation}
  \begin{split}
    &(CB)_{jk} \\
    &=\begin{cases}
      (\lambda_{N+j}^{-1} \mu i N) (3^{-1} T_{a})
    \end{cases}
  \end{split}
\end{equation}


\subsubsection*{$\|D\|$}

\begin{equation}
  \begin{split}
    & D \\
    &=\left\|
    \begin{bmatrix}
      1 & \lambda_N^{-1} \mu i N \omega (a*a)_{1} & \lambda_N^{-1} \mu i N \omega (a*a)_{2} & \cdots & \lambda_N^{-1} \mu i N \omega (a*a)_{-2N+2} & 0 & \cdots\\
      \lambda_{N+1}^{-1} \mu i (N+1) \omega (a*a)_{-1} & 1 & \lambda_{N+1}^{-1} \mu i (N+1) \omega (a*a)_{1} & \ddots & \lambda_{N+1}^{-1} \mu i (N+1) \omega (a*a)_{2N-1} & \ddots & \ddots\\
      \lambda_{N+2}^{-1} \mu i (N+2) \omega (a*a)_{-2}  & \lambda_{N+2}^{-1} \mu i (N+2) \omega (a*a)_{-1} & 1 & \ddots & \ddots & \ddots & \ddots\\
      \vdots & \ddots & \ddots & \ddots & \ddots & \ddots & \ddots\\
      \lambda_{3N-1}^{-1} \mu i (3N-1) \omega (a*a)_{-2N+2}  & \lambda_{3N-1}^{-1} \mu i (3N-1) \omega (a*a)_{-2N+3} & \ddots & \ddots & \ddots & \ddots & \ddots\\
      0 & \lambda_{3N-2}^{-1} \mu i (3N-2) \omega (a*a)_{-2N+2} & \ddots & \ddots & \ddots & \ddots & \ddots\\
      \vdots & \ddots & \ddots & \ddots & \ddots & \ddots & \ddots\\
    \end{bmatrix}\right\|
  \end{split}
\end{equation}


\begin{equation}
  \begin{split}
    \| D \| =
    \sup \left\lbrace \right. &\\
      &1 + \frac{\mu i (N+1) \omega (a*a)_{-1} }{-(N+1)^2 \omega^2 - \mu i (N+1) \omega +1} + \frac{ \mu i (N+2) \omega (a*a)_{-2} }{-(N+2)^2 \omega^2 - \mu i (N+2)\omega + 1} + \cdots  \\ & \quad + \frac{ \mu i (3N-1) \omega (a*a)_{-2N+2} }{-(3N-1)^2 \omega^2 - \mu i (3N-1)\omega + 1 }, \\
      &\frac{\mu i N \omega (a*a)_{1} }{-N^2 \omega^2 - \mu i N \omega +1} + 1 + \frac{\mu i (N+2) \omega (a*a)_{-1} }{-(N+2)^2 \omega^2 - \mu i (N+2) \omega +1} + \cdots \\ & + \frac{ \mu i (3N-2) \omega (a*a)_{-2N+2} }{-(3N-2)^2 \omega^2 - \mu i (3N-2)\omega + 1 }, \\
      & \qquad \vdots \\
      &\frac{\mu i N \omega (a*a)_{2N-2} }{-N^2 \omega^2 - \mu i N \omega +1} + \cdots + \frac{ \mu i (5N-3) \omega (a*a)_{-2N+2} }{-(5N-3)^2 \omega^2 - \mu i (5N-3)\omega + 1 }, \\
      &\frac{\mu i (N+1) \omega (a*a)_{2N-2} }{-(N+1)^2 \omega^2 - \mu i (N+1) \omega +1} + \cdots + \frac{ \mu i (5N-2) \omega (a*a)_{-2N+2} }{-(5N-2)^2 \omega^2 - \mu i (5N-2)\omega + 1 }, \\
      & \qquad \vdots \\
    \left. \right\rbrace \geq 1 &
  \end{split}
\end{equation}


$(\lambda_k = -k^2 \omega^2 - \mu i k \omega + 1)$

\subsubsection*{11/19}
1. $\sum < \sum$

\begin{equation}
  \begin{split}
    & \sum_{\substack{t=0 \\ t \neq 2t-1}}^{4t-3} \frac{\mu i \omega (N+a+t)(a*a)_{-2N+2+t}}{-(N+a+t)^2-\mu i \omega (N+a+t)+1} +1 \\
    & \ -\left\lparen \sum_{\substack{t=0 \\ t \neq 2t-1}}^{4t-3} \frac{\mu i \omega (N+a+1+t)(a*a)_{-2N+2+t}}{-(N+a+1+t)^2-\mu i \omega (N+a+1+t)+1} +1 \right\rparen\\
    & = \sum_{\substack{t=0 \\ t \neq 2t-1}}^{4t-3} \left\lbrace \right. \left\lbrace \right.  \frac{ (N+a+t)}{-(N+a+t)^2-\mu i \omega (N+a+t)+1} \\
    & \ - \frac{ (N+a+1+t)}{-(N+a+1+t)^2-\mu i \omega (N+a+1+t)+1} \left. \right\rbrace \mu i \omega(a*a)_{-2N+2+t} \left. \right\rbrace \\
    &\text{(分母を合わせる)} \\
    & = \left\lbrace \right. \frac{ (N+a) (a*a)_{-2N+2} }{ -(N+a)^2-\mu i \omega (N+a)+1} \\
    & \ + \sum_{\substack{t=1 \\ t \neq 2t-1}}^{4t-2} \left\lbrace \right. \frac{ (N+a+t) (a*a)_{-2N+3+t} }{ -(N+a+t)^2-\mu i \omega (N+a+t)+1 } - \frac{ (N+a+t) (a*a)_{-2N+2+t} }{-(N+a+t)^2-\mu i \omega (N+a+t)+1} \left. \right\rbrace \\
    & \ + \frac{ (N+a+4N-2) (a*a)_{(2N-1)} }{ -(N+a+4N-2)^2-\mu i \omega (N+a+4N-2)+1 } \left. \right\rbrace (\mu i \omega)\\
    & = \left\lbrace \right. \frac{ (N+a) (a*a)_{-2N+2} }{ -(N+a)^2-\mu i \omega (N+a)+1} \\
    & \ + \sum_{\substack{t=1 \\ t \neq 2t-1}}^{4t-2} \left\lbrace \right. \frac{ (N+a+t)  }{ -(N+a+t)^2-\mu i \omega (N+a+t)+1 } ( (a*a)_{-2N+3+t} - (a*a)_{-2N+2+t} ) \left. \right\rbrace \\
    & \ + \frac{ (N+a+4N-2) (a*a)_{(2N-1)} }{ -(N+a+4N-2)^2-\mu i \omega (N+a+4N-2)+1 } \left. \right\rbrace (\mu i \omega)\\
    & = ?
  \end{split}
\end{equation}

%---------------------

2. $\|C\|, \|T^{-1}\|, \|M_{01}\|$
\begin{equation}
  \begin{split}
    \|C\| =& \sup \left\lbrace \right. \\
    & \| \sum_{t=0}^{2N-3} \frac{\mu i \omega (N+t) (a*a*a)_{N+t}}{3(-(N+t)^2 \omega^2 - \mu i \omega (N+t)+ 1)} \|,\\
    & 0,\\
    & \| \sum_{t=0}^{0} \frac{ \mu i \omega (N+t) (a*a)_{2N-2 + t}}{-(N+t)^2 \omega^2 - \mu i \omega (N+t)+ 1} \|,\\
    & \| \sum_{t=0}^{1} \frac{ \mu i \omega (N+t) (a*a)_{2N-1 + t} }{-(N+t)^2 \omega^2 - \mu i \omega (N+t)+ 1}\|,\\
    & \vdots , \\
    & \| \sum_{t=0}^{2N-3} \frac{ \mu i \omega (N+t) (a*a)_{1 + t} }{-(N+t)^2 \omega^2 - \mu i \omega (N+t)+ 1}\|,\\
    \left. \right\rbrace&
  \end{split}
\end{equation}


\begin{equation}
  \begin{split}
    \|T^{-1}\| =& \sup \left\lbrace \right. \\
    & \|\sum_{t=0}^{2N} T_{0\ t}^{-1}\|, \\
    & \vdots \\
    & \|\sum_{t=0}^{2N} T_{2N\ t}^{-1}\|, \\
    \left. \right\rbrace&
  \end{split}
\end{equation}

\begin{equation}
  \begin{split}
    \|M_{01}\| =& \sup \left\lbrace \right. \\
    &\| 1 + \mu i \omega \sum_{t=0}^{2N-3} (-N+1 + t) (a*a)_{-2N+2+t} \| , \\
    &\| 1 + \mu i \omega \sum_{t=0}^{2N-4} (-N+2 + t) (a*a)_{-2N+2+t} \| , \\
    & \vdots \\
    &\| 1 + \mu i \omega \sum_{t=0}^{0} (N-1 + t) (a*a)_{-2N+2+t} \| , \\
    & 1 \\
    \left. \right\rbrace&
  \end{split}
\end{equation}

\subsubsection*{11/25}
1. $\sum < \sum$

\begin{equation}
  \begin{split}
    \left\| \frac{\mu i \omega N (a*a)_{-2N+2}}{-N^2 \omega^2 - \mu i \omega N + 1} \right\| - \left\| \frac{\mu i \omega (N+1) (a*a)_{-2N+2}}{-(N+1)^2 \omega^2 - \mu i \omega (N+1) + 1} \right\|
  \end{split}
\end{equation}

差分が正になることを確かめる.各項の共通部分を(くくり出したテイで)削除すると
\begin{equation}
  \begin{split}
    &\left\| \frac{N}{-N^2 \omega^2 - \mu i \omega N + 1} \right\| - \left\| \frac{N+1}{-(N+1)^2 \omega^2 - \mu i \omega (N+1) + 1} \right\| \\
    = &  \frac{N}{\sqrt{(-N^2 \omega^2 + 1 )^2- (\mu \omega N)^2}}  -  \frac{N+1}{\sqrt{(-(N+1)^2 \omega^2 + 1 )^2- (\mu \omega (N+1))^2}}  \\
    = &  \frac{1}{\sqrt{(-N \omega^2 + N^{-1} )^2- (\mu \omega)^2}}  -  \frac{1}{\sqrt{(-(N+1) \omega^2 + (N+1)^{-1} )^2- (\mu \omega)^2}}  \\
  \end{split}
\end{equation}

分母を比較する.左の項の分母が小さくなると嬉しい.つまり,次の式が負になってほしい.
\begin{equation}
  \begin{split}
    \sqrt{(-N \omega^2 + N^{-1} )^2- (\mu \omega)^2}  -  \sqrt{(-(N+1) \omega^2 + (N+1)^{-1} )^2- (\mu \omega)^2}
  \end{split}
\end{equation}

各項は正であるため,大小関係は二乗しても変わらない.
\begin{equation}
  \begin{split}
    &((-N \omega^2 + N^{-1} )^2 - (\mu \omega)^2) - ((-(N+1) \omega^2 + (N+1)^{-1} )^2- (\mu \omega)^2) \\
    = & (N^2 \omega^4 - 2 \omega^2 + N^{-2} ) - ((N+1)^2\omega^4 -2 \omega^2 + (N+1)^{-2}) \\
    = & (-2N+1)\omega^4 + \frac{-2N-1}{N^2(N+1)^2}
  \end{split}
\end{equation}

$N$はサイズ数であるため,上式は明らかに負である.よって,確かめられた.


%---------------------

2. $\|C\|, \|T^{-1}\|, \|M_{01}\|$
\begin{equation}
  \begin{split}
    \|C\| =& \sup \left\lbrace \right. \\
    &  \sum_{t=0}^{2N-3} \left\| \frac{\mu i \omega (N+t) (a*a*a)_{N+t}}{3(-(N+t)^2 \omega^2 - \mu i \omega (N+t)+ 1)} \right\|,\\
    & 0,\\
    & \sum_{t=0}^{0} \left\| \frac{ \mu i \omega (N+t) (a*a)_{2N-2 + t}}{-(N+t)^2 \omega^2 - \mu i \omega (N+t)+ 1} \right\|,\\
    & \sum_{t=0}^{1} \left\| \frac{ \mu i \omega (N+t) (a*a)_{2N-3 + t} }{-(N+t)^2 \omega^2 - \mu i \omega (N+t)+ 1}\right\|,\\
    & \vdots , \\
    & \sum_{t=0}^{2N-3} \left\| \frac{ \mu i \omega (N+t) (a*a)_{1 + t} }{-(N+t)^2 \omega^2 - \mu i \omega (N+t)+ 1}\right\|,\\
    \left. \right\rbrace \\
    = & \sup \left\lbrace \right. \\
    &  \sum_{t=0}^{2N-3} \left\| \frac{\mu i \omega (N+t) (a*a*a)_{N+t}}{3(-(N+t)^2 \omega^2 - \mu i \omega (N+t)+ 1)} \right\|,\\
    & \sum_{t=0}^{2N-3} \left\| \frac{ \mu i \omega (N+t) (a*a)_{1 + t} }{-(N+t)^2 \omega^2 - \mu i \omega (N+t)+ 1}\right\|,\\
    \left. \right\rbrace&
  \end{split}
\end{equation}


\begin{equation}
  \begin{split}
    \|T^{-1}\| =& \sup \left\lbrace \right. \\
    & \left\|\sum_{t=0}^{2N} T_{0\ t}^{-1}\right\|, \\
    & \vdots \\
    & \left\|\sum_{t=0}^{2N} T_{2N\ t}^{-1}\right\|, \\
    \left. \right\rbrace&
  \end{split}
\end{equation}

\begin{equation}
  \begin{split}
    \|M_{01}\| =& \sup \left\lbrace \right. \\
    &\| 1 + \mu i \omega \sum_{t=0}^{2N-3} (-N+1 + t) (a*a)_{-2N+2+t} \| , \\
    &\| 1 + \mu i \omega \sum_{t=0}^{2N-4} (-N+2 + t) (a*a)_{-2N+2+t} \| , \\
    & \vdots \\
    &\| 1 + \mu i \omega \sum_{t=0}^{0} (N-1 + t) (a*a)_{-2N+2+t} \| , \\
    & 1 \\
    \left. \right\rbrace&
  \end{split}
\end{equation}

\subsubsection*{12/2}
プログラムの際に,周期解のインデックスをJuliaに合わせたら,計算結果が変わった?

\begin{equation}
  \begin{split}
    &ADF(\bar{x}) \\
    &= \begin{bmatrix}
      A_N & 0  & \cdots & \cdots  \\
      0 & \lambda_{N+1}^{-1} & & \mathbf{0} \\
      \vdots & & \lambda_{N+2}^{-1} & \\
      \vdots & \mathbf{0} & & \ddots
    \end{bmatrix}
    \begin{bmatrix}
      0 & 1  & \cdots & \cdots \\
      \vdots & & \vdots &\\
      \partial_w f_k & \cdots & \partial_{a_j}f_k & \cdots \\
      \vdots & & \vdots &
   \end{bmatrix}\\
    &= \begin{bmatrix}
      & \multicolumn{1}{c|}{A_N} & 0 & \cdots & \cdots  & \\ \cline{2-5}
      & \multicolumn{1}{c|}{0} & \lambda_{N+1}^{-1} & & \mathbf{0} & \\
      & \multicolumn{1}{c|}{\vdots} & & \lambda_{N+2}^{-1} & & \\
      & \multicolumn{1}{c|}{\vdots} & \mathbf{0} & & \ddots &
    \end{bmatrix}
    \begin{bmatrix}
      & \multicolumn{1}{c|}{DF^{(N)}(\bar{x})^{-1}} & 1  & \cdots & \cdots &\\ \cline{2-5}
      & \multicolumn{1}{c|}{\vdots} & & \vdots & &\\
      & \multicolumn{1}{c|}{\partial_w f_k} & \cdots & \partial_{a_j}f_k & \cdots &\\
      & \multicolumn{1}{c|}{\vdots} & & \vdots & &
    \end{bmatrix}\\
    &= \begin{bmatrix}
      & \multicolumn{1}{c|}{T^{-1}} & \mathbf{0} &  \\ \cline{2-3}
      & \multicolumn{1}{c|}{\mathbf{0}} & \Lambda &
    \end{bmatrix}
    \begin{bmatrix}
      & \multicolumn{1}{c|}{T} & M_{01} &  \\ \cline{2-3}
      & \multicolumn{1}{c|}{M_{10}} & M_{11} &
    \end{bmatrix}\\
    &= \begin{bmatrix}
      T^{-1}T & T^{-1}M_{01} \\
      \Lambda M_{10} & \Lambda M_{11}
    \end{bmatrix}
    = \begin{bmatrix}
      I & B \\
      C & D
    \end{bmatrix}
  \end{split}
  \label{eq:ADF-1}
\end{equation}

\begin{equation}
  \begin{split}
    &M_{01} =
    \begin{bmatrix}
      1 & 1 & \cdots \\
      0 & 0 & \cdots \\
      0 & 0 & \cdots \\
      \vdots & \vdots & \ddots \\
    \end{bmatrix}
  \end{split}
\end{equation}

\begin{equation}
  \begin{split}
    &M_{10} \\
    &= \begin{bmatrix}
      \partial_{\omega} f_{N+1} & \partial_{a_{1}} f_{N+1}  & \cdots & \partial_{a_{N-1}} f_{N+1} & 0\\
      \partial_{\omega} f_{N+2} & \partial_{a_{1}} f_{N+2}  & \cdots & \partial_{a_{N-1}} f_{N+2} &\partial_{a_{N}} f_{N+2}\\
      \vdots & \vdots & \cdots & \vdots & \vdots \\
      \partial_{\omega} f_{3N-3} & 0  & \cdots &  \partial_{a_{N-1}} f_{3N-3} & \partial_{a_{N}} f_{3N-3} \\
      0 & 0 & \cdots & 0 & \partial_{a_{N}} f_{3N-2} \\
      \vdots & \vdots  & \cdots & \vdots & 0 \\
      \vdots & \vdots & \cdots & \vdots & \vdots \\
    \end{bmatrix}
  \end{split}
\end{equation}

\begin{equation}
  \begin{split}
    &M_{11} \\
    &= \begin{bmatrix}
      \lambda_{N+1} & 0 & \cdots \\
      0 & \lambda_{N+2} & \cdots \\
      \partial_{a_{N+1}} f_{N+3} & 0 & \ddots \\
      \partial_{a_{N+1}} f_{N+4} & \partial_{a_{N+2}} f_{N+4} & \ddots \\
      \vdots & \vdots & \vdots \\
      \partial_{a_{N+1}} f_{3N-3} & 0 & \vdots \\
      \vdots & \vdots & \vdots \\
    \end{bmatrix}
  \end{split}