\chapter{はじめに}

今日では,コンピュータを用いることで,微分方程式など手計算では困難な数式問題を解析することができる.しかし,コンピュータは有限の数値のみを扱うことしかできないため,計算結果が正確にならず,出力した解には誤差が含まれる.そこで,問題と解とその誤差範囲を評価し保証するために,精度保証付き数値計算を用いる\cite{b1}.

精度保証付き数値計算に関する定理の一つに,Newton-Kantorovich型の定理を利用したradii polynomial approachがある.この定理は有限次元や無限次元を問わず, 非線形方程式や偏微分方程式など殆どの微分方程式に用いることができる.

本研究では,Banach空間の定義に用いられている$l_\omega^1$


本論文の構成は以下の通りである.2章では準備として関数解析の基本知識をまとめる.3章では既存手法である\rad{}の定理と証明,4章では\vdp{}方程式を対象として既存の\rad{}を用いた計算機での実装法を提示する.5章では,無限次元ガウスの消去法に関する知識をまとめる.6章では提案手法として,無限次元ガウスの消去法を\rad{}に適用させる手法を示す.7章では,6章で述べた手法の計算機を用いた検証の結果を記載する.8章では,本論文のまとめを行う.