\chapter{はじめに}
今日の理工学では, 理想化された物理モデルを微分方程式の数学モデルで表し解くことによって,

今日まで理工学においては,実在する現象から物理モデルをつくり,これから導かれる微分方程式で記述される数学モデルを解いてきた.
このことによって,実在する現象の予測や工学製品を設計することが可能になり,理工学の発展を促してきた.\cite{}

微分方程式の中でもとりわけ非線形微分方程式を解くことは,従来は難しいものであった.しかし,近年のコンピュータ性能の発達により,今日では計算機を利用して数値解析を行い,近似解を求めることができるようになった.これにより,微分方程式で記述されたモデルや現象を解析することが可能となった.

コンピュータを利用した計算は正確ではないため,出力された解には誤差が含まれる.精度保証付き数値計算は,方程式の真の解とその誤差範囲を保証した数値計算のことである.

radii polynomial approachは,Newton-Kantorovichの定理を精度保証付き数値計算に応用した定理である.

既存のradii polynomial approachでは,無限次元の計算箇所が有限次元に近似して計算されているため,大きな誤差が起こる原因になっている.

本研究では,無限次元ガウスの消去法を用いて,無限次元を有限次元に近似せずに計算を行う.このことで,radii polynomial approachによる精度保証ができる誤算の範囲を改善することを目的とする.

本論文の構成は,以下のとおりである.まず,
