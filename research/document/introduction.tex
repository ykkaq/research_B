\chapter{はじめに}

今日では,コンピュータを用いることで,微分方程式など手計算では困難な数式問題を解析することができる.しかし,コンピュータは有限の数値のみを扱うことしかできないため,計算結果が正確にならず,出力した解には誤差が含まれる.そこで,問題と解とその誤差範囲を評価し保証するために,精度保証付き数値計算を用いる\cite{b1}.

精度保証付き数値計算に関する定理の一つに,Newton-Kantorovich型の定理を利用したradii polynomial approachがある.この定理は有限次元や無限次元を問わず, 非線形方程式や偏微分方程式など殆どの微分方程式に用いることができる.

従来の\rad{}では,Banach空間を重み付き$l_1$空間より定義している.重みがあることにより,精度保証できる条件が厳しくなる問題がある.この問題は,重みを除いた$l_1$空間でBanach空間を定義することで,解決することができる.

ある微分方程式$F(x)=0$を,$l_1$空間を用いた\rad{}で精度保証するには,$l_1$空間を用いたBanach空間上で$DF(x)$が全単射であることが必要である.

そこで本研究では,重みを除いた$l_1$空間を用いたBanach空間上で,$DF(x)$が

本論文の構成は以下の通りである.2章では準備として関数解析の基本知識をまとめる.3章では既存手法である\rad{}の定理と証明,4章では\vdp{}方程式を対象として既存の\rad{}を用いた計算機での実装法を提示する.5章では,無限次元ガウスの消去法に関する知識をまとめる.6章では提案手法として,無限次元ガウスの消去法を\rad{}に適用させる手法を示す.7章では,6章で述べた手法の計算機を用いた検証の結果を記載する.8章では,本論文のまとめを行う.

\begin{comment}
\hrulefill

精度保証付き数値計算における重要な定理の一つに,nk型の定理を利用したradがある.この定理は,有限次元から無限次元まで幅広い問題に適用可能であり,非線形方程式や偏微分方程式など,様々な微分方程式の解の存在証明に利用されている.

しかし,従来のradには改善の余地が存在する.従来手法では,ノルム空間の定義に重み付きl1空間を採用しているが,これは通常のl1空間と比較して定理の適用範囲を制限する要因となっている.この制限は,特に[具体的な例や状況]において顕著である.

\hrulefill
\end{comment}
