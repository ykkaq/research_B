%ctrl+alt+b -> build, ctrl+alt+v => preview at new tab
%ctrl+click -> highlight line

\documentclass[11pt,a4paper,titlepage]{jsreport}

\usepackage{algorithmic}
\usepackage{amsmath}
\usepackage{amssymb,amsfonts,amsthm}
\usepackage{ascmac}
% \usepackage{amsmath}
\usepackage{bm}
\usepackage{caption}
\usepackage{cite}
\usepackage{comment}
% \usepackage[dvipdfmx]{color}
\usepackage{colortbl}
\usepackage{float}
\usepackage{graphicx}
% \usepackage[dvipdfmx]{graphicx}
\usepackage{mathtools}
\usepackage{multicol}
\usepackage{latexsym}
\usepackage[dvipdfmx]{pict2e}
\usepackage[ipaex]{pxchfon}
\usepackage{remreset}
\usepackage{tabularx}
\usepackage{textcomp}
\usepackage{underscore}
\usepackage{ulem}
\usepackage{url}
%\usepackage{subfiles}
\usepackage{wrapfig}
\usepackage[dvipsnames,dvipdfmx]{xcolor}
\usepackage{mdframed}
\usepackage{makecell}
\usepackage{SekineLabMacro}
%\usepackage{physics2}
% \usepackage{amsmath,amssymb}% 数式入力に数学必携のパッケージ読み込み
\usepackage{listings,jvlisting}
% \usepackage{here}
\lstset{%
    captionpos=t,
    % backgroundcolor=\color[rgb]{0.95,0.95,0.92},
    % commentstyle=\color{Green},
    % keywordstyle=\color{Magenta},
    % numberstyle=\tiny\color{gray},
    % stringstyle=\color{violet},1
    basicstyle=\ttfamily\footnotesize,
    breakatwhitespace=false,
    breaklines=true,
    keepspaces=true,
    numbers=left,
    numbersep=5pt,
    showspaces=false,
    showstringspaces=false,
    showtabs=false,
    tabsize=2,
    frame={tb},
    caption=ソースコード
}

\makeatletter
	\AtBeginDocument{
	\@removefromreset{lstlisting}{chapter}
	\def\thelstlisting{\arabic{lstlisting}}}
\makeatother
\allowdisplaybreaks[1]


%プログラムのソースコード関連はここまで

%\setcounter{secnumdepth}{4}
%\usepackage[margin=10truemm]{geometry}

\captionsetup[figure]{labelsep=space}
\captionsetup[table]{labelsep=space}

\mathtoolsset{showonlyrefs}
\theoremstyle{definition}
\newtheorem{dfn}{定義}
\newtheorem{thm}{定理}
\newtheorem{lmm}{補題}

\renewcommand{\qedsymbol}{$\blacksquare$}
\renewcommand{\proofname}{\textbf{証明}}
\renewcommand{\lstlistingname}{ソースコード}
\newcommand{\rad}{radii polynomial approach}
\newcommand{\nk}{Newton-Kantorovich}
\newcommand{\vdp}{van der Pol}
\newcommand{\infg}{無限次元ガウスの消去法}


\title{無限次元ガウスの消去法を用いた\ radii polynomial approachの改良}
\author{齋藤悠希}
\date{\today}

\begin{document}

\JYear{令和6年度}
\Kinds{卒業論文}
\StudentNumber{2131701}

\pagestyle{empty}
\maketitle
\tableofcontents
\clearpage
\pagestyle{plain}

\chapter{はじめに}
今日の理工学では, 理想化された物理モデルを微分方程式の数学モデルで表し解くことによって,

今日まで理工学においては,実在する現象から物理モデルをつくり,これから導かれる微分方程式で記述される数学モデルを解いてきた.
このことによって,実在する現象の予測や工学製品を設計することが可能になり,理工学の発展を促してきた.\cite{}

微分方程式の中でもとりわけ非線形微分方程式を解くことは,従来は難しいものであった.しかし,近年のコンピュータ性能の発達により,今日では計算機を利用して数値解析を行い,近似解を求めることができるようになった.これにより,微分方程式で記述されたモデルや現象を解析することが可能となった.

コンピュータを利用した計算は正確ではないため,出力された解には誤差が含まれる.精度保証付き数値計算は,方程式の真の解とその誤差範囲を保証した数値計算のことである.

radii polynomial approachは,Newton-Kantorovichの定理を精度保証付き数値計算に応用した定理である.

既存のradii polynomial approachでは,無限次元の計算箇所が有限次元に近似して計算されているため,大きな誤差が起こる原因になっている.

本研究では,無限次元ガウスの消去法を用いて,無限次元を有限次元に近似せずに計算を行う.このことで,radii polynomial approachによる精度保証ができる誤算の範囲を改善することを目的とする.

本論文の構成は,以下のとおりである.まず,

\include{preparation}

\chapter{Newton Kantorovichの定理を用いた精度保証付き数値計算}
$X,Y$をBanach空間とし,非線形方程式
\begin{equation*}
  F(u)=0,\ u\in X
\end{equation*}
を求める問題を考える.この問題に対して近似解の品質保証を行う,定理の1つを以下に説明する.

\section{radii polynomial approach \cite{github}}
\begin{thm}[Newton-Kantorovich type argument]
  有界線形作用素$A^\dagger \in \mathcal{L}(X,Y),\ A\in \mathcal{L}(Y,X)$を考え,作用素$F:X\rightarrow Y$が$C^1$-Fr\'{e}chet微分可能とする.また,$A$が単射で$AF:X\rightarrow X$とする.いま,$\bar{x} \in X$に対して%\overline{x}は,F(x) \approx 0となる近似解とする.
  \begin{align*}
    \|AF(\bar{x})\|_X                             & \leq Y_0                                             \\
    \|I-A^\dagger\|_{\mathcal{L}(X)}              & \leq Z_0                                             \\
    \|A(DF(\bar{x})-A^\dagger)\|_{\mathcal{L}(X)} & \leq Z_1                                             \\
    \|A(DF(b)-A^\dagger)\|_{\mathcal{L}(X)}       & \leq Z_2(r)r,\ \forall b \in \overline{B(\bar{x},r)} \\
  \end{align*}
  が成り立つとする.このとき
  \begin{equation*}
    p(r) := Z_2(r)r^2 - (1-Z_1-Z_0)r+Y_0
  \end{equation*}
  をradii polynomialといい,もし$p(r_0)<0$となる$r_0>0$が存在すれば,$F(\tilde{x})=0$をみたす$\tilde{x}$が$\overline{B(\tilde{x},r_0)}$内に一意存在する.
\end{thm}

\begin{proof}
  写像$T$が閉球$\overline{B(\bar{x},r_0)}$において縮小写像になることを示す.

  まず,
  \begin{equation*}
    DT(x)=I-ADF(x),\ x\in X
  \end{equation*}
  から,任意の$x\in \overline{B(\bar{x},r_0)}$に対して,仮定より
  \begin{align*}
    \|DT(x)\|_{\mathcal{L}(X)} & = \|I-ADF(x)\|_{\mathcal{L}(X)}                                                                                                  \\
                               & \leq \|I-AA^\dagger\|_\mathcal{L(X)} + \|A(A^\dagger-DF(\bar{x}))\|_{\mathcal{L}(X)} + \|A(DF(\bar{x})-DF(x))\|_{\mathcal{L}(X)} \\
                               & \leq Z_0 + Z_1 + Z_2(r_0)r_0
  \end{align*}
\end{proof}

そして,任意の$x\in \overline{B(\bar{x},r_0)}$に対して,平均値の定理より,
\begin{align*}
  \|T(x)-\bar{x}\|_X & = \|T(x)-T(\bar{x})\|_X + \|T(\bar{x}) - \bar{x}\|_X                                           \\
                     & \leq \sup_{b\in B(\bar{x},r_0)} \|DT(b)\|_{\mathcal{L}(X)} \|x-\bar{x}\|_X + \|AF(\bar{x})\|_X \\
                     & \leq (Z_0 + Z_1 + Z_2(r_0)r_0) r_0 + Y_0                                                       \\
                     & = p(r_0)+r_0
\end{align*}

よって,$p(r_0)<0$ならば,$\|T(x)-\bar{x}\|_X<r_0$.よって,$T(x)\in\overline{B(\bar{x},r_0)}$.

次に,$X$の距離として,$d(x,y):=\|x-y\|_X,\ x,y\in X$と定義すると,
\begin{align*}
  d(T(x),T(y))\leq \sup_{b\in B(\bar{x},r_0)} \|DT(b)\|_{\mathcal{L}(X)} d(x,y) \\
  \leq (Z_0 + Z_1 + Z_2(r_0)r_0) d(x,y),\ x,y\in \overline{B(\bar{x},r_0)}
\end{align*}

このとき,$p(r_0)<0$より,
\begin{equation*}
  Z_0 + Z_1 + Z_2{r_0}r_0 + \frac{Y_0}{r_0} < 1
\end{equation*}

よって,$k:=Z_0+Z_1+Z_2(r_0)r_0<1$となるため,$T$は$\overline{B(\bar{x},r_0)}$において縮小写像になる.

したがって,Banachの不動点定理より,ただ1つの不動点$\tilde{x}$が$\overline{B(\bar{x},r_0)}$内に存在し,$A$の単射性により,この不動点は写像$F$の零点となる,$\mathrm{i.e.},F(\tilde{x})=0$

% 修正:先にDFを表記しろ.
\section{Newton-Kantorovichの定理の特徴\cite{r1}}
\begin{thm}[Newton-Kantorovichの定理の亜種]
  \label{thm:Newton-Kantorovichの定理の亜種}
  $X$と$Y$をBanach空間とする.$F:X\rightarrow Y$を与えられた作用素とし,$\bar{x}\in X$を与えられているとする.$A\in \mathcal{L}(Y,X)$とする.$F$は$\bar{x}$でFr\'{e}chet微分可能とし,$DF(\bar{x})$と表記する.$DF(\bar{x})$は全射であるとする.$\eta,\delta$を不等式
  \begin{align*}
    \|AF(\bar{x})\|_X &\leq \eta \\
    \|I-A(DF(\bar{x}))\|_{\mathcal{L}(X)} &\leq \delta < 1
  \end{align*}
  を満たす定数とする.

  $\overline{B(0,\frac{2\eta}{1-\delta})} = \{z\in X \mid \|z\|_X \leq \frac{2\eta}{1-\delta} \}$とし,定数$K$を不等式
  \begin{equation*}
    \|A(DF(\bar{x})-DF(\bar{x}+z))\|_\mathcal{{L}(X)} \leq K,\ z\in \overline{B\left(0,\frac{2\eta}{1-\delta}\right)}
  \end{equation*}
  を満たす定数とする.もし,$2K+\delta\leq 1$ならば.
  \begin{equation*}
    \|\tilde{x}-\bar{x}\|_X \leq \frac{\eta}{1-(K+\delta)} := p
  \end{equation*}
  に対し,真の解は$\bar{B}(\bar{x},p)$内に存在する.その上,$\bar{B}\left(\bar{x},\frac{2\eta}{1-\delta}\right)$内で一意である.
\end{thm}

\begin{proof}
  まず,$AとDF(\bar{x})$が全単射であることを示す.$DF(\bar{x})$は$Fの\bar{x}$におけるFr\'{e}chet微分であるから,$DF(\bar{x})$は$\mathcal{L}(X,Y)$に属する.さらに,Neumann級数の定理(\ref{thm:Neumann級数})と仮定$\|I-ADF(\bar{x})_{\mathcal{L}(X)}\leq \delta 1$より,$ADF(\bar{x})$は全単射である.その上,定理(\ref{thm:Banachの全単射})より,$DF(\bar{x})$は単射であり,$A$は全射である.また,定理の仮定より,$DF(\bar{x})$は全単射となるため,逆作用素$DF(\bar{x})^{-1}$が存在し,$\mathcal{L}(Y,X)$に属する.また$A$の単射性について,定理(\ref{thm:線形作用素に対する単射性(1)})と逆作用素$DF(\bar{x})^{-1}$を用いて
  \begin{equation*}
    A\phi = 0 \Rightarrow ADF(\bar{x}) DF(\bar{x})^{-1}\phi = 0 \Rightarrow \phi = 0
  \end{equation*}
  となるため,$A$も全単射である.

  続いて,Banachの不動点定理(\ref{thm:Banachの不動点定理})を用いて解の存在を示す.まず,作用素方程式$F(x)=0$を不動点方程式に変形すr.$w:=\tilde{x}-\bar{x}$とする.$A$が単射であるため,
  \begin{align*}
     & F(\tilde{x})=0                                                    \\
     & \Leftrightarrow w = w-AF(\tilde{x} + w)                           \\
     & \Leftrightarrow w = -AF(\bar{x}) + w - A(F(\bar{x}+w)-F(\bar{x})) \\
  \end{align*}
  $\mathcal{T}: V \rightarrow V$を
  \begin{equation*}
    \mathcal{T}(w) := -AF(\bar{x}) + w - A(F(\bar{x}+w)-F(\bar{x}))
  \end{equation*}
  となる非線形作用素とし,不動産方程式$w=\mathcal{F}(w)$の解の存在をBanachの不動点定理($\ref{thm:Banachの不動点定理}$)を用いて示す.

  Banachの不動点定理(\ref{thm:Banachの不動点定理})では$M$を決めて,$\mathcal{T}がMからMへ$の縮小写像になることを確認しなければならない.特に,ポイントの1つは$\mathcal{T}$の定義域を$M$としたときに,値域が$M$に含まれることを確かめなければならない.すなわち,$\mathcal{T}(M) \ subset M$となるように$M$を選ぶことが重要である.この定理では,$M=\overline{B(0,\rho)},\ \rho = \frac{\eta}{1-(K+\delta)}$と選ぶ.まず,$M$として選んだ閉球$\overline{B(0,\rho)}$と定理で出てくるもう1つの閉球$B\left(0, \frac{\eta}{1-\delta}\right)$の関係性を確認する.定理の仮定より$1-\delta>0$と$1-\left(\delta + 2K\geq 0\right)$を持つため,
  \begin{align*}
    \rho - \frac{2\eta}{1-\delta} & = \frac{\eta(1-\delta)}{(1-(\delta+K))(1-\delta)}-\frac{2\eta(1-(\delta +K))}{(1-(\delta+K))(1-\delta)} \\
                                  & = \frac{\eta((1-\delta)-2(1-\delta)+2K)}{(1-(\delta+2K))(1-\delta)}                                     \\
                                  & = \frac{-\eta(1-\delta-2K)}{(1-(\delta+K))(1-\delta)}
    \leq 0
  \end{align*}
  となる.よって,$\rho\leq\frac{2\eta}{1-\delta}$から$\overline{B(\bar{x},\rho)}\subset \overline{B\left(0,\rho\right)}$となる.

  次に,$\mathcal{T}(\overline{B(0,\rho)}) \subset \overline{B(0,\rho)}$を示す.任意の$w\in \overline{B(0,\rho)}$に対し,Fr\'{e}chet微分とBochner積分に対する微分積分学の定理を用いることで
  \begin{align*}
    \|\mathcal{T}(w)\|_X & \leq \|AF(\bar{x})\|_X + \|w-A(F(\bar{x}+w)-F(\bar{x}))\|_X                                                             \\
                         & \leq \eta + \|w - R\int_0^1 DF((1-t)\bar{x})+t(\bar{x}+w)wdt\|_X                                                        \\
                         & \leq \eta + \int_0^1 \|w - ADF(\bar{x}+tw))w\|_X dt                                                                     \\
                         & \leq \eta + \int_0^1 \|I - ADF(\bar{x}+tw)\|_{\mathcal{L}(V,V)}\|w\|_X dt                                               \\
                         & \leq \eta + \int_0^1 \|A(DF(\bar{x})-DF(\bar{x}+tw))\|_{\mathcal{L}(X)} + \|I-ADF(\bar{x})\|_{\mathcal{L}(X)}\|w\|_X dt \\
                         & \leq \eta + \int_0^1 (\|A(DF(\bar{x})-DF(\bar{x}+tw))\|_{\mathcal{L}(X)} +m) \|w\|_{X}dt                                \\
  \end{align*}
  を得る.さらに,$t\in[0,1]$に対し$tw\in\overline{B(0,\rho)}\subset\overline{B\left(0,\frac{2\eta}{1-\delta}\right)}$となるため,定理の仮定$\|R(DF(\bar{x})-DF(\bar{x}+tw))\|_{\mathcal{L}(X)}\leq K$から,
  \begin{align*}
    \|\mathcal{T}(w)\|_X & \leq \eta + (K+\delta)\|w\|_X                                             \\
                         & \leq \eta + (K+\delta)\rho                                                \\
                         & = \eta + \frac{\eta(K+\delta)}{1-(K+\delta)}                              \\
                         & = \frac{\eta-\eta(\delta+K)}{1-(\delta+K)}+{\eta(K+\delta)}{1-(\delta+K)} \\
                         & = \rho
  \end{align*}
  となる.よって,任意の$w\in \overline{B(0,\rho)}$に対して$\|\mathcal{T}(w)\|_X \leq \rho$となることから,$\mathcal{T}\left(\overline{B(0,\rho)}\right) \subset \overline{B(0,\rho)}$となる.

  次に,$\mathcal{T}:\overline{B(0,\rho)}\rightarrow \overline{B(0,\rho)}$が縮小写像になることを確認する.すなわち
  \begin{equation*}
    \|\mathcal{T}(w_1) - \mathcal{T} (w_2)\|_X \leq k \|w_1-w_2\|_X,\ \forall w_1,w_2\in \overline{B(0,\rho)}
  \end{equation*}
  となる定数$k$が1未満になることを確認しなければならない.この定理では,一意性の範囲を広げるために,$\overline{B(0,\rho)}\subset \overline{B\left(\frac{2\eta}{1-m}\right)}$であることを用いて
  \begin{equation*}
    \|\mathcal{T}(w_1) - \mathcal{T} (w_2)\|_X \leq k \|w_1-w_2\|_X,\ \forall w_1,w_2\in \overline{B\left(0,\frac{2\eta}{1-\delta}\right)}
  \end{equation*}
  となる定数$k$が$1$未満になることを確かめる.その上,任意の$\overline{B(0,\rho)}\in\overline{B\left(\frac{2\eta}{1-m}\right)}$に対し,
  \begin{align*}
    \|\mathcal{T}(w_1) - \mathcal{T} (w_2)\|_X & = \|w_1-w_2-A(D(\bar{x}+w_1)-F(\bar)+w_2)\|_X                                                         \\
                                               & = \|(w_1-w_2)-A\int_{0}^{1}DF(\bar{x}+(1-t)w_2+tw_1)\|_{\mathcal{L}(X)}dt\|w_1-w_2\|_X                \\
                                               & \leq \int_{0}^{1} \|I-ADF(\bar{u}+(1-t)w_2+tw_1)\|_{\mathcal{L}(X)}dt\|w_1-w_2\|_X                    \\
                                               & \leq \int_{0}^{1} \|A(DF(\bar{x})-DF(\bar{x}+(1-t)w_2+tw_1)\|_{\mathcal{L}(X)}+\delta)dt\|w_1-w_2\|_X \\
  \end{align*}
  となる.任意の$w_1,w_2\in\overline{B\left(0,\frac{2\eta}{1-\delta}\right)}$と$0\leq t \leq 1$に対し,$\|(1-t)w_2+tw_1\|_X\leq(1-t)\|w_2\|_x+t\|w_1\|_X\leq \frac{2\eta}{1-\delta}$となるため,
  \begin{equation*}
    \|\mathcal{T}(w_1) - \mathcal{T} (w_2)\|_X \leq (K+\delta)\|w_1-w_2\|_X
  \end{equation*}
  となる.その上,仮定$2K+\delta\leq 1かつ\delta<1$より$K+\delta < 1$も満たすため,$\mathcal{T}は\overline{B(0,\rho)}から\overline{B(0,\rho)}$への縮小写像となる.よって,Banachの不動点定理(\ref{thm:Banachの不動点定理})より不動点方程式$w=\mathcal{T}(w)$の$F(u)=0$を満たす解が$\overline{B(0,\rho)}$に一意に存在する.$w=\tilde{x}-\bar{x}$であったことから,作用素方程式$F(x)=0$を満たす解が$\overline{B(\bar{x}),\rho}$内に一意に存在する.

  最後に,$\overline{B\left(\bar{x},\frac{2\eta}{1-\delta}\right)}$内で一意に存在することを示す.$\overline{B\left(0,\frac{2\eta}{1-\delta}\right)}$内に不動点方程式$w=\mathcal{T}(w)$の解が2つあったとする.すなわち,2つの解$\tilde{z_1},\tilde{w_2}\in\overline{\bar{x},\frac{2\eta}{1-\delta}}$とし,$\tilde{x_1}=\mathcal{T}(\tilde{w_1})$と$\tilde{x_2}=\mathcal{T}(\tilde{w_2})$を満たすとする.そのとき,
  \begin{equation*}
    \|\tilde{w_1}-\tilde{w_2}\|_X = \|\mathcal{T}(\tilde{w_1})-\mathcal{T}(\tilde{w_2})\|_X \leq (K+m)\|\tilde{w_1}-\tilde{w_2}\|_X
  \end{equation*}
  となる.ここで,$2K+m<1$であるため,不等式を満たすものは$\tilde{w_1}=\tilde{w_2}$の場合のみである.すなわち,不動点方程式$w=\mathcal{T}(w)$を満たす解は$\overline{B\left(\bar{x},\rho\right)}$内に一意である.よって,作用素方程式$F(x)=0$を満たす解が$\overline{B(\bar{x},\frac{2\eta}{1-\delta})}$内に一意に存在する.

\end{proof}
%-----------------

\newpage
\chapter{既存のvan der Pol方程式の精度保証付き数値計算\cite{github}}

\section{van der Pol方程式}
van der Pol方程式とは,ある発振現象をもつ電気回路の方程式であり,以下のように表す.
\begin{equation*}
  \frac{d^2x}{dt^2} - \mu (1-x^2)\frac{dx}{dt}+x=0
\end{equation*}

未知関数は$x(t)$で,$\mu >0$は非線形の減衰の強さを表すパラメータである.


\section{フーリエ・スペクトル法}
フーリエ・スペクトル法の計算に必要な参照軌道をまず得るために,van der Pol方程式の解を数値計算する.まず,van der Pol方程式を次の連立常微分方程式系にしてODEソルバーで数値を計算する.

\begin{equation*}
  \begin{cases}
    \frac{dx}{dt} = y \\
    \frac{dy}{dt} = \mu (1-x^2)y-x
  \end{cases}
\end{equation*}

初期値$x(0)=0,\ y(0)=2$とし,$\mu=1$の時の数値計算を実行する.周期解をフーリエ級数で表し,その係数と周期を求めるため,Van der Pol方程式の周期解の周期を大まかに求める.得た近似周期軌道と近似周期を使って,起動のフーリエ補完を計算する.van der Pol方程式は,$\dot{x}=\frac{dx}{dt}$とおくと,以下のように表すことができる.
\begin{equation*}
  \ddot{x} - \mu (1-x^2)\dot{x} - x = 0
\end{equation*}
後の計算のために,式を整理すると,
\begin{equation*}
  \ddot{x} - \mu \dot{x} + \frac{\mu}{3} \left(\dot{x}^3\right) + x = 0
\end{equation*}

周期解$x(t)$を周期$L$の周期関数とし,$\omega = \frac{2\pi}{L}$とおくと,$x(t)$とその微分やべき乗はフーリエ級数を使って,
\begin{align*}
  x(t)                 & = \sum_{k\in\mathbb{N}} a_k \exp(ik \omega t)                \\
  \frac{d(x)}{dt}      & = \sum_{k\in\mathbb{N}} (ik \omega) a_k \exp(ik \omega t)    \\
  \frac{d^2x(t)}{dt^2} & = \sum_{k\in\mathbb{N}} (-k^2\omega^2) a_k \exp(ik \omega t) \\
  x(t)^3               & = \sum_{k\in\mathbb{N}} (a*a*a)_k \exp(ik \omega t)
\end{align*}
と書くことができる.ここで

\begin{equation*}
  (a*a*a)_k = \sum_{\substack{k_1+k_2+k_3=k\\k_i\in\mathbb{Z}}} a_{k_1}a_{k_2}a_{k_3},\ k\in\mathbb{Z}
\end{equation*}

は3次の離散畳み込みである.

そしてフーリエ係数に関する式を立てる.係数$a=\left(a_k\right)_{k\in\mathbb{Z}}$に対して,van der Pol方程式に求めたフーリエ級数を代入すると,
\begin{equation*}
  f_k(a) := -k^2\omega^2 a_k - \mu ik \omega a_k + \frac{\mu}{3} (ik \omega)(a*a*a)_k + a_k
\end{equation*}
となる点列$(f_k(a))_{k\in\mathbb{Z}}$が得られる.そして,各$k\in\mathbb{Z}$について
\begin{equation*}
  f_k(a) = 0
\end{equation*}
となる点列$a$があれば,van der Pol方程式の解のフーリエ係数になる.未知数は周期数$\omega$と点列$a$であり,これらを並べて$x=(\omega,a)$と書くことにする.未知数$x$に対して,$f_k(a)=0$という方程式だけでは不逞な方程式になるため,解の形を1つに定めることができない.そこで,位相条件
\begin{equation*}
  \eta(a) = \sum_{|k|\in N} a_k-\eta_0 = 0,\ \eta_0\in\mathbb{R}
\end{equation*}
を加える.この条件は,$x(t)$の切片$x(0)=\eta_0$を表している.最終的にvan der Pol方程式の周期解の球解は次の代数方程式を解くことに帰着される.
\begin{equation*}
  F(x) := \begin{bmatrix}
    \eta(a) \\
    \left(f_k(a)_{k\in\mathcal{Z}}\right)
  \end{bmatrix}
\end{equation*}

以下,この零点探索問題$F(x)=0$についてNewton法で解を得ることを考える.まず,$N$をフーリエ係数の打ち切り番号(最大波数:$N-1$)とし,周期解の近似を次のように構成する.
\begin{equation*}
  x_N(t) = \sum_{|k|<N} \bar{a}_k \exp(ik\omega t)
\end{equation*}

このとき,フーリエ係数と(近似)周期をならべた
\begin{equation*}
  \bar{x} = \left(\bar{\omega},\bar{a}_{-N+1},\cdots,\bar{a}_{N-1}\right) \in \mathbb{C}^{2N}
\end{equation*}
を近似解と呼ぶ.近似解$\bar{x}$の項数は$2N$個.そして,$f_k(a)=0$を$|k|<N$の範囲で打ち切る方程式
\begin{equation*}
  F^{(N)}\left(x^{(N)}\right) = \begin{bmatrix}
    \eta\left(a^{(N)}\right) \\
    \left(f_k(a^{(N)})\right)_{|k|<N}
  \end{bmatrix}
  = 0
\end{equation*}
を考える.ここで$a^{(N)}=\left(a_k\right)_{|k|<N},\ x^{(N)}=\left(\omega, a^{(N)}\right)$をそれぞれ表し,$F^{(N)}:\mathbb{C}^{2N}\rightarrow\mathbb{C}^{2N}$である.したがって$F^{(N)}\left(x^{(N)}\right)=0$という有限次元の非線形方程式を解くことで,近似解$\bar{x}$を得られる.

実際にNewton法を用いて,周期化の数値計算を行っていく.Newton法では,ある適当な初期値$x_0$を最初に定め,以下の反復計算によって計算できる.
\begin{equation*}
  x_{n+1} = x_n - DF^{(N)}\left(x_n\right)^{-1} F^{(N)}\left(x_n\right),\ n=0,1,\cdots
\end{equation*}
このことから,$DF^{(N)}\left(x_n\right)^{-1}$と$F^{(N)}\left(x_n\right)$を計算することができれば,近似解を得ることができる.はじめに,離散畳み込みの関数を用意する.

$F^{(x)}(X^{(N)})$のヤコビ行列は
\begin{equation*}
  DF^{(N)}\left(x^{(N)}\right)=\begin{bmatrix}
    \multicolumn{1}{c|}{0}              & 1      & \cdots            & 1      \\ \hline
    \multicolumn{1}{c|}{\vdots}         &        & \vdots            &        \\
    \multicolumn{1}{c|}{\partial_w f_k} & \cdots & \partial_{a_j}f_k & \cdots \\
    \multicolumn{1}{c|}{\vdots}         &        & \vdots            &        \\
  \end{bmatrix} \in \mathbb{C}^{2N\times 2N}\ \left(|k|,|j|<N\right)
\end{equation*}
ここで,

\begin{align*}
   & \begin{cases}
       \partial_w f_k = (-2k^2\omega - \mu ik)a_k + \frac{\mu ik}(a*a*a)_k                     & (|k|<N) \\
       \partial_{a_j} f_k = (k^2\omega^2 - \mu ik\omega)\delta_{k_j} + \mu ik\omega(a*a)_{k-j} & (|k|<N)
     \end{cases}
  \\
   & \delta_{k_j} = \begin{cases}
                      1 & (k=j)     \\
                      0 & (k\neq j)
                    \end{cases}
\end{align*}

である.ヤコビ行列の各要素との対応は
\begin{align*}
   & \left(DF^{(N)}\left(x^{(N)}\right)\right)_{l,m}                                                                                        \\
   & = \begin{cases}
         0                   & \left(l=m=1\right)                                                                                             \\
         1                   & \left(l=1,\ m= 2,\cdots,2N\right)                                                                              \\
         \partial_\omega f_k & \left(l=2,\cdots,2N,\ m=1, \ \mathrm{i.e.},\ l=k+N+1 \ \mathrm{for} \ |k|<N\right)                             \\
         \partial_{a_j}f_k   & \left(l,m=2,\cdots,2N, \ \mathrm{i.e.}, l=k+N+1 \ \mathrm{for} \ |k|<N,\ m=j+N+1 \ \mathrm{for} \ |j|<N\right)
       \end{cases}
\end{align*}


\section{重み付き空間と作用素の決定}
\begin{dfn}[許容重み]
  \label{dfn:許容重み}
  点列$\omega = \left(\omega_k\right)_{k\in\mathbb{Z}}$について,
  \begin{align}
    \omega_k>0,\                           & \forall k \in \mathbb{Z}   \\
    \omega_{n+k} \leq \omega_n \omega_k,\  & \forall n,k\in\mathbb{Z}
  \end{align}
  が成立するとき,許容重み(admissible wights)であるという.
\end{dfn}

\begin{thm}[重み付き$l^1$空間]
  \label{thm:重み付きl1空間}
  \begin{equation*}
    l_\omega^1 := \left\{ a = (a_k)_{k\in\mathbb{Z}}:a_k\in\mathbb{C},\ \|a\|_\omega:=\sum_{k\in\mathbb{Z}}|a_k|\omega_k < \infty\right\}
  \end{equation*}

  点列$\omega$を許容重みとする.$\left(l_m^1, *\right)$は可換なBanach環となる.すなわち,点列$a,b\in l^1_m$として,離散畳み込み$*$に対して
  \begin{equation*}
    \|a*b\|_\omega \leq \|a\|_\omega\|b\|_\omega
  \end{equation*}
  が成立する.
\end{thm}

\begin{proof}
  $a,b\in l_{\omega}^1$として,
  \begin{equation*}
    \begin{split}
      \|a*b\|_\omega &= \sum_{k\in\mathbb{Z}} \|(a*b)\|\omega_k \\
      &= \sum_{k\in\mathbb{Z}} \left\|\sum_{\substack{k_1+k_2=k \\ k_1,k_2\in\mathbb{Z}}}a_{k_1} a_{k_2}\right\| \omega_k \\
      &\leq \sum_{k\in\mathbb{Z}} \left(\sum_{\substack{k_1+k_2=k \\ k_1,k_2\in\mathbb{Z}}}|a_{k_1}||a_{k_2}|\omega_k\right)\\
      &\leq \sum_{k\in\mathbb{Z}} \left(\sum_{\substack{k_1+k_2=k \\ k_1,k_2\in\mathbb{Z}}}|a_{k_1}|\omega_{k_1}|a_{k_2}|\omega_{k_2}\right)\\
      &\leq \left(\sum_{k_1\in\mathbb{Z}}|a_{k_1}|\omega_{k_1}\right) \left(\sum_{k_\in\mathbb{Z}}|a_{k_2}|\omega_{k_2}\right) \\
      &= \|a\|_\omega\|b\|_\omega
    \end{split}
  \end{equation*}
\end{proof}

\begin{dfn}[Banach空間$X$]
  \label{dfn:Banach空間X-重み付きl1空間}
  Banach空間$X$を次のように定める.はじめに重み付き$l^1$空間を重み$w_k=\nu^{|k|}\ (\nu=1.05)$として次のように定める.
  \begin{equation*}
    l_{\nu}^1 := \left\{ a = (a_k)_{k\in\mathbb{Z}}:a_k\in\mathbb{C},\ \|a\|_{w} := \sum_{k\in\mathbb{Z}}|a_k|\nu^{|k|} < \infty\right\}
  \end{equation*}

  そして,検証に用いいる関数空間$X$はBanach空間となる.
  \begin{equation*}
    X:=\mathbb{C}\times l_{\nu}^1,\ x=(\omega,a),\ \omega \in \mathbb{C},\ a\in l_{\nu}^1
  \end{equation*}
  と定め,そのノルムを
  \begin{equation*}
    \|x\|_X:=\max \{|\omega|,\ \|a\|_w \}
  \end{equation*}
  として,定義する.このとき,$X$はBanach空間となる.
\end{dfn}

\begin{dfn}[ヤコビ行列$DF^{(N)}\left(x^{(N)}\right)$]
  $F^{(N)}\left(x^{(N)}\right)$のヤコビ行列は,次のような形をしている.
  \begin{equation*}
    DF^{(N)}\left(x^{(N)}\right)=\begin{bmatrix}
       & \multicolumn{1}{c|}{0}              & 1      & \cdots            & 1      & \\ \cline{2-5}
       & \multicolumn{1}{c|}{\vdots}         &        & \vdots            &        & \\
       & \multicolumn{1}{c|}{\partial_w f_k} & \cdots & \partial_{a_j}f_k & \cdots & \\
       & \multicolumn{1}{c|}{\vdots}         &        & \vdots            &        & \\
    \end{bmatrix}
    \in \mathbb{C}^{2N\times 2N},\ \left(|k|,|j|<N\right)
  \end{equation*}
  ここで,
  \begin{align*}
     & \begin{cases}
         \partial_w f_k = (-2k^2\omega - \mu ik)a_k + \frac{\mu ik}{3}(a*a*a)_k                     & (|k|<N) \\
         \partial_{a_j} f_k = (k^2\omega^2 - \mu ik\omega)\delta_{k_j} + \mu ik\omega(a*a)_{k-j} & (|k|<N)
       \end{cases}
    \\
     & \delta_{k_j} = \begin{cases}
                        1 & (k=j)     \\
                        0 & (k\neq j)
                      \end{cases}
  \end{align*}
\end{dfn}
である.ヤコビ行列の各要素の伏せ字とその対応は
\begin{align*}
   & \left(DF^{(N)}\left(x^{(N)}\right)\right)_{l,m}                                                                                        \\
   & = \begin{cases}
         0                   & \left(l=m=1\right)                                                                                             \\
         1                   & \left(l=1,\ m= 2,\cdots,2N\right)                                                                              \\
         \partial_\omega f_k & \left(l=2,\cdots,2N,\ m=1, \ \mathrm{i.e.},\ l=k+N+1 \ \mathrm{for} \ |k|<N\right)                             \\
         \partial_{a_j}f_k   & \left(l,m=2,\cdots,2N, \ \mathrm{i.e.}, l=k+N+1 \ \mathrm{for} \ |k|<N,\ m=j+N+1 \ \mathrm{for} \ |j|<N\right)
       \end{cases}
\end{align*}

\begin{dfn}[作用素$A^\dagger,A$の定義]
  まず,Banach空間$X=\mathbb{C}\times l_\nu^1$から$Y=\mathbb{C}\times l_\nu^1,\ (\nu^\prime<\nu)$と設定し,$A^\dagger$を$A^\dagger\in (X,Y)$として,$b=(b_0,b_1)\in \mathbb{C}\times l_\nu^1=X$に対して,$A^\dagger b=\left(\left(A^\dagger b\right)_0,\left(A^\dagger b\right)_1\right)$と作用素するように定義する.ここで,$A^\dagger$を形式的に見ると,
  \begin{equation*}
    A^\dagger =
    \begin{bmatrix}
       & \multicolumn{1}{c|}{0}              & 1      & \cdots            & 1      & \cdots           & 0               & \cdots           & \\ \cline{2-8}
       & \multicolumn{1}{c|}{\vdots}         &        & \vdots            &        &                  &                 &                  & \\
       & \multicolumn{1}{c|}{\partial_w f_k} & \cdots & \partial_{a_j}f_k & \cdots &                  & \text{\huge{0}} &                    \\
       & \multicolumn{1}{c|}{\vdots}         &        & \vdots            &        &                  &                 &                    \\
       & \multicolumn{1}{c|}{\vdots}         &        &                   &        & \lambda_N        &                 & \text{\Large{0}} & \\
       & \multicolumn{1}{c|}{0}              &        & \text{\huge{0}}   &        &                  & \lambda_{N+1}   &                    \\
       & \multicolumn{1}{c|}{\vdots}         &        &                   &        & \text{\Large{0}} &                 & \ddots
    \end{bmatrix}
    =
    \begin{bmatrix}
       & \multicolumn{1}{c|}{0}                    & A_{a,0}^\dagger & \\ \cline{2-3}
       & \multicolumn{1}{c|}{A_{\omega,1}^\dagger} & A_{a,0}^\dagger &
    \end{bmatrix}
  \end{equation*}

  このことから,$A^\dagger b$は
  \begin{equation*}
    A^\dagger b = \begin{bmatrix}
      0                    & A^\dagger_{a,0} \\
      A^\dagger_{\omega,1} & A^\dagger_{a,1}
    \end{bmatrix}
    \begin{bmatrix}
      b_0 \\
      b_1
    \end{bmatrix}
    = \begin{bmatrix}
      A_{a,0}^\dagger b_1 \\
      A_{\omega,1}^\dagger b_0 + A_{a,1}^\dagger b_1
    \end{bmatrix}
    =: \begin{bmatrix}
      \left(A^\dagger b\right)_0 \\
      \left(A^\dagger b\right)_1
    \end{bmatrix}
  \end{equation*}
  と表すことができ,
  \begin{align*}
    \left(A^\dagger b\right)_0                & = \sum_{|k|<N} (b_1)_k                                                                                                             \\
    \left(\left(A^\dagger b\right)_1\right)_k & := \begin{cases}
                                                     \partial_\omega f_k b_0 + \sum_{|j|<N} \partial_{a_j} f_k (b_1)_j & ,\ |k|<N                                                    \\
                                                     \lambda_k (b_1)_k                                                 & ,\ |k|\geq N,\ \lambda_k := -k^2\omega^2 - \mu ik\omega + 1 \\
                                                   \end{cases}
  \end{align*}
  と書ける.またこのとき,$(A^\dagger b)_0$と$(A^\dagger b)_1$はそれぞれ
  \begin{align*}
    \left(A^\dagger b\right)_0 & = A_{a,0}^\dagger b_1 = \sum_{|k|<N} (b_1)_k \in \mathbb{C} \\
    \left(A^\dagger b\right)_1 & = A_{\omega,1}^\dagger b_0 + A_{a,1}^\dagger b_1 \in l_{\nu^\prime}^1
  \end{align*}

  次に,$作用素A$について考える.
  \begin{equation*}
    A^{N} = \begin{bmatrix}
      A_{\omega,0}^{(N)} & A_{a,0}^{(N)} \\
      A_{\omega,1}^{(N)} & A_{a,1}^{(N)}
    \end{bmatrix}
    \approx DF^{(N)} \left(\bar{x}\right)^{-1} \in \mathbb{C}^{2N\times 2N}
  \end{equation*}
  をヤコビ行列の近似逆行列とする.$A\in\mathcal{L}(Y,X)$として,$b=\left(b_0,b_1\right)\in X$に対して,$Ab=\left((Ab)_0,(Ab)_1\right)$と作用するように定義する.ここで,
  \begin{align*}
    (Ab)_0 & = A_{\omega,0}^{(N)}b_0 + A_{a,0}^{(N)} b_1^{(N)} \\
    (Ab)_1 & = A_{\omega,1}^{(N)}b_0 + A_{a,1}^{(N)} b_1
  \end{align*}

  ただし,無限次元の$A_{a,1}b_1$は以下のようになる.
  \begin{equation*}
    \left(A_{a,1}b_1\right)=\begin{dcases}
      \left(A_{a,1}^{(N)}b_1{(N)}\right)_k & \ (|k|<N)     \\
      \frac{(b_1)_k}{\lambda_k}            & \ (|k|\geq N)
    \end{dcases}
  \end{equation*}

  この定義を形式的に見ると,
  \begin{equation*}
    A =
    \begin{bmatrix}
       & \multicolumn{1}{c|}{A_{\omega,0}^{(N)}} & A_{a,0}^{(N)} & 0                   & \cdots                  & 0      & \\ \cline{2-6}
       & \multicolumn{1}{c|}{A_{\omega,1}^{(N)}} & A_{a,1}^{(N)} &                     & \text{\large{0}}        &        & \\
       & \multicolumn{1}{c|}{0}                  &               & \frac{1}{\lambda_N} &                         &        & \\
       & \multicolumn{1}{c|}{\vdots}             &               &                     & \frac{1}{\lambda_{N+1}} &        & \\
       & \multicolumn{1}{c|}{0}                  &               & \text{\large{0}}    &                         & \ddots & \\
    \end{bmatrix}
    =
    \begin{bmatrix}
       & \multicolumn{1}{c|}{0}            & A_{a,0} & \\ \cline{2-3}
       & \multicolumn{1}{c|}{A_{\omega,1}} & A_{a,0} &
    \end{bmatrix}
  \end{equation*}
  と表記できる.
\end{dfn}

\section{$Y_0,Z_0,Z_1,Z_2$の評価}
\label{sct:Y0,Z0,Z1,Z2の評価}
\begin{description}
  \item[$Y_0$]
    \begin{equation}
      \label{eq:y0}
      \left\| AF\left(\bar{x}\right)\right\|_X \leq Y_0
    \end{equation}
  \item[$Z_0$]
    \begin{equation*}
      \left\| I-AA^\dagger \right\|_{\mathcal{L}\left(
      X
      \right)} \leq Z_0
    \end{equation*}
  \item[$Z_1$]
    \begin{equation*}
      \left\| A \left(DF \left( \bar{x} \right) A^\dagger \right) c \right\|_X \leq Z_1,\ c\in \overline{B(0,1)}
    \end{equation*}
    また,ここでの$\overline{B(0,1)}$とは$\|c\|_X=1$ということである.
  \item[$Z_2$]
    \quad \\
    $b\in \overline{B\left(\bar{x},r\right)},\ h\in \overline{B(0,1)}$として,
    \begin{equation*}
      \left\| A \left(DF \left( b \right) - DF \left( \bar{x} \right) h \right) \right\|_{X} \leq Z_2 (r)r
    \end{equation*}
\end{description}

\subsection{$Y_0$を計算する}
\begin{equation*}
  F(\bar{x}) = (\delta_0, \delta_1) \in \mathbb{C}\times l_\nu^\prime
\end{equation*}
とすると,$A$の定義より,
\begin{equation*}
  \left\| AF(\bar{x}) \right\|_X \leq \max \left\{ \left| A_{\omega, 0}^{(N)} \delta_0 + A_{a,0}^{(N)} \delta_1^{(N)} \right|,\  \left\| A_{\omega,1}^{(N)} \delta_0 + A_{a,1} \delta_1^{(N)} \right\|_\omega + \sum_{|k|>N} \left\| \frac{\left(\delta_1^\infty\right)_k}{\lambda_k}  \right\| \nu^{|k|} \right\}
\end{equation*}

ここで,$\delta_1 = \left( \delta_1^{(N)}, \delta_1^{(\infty)} \right) \in \mathbb{C}^{2 \times 3(N-1)+1}$であり,

\begin{equation*}
  \left( \delta_1 \right)_k = \begin{cases}
    \delta_1^{(N)}      & (k<|N|)      \\
    \delta_1^{(\infty)} & (k \geq |N|)
  \end{cases}
\end{equation*}
と表す.

\subsection{$Z_0$を計算する}
\begin{equation*}
  B := I - AA^\dagger = \begin{bmatrix}
    B_{\omega,0} & B_{a,0} \\
    B_{\omega,1} & B_{a,1}
  \end{bmatrix}
\end{equation*}

この$Bをc\in \overline{B(0,1)},\ \|c\|_X \leq 1$である$c=(c_0,c_1)$に作用させると,
\begin{align*}
  (Bc)_0 & = B_{\omega,0} c_0 + B_{a,0} c_1 \\
  (Bc)_1 & = B_{\omega,1} c_0 + B_{a,1} c_1
\end{align*}

$Bc_0$はスカラ値なので,
\begin{align*}
  \left| B_{a,0}c_1 \right| & \leq \sum_{k \in \mathbb{Z}} \left|(B_{a,0})_k\right| \left|(c_1)_k \right|                                                            \\
                            & = \sum_{k \in \mathbb{Z}} \frac{|(B_{a,0})_k|}{\omega_k} |(c_1)_k| \omega_k                                                            \\
                            & \leq \max_{k < |N|} \frac{|(B_{a,0})_k|}{\omega_k} \sum_{k \in \mathbb{Z}} |(c_1)_k| \omega_k                                          \\
                            & \leq \max_{k < |N|} \frac{|(B_{a,0})_k|}{\omega_k},\ \left( \sum_{k \in \mathbb{Z}} |(c_1)_k| \omega_k = \|c_1\|_\omega \leq 1 \right)
\end{align*}

以上より,
\begin{equation*}
  |(Bc)_0| \leq |B_{\omega,0}| + \max_{|k| < N} \frac{|(B_{a,0})|_k}{\omega_k} = Z_0^{(0)}
\end{equation*}

またここで,作用素$M:l_\nu^1 \Rightarrow l_\nu^1 $の作用素ノルムについて以下の補題を陣日する.

\begin{lmm}
  行列$M^{(N)}$を$M^{(N)} \in \mathbb{C}^{(2N-1) \times (2N-1)}$,双方向の複素無限点列を$(\delta_k)_{|k| \geq N}$と定義する.ここで,$\delta_N > 0$であり,
  \begin{equation*}
    |\delta_k| \leq \delta_N,\ \forall |k| \geq N
  \end{equation*}
  を満たすとする.そして,$a=(a_k)_{k<\mathbb{Z}} \in l_\nu^1$に対して$a^{(N)}=(a_{-N+1},\cdots,a_{N-1})\in \mathbb{C}^{2N-1}$と表し,作用素$M:l_\nu^1 \Rightarrow l_\nu^1$を
  以下のように定義する.
  \begin{equation*}
    [Ma]_k := \begin{cases}
      [M^{(N)}a^{(N)}]_k & (|k|<N)     \\
      \delta_k a_k       & (|k|\geq N)
    \end{cases}
  \end{equation*}

  このとき,$M$は有界線形作用素であり,
  \begin{equation*}
    \|M\|_{\mathcal{L}(l_\nu^1)} \leq \max \left( K,\delta_N \right),\ K:=\max_{|n|<N} \frac{1}{\nu^{|n|}} \sum_{|k|<N} |M_{n,k}| \nu^{|k|}
  \end{equation*}
  と評価させる.
\end{lmm}

上の補題を利用すると,
\begin{eqnarray}
  \|(Bc)_1\|_\omega \leq \|B_{\omega,1}\|_\omega + \|B_{a,1}\|_{\mathcal{L}(l_\nu^1)} = Z_0^{(1)}
\end{eqnarray}
が評価可能となり,結論としては,求めたい$Z_0はZ_0:=\{Z_0^{(0)},Z_0^{(0)}\}$となる.

% ---------------------
\subsection{$Z_1$を計算する}
\begin{equation*}
  \left\| A \left( DF \left( \bar{x} \right) - A^\dagger \right) c \right\|_X \leq Z_1,\ c=(c_0,c_1) \in \overline{B(0,1)} \Leftrightarrow \|c\|_X \leq 1
\end{equation*}

点列$z$を下記のように定義する.
\begin{equation}
  z := \left(DF(\bar{x})-A^\dagger\right)c = \begin{bmatrix}
    z_0 \\
    z_1
  \end{bmatrix}
\end{equation}

ここで,$DF(\bar{x})とA^\dagger$は
\begin{align*}
  DF(\bar{x}) & = \begin{bmatrix}
                     & \multicolumn{0}{c|}{0}                         & \cdots & 0      & \cdots & \\ \cline{2-5}
                     & \multicolumn{1}{c|}{\partial_\omega f_k^{(N)}} & \cdots & 0      & \cdots & \\
                     & \multicolumn{1}{c|}{\vdots}                    &        & \vdots &        &
                  \end{bmatrix}                                         \\
  A^\dagger   & = \begin{bmatrix}
                     & \multicolumn{1}{c|}{0}              & 1      & \cdots            & 1      & \cdots    & 0                & \cdots & \\ \cline{2-8}
                     & \multicolumn{1}{c|}{\vdots}         &        & \vdots            &        &           &                  &        & \\
                     & \multicolumn{1}{c|}{\partial_w f_k} & \cdots & \partial_{a_j}f_k & \cdots &           &                  &          \\
                     & \multicolumn{1}{c|}{\vdots}         &        & \vdots            &        &           & \text{\Large{0}} &          \\
                     & \multicolumn{1}{c|}{\vdots}         &        &                   &        & \lambda_N &                  &        & \\
                     & \multicolumn{1}{c|}{0}              &        & \text{\Large{0}}  &        &           & \lambda_{N+1}    &          \\
                     & \multicolumn{1}{c|}{\vdots}         &        &                   &        &           &                  & \ddots
                  \end{bmatrix} \\
  \lambda_k   & := -k^2\omega^2 - \mu ik\omega + 1
\end{align*}
と表される.すると$z_0$は,
\begin{align*}
  z_0 & = \sum_{|k|<N} (c_1)_k, \\
  |z_0| &\leq \frac{1}{\omega_N} \sum_{|k|<N} |(c_1)_k| \omega_k \leq \frac{1}{\omega_N}
\end{align*}

次に,$z_1$について考える.$DF(\bar{x})c$部分は,
\begin{equation*}
  \begin{split}
    \left(\left(DF(\bar{x})c\right)_1 \right)_k &= \partial_\omega f_k c_0 + \partial_a f_k c_1 \\
    &= \frac{\partial \lambda_k}{\partial} c_0 \bar{a}_k + \frac{\mu ik}{3} \left( \bar{a} * \bar{a} * \bar{a} \right)_k c_0 + \lambda_k (c_1)_k + \mu ik \omega \left( \bar{a} * \bar{a} * c_1 \right)_k,\ k \in \mathbb{Z}
  \end{split}
\end{equation*}

と書け,$|k| \geq N で \bar{a_k} = 0$より,$c_1 = c_1^{(N)} + c_1^{(N)}$として,
\begin{align*}
  |(z_1)_k| & \leq |\mu ik \omega| \max \left\{ \max_{k-N+1\leq j \leq -N} \frac{|\left( \bar{a} * \bar{a} \right)_{k-j}|}{\omega_j} ,\ \max_{N \leq j \leq k+N-1 \frac{|\left( \bar{a} * \bar{a}\right)_{k-j} |}{\omega_j}} \right\} := \zeta, \\
  \zeta     & = (\zeta_k)_{k < N} \in \mathbb{R}^{2N-1}
\end{align*}

最後に,$Z_1$を評価していく.$Z_1^{(0)}$の評価は
\begin{align*}
  |(A(DF(\bar{x})-A^\dagger)c)_0| & = |(Az)_0|                                                                                                \\
                                  & \leq \left|A_{\omega,0}^{(N)}\right| \left|Z_0\right| + \left|A_{a,0}^{(N)}\right| \left|Z_1^{(N)}\right| \\
                                  & \leq \frac{|A_{\omega, 0}^{(N)}|}{\omega_N} + \left|A_{a,0}^{(N)}\right| \zeta                            \\
                                  & =: z_1^{(0)}
\end{align*}

$Z_1^{(1)}$の評価は
\begin{equation*}
  \begin{split}
    \left\| (A(DF(\bar{x})-A^\dagger)c)_1 \right\|_\omega =& \left\| (Az)_1 \right\|_\omega \\
    =& \left\| A_{\omega,1}^{(N)} z_0 + A_{a,1} z_1 \right\|_\omega \\
    \leq& \frac{\left\| A_{\omega,1}^{(N)} \right\|_\omega}{\omega_N} + \sum_{|k|<N} \left( \left| A_{a,1}^{(N)} \right| \zeta \right)_k \omega_k + \sum_{N \leq |k| \leq 3(N-1)} \frac{\left| \mu ik \left( \bar{a} * \bar{a} * \bar{a} \right)_k \right|}{3\left| \lambda_k \right|} \omega_k \\
    &+ \sum_{|k| \geq N} \frac{\left| \mu ik \omega \left(\bar{a}*\bar{a}*\bar{x}\right) \right|}{\left| \lambda_k \right|} \omega_k \\
    \leq& \frac{\left\| A_{\omega,1}^{(N)} \right\|_\omega}{\omega_N} + \left\| \left| A_{a,1}^{(N)} \right| \zeta \right\|_\omega + \sum_{N\leq|k|\leq 3(N-1)} \frac{\left| \mu ik \left( \bar{a} * \bar{a} * \bar{a} \right)_k \right|}{3\left| \lambda_k \right|} \omega_k \\
    &+ \frac{1}{N} \frac{\mu \omega \left\|\bar{a}\right\|_\omega^2}{\omega^2-\frac{1}{N^2}}\\
    =:& Z_1^{(1)}
  \end{split}
\end{equation*}

よって,
\begin{equation*}
  Z_1 := \max \left\{ Z_1^{(0)}, Z_1^{(1)} \right\}
\end{equation*}

%---------------------
\subsection{$Z_2$を計算する}
$b \in \overline{B\left(\bar{x},r\right)},\ c=\left(c_0,c_1\right)\in\overline{Z(0,1)}$について
\begin{equation*}
  \left\| A \left( DF \left( b \right) - DF \left( \bar{x} \right) \right) c \right\|_X \leq Z_2(r)r
\end{equation*}
を考える.まず$z$を
\begin{equation*}
  z := \left( DF(b) - DF(\bar{x}) \right) c = \begin{bmatrix}
    z_0 \\
    z_1
  \end{bmatrix}
  = \begin{bmatrix}
    0 \\
    z_1
  \end{bmatrix}
\end{equation*}
と定義する.$z_0=0$となるので,$z_1$だけを考えればよく
\begin{equation*}
  \left(z_1\right)_k := \left(\partial_\omega f_k \left(b\right) - \partial_\omega f_k \left(\bar{x}\right) \right)c_0 + \left[ \left(\partial_a f\left(b\right) - \partial_a f \left(\bar{x}\right) \right) c_1 \right]_k,\ k \in \mathbb{Z}
\end{equation*}
と書ける.$b=\left( \omega, \left(a_k\right)_{k\in\mathbb{Z}} \right), \bar{x}=\left( \bar{\omega}, \left(\bar{a_k}\right)_{|k|<N} \right)$として,第1項は
\begin{equation*}
  \begin{split}
    &\left(\partial_\omega f_k \left(b\right) - \partial_\omega f_k \left(\bar{x}\right) \right)c_0 \\
    =& \left[ \left( \left( -2k^2\omega - \mu ik \right) a_k + \frac{\mu ik}{3} \left( a*a*a \right)_k \right) - \left( \left( -2k^2\bar{\omega} - \mu ik \right) \bar{a}_k       + \frac{\mu ik}{3} \left( \bar{a}*\bar{a}*\bar{a} \right)_k \right) \right]c_0 \\
    =& \left[  -2k^2\omega \left( a_k - \bar{a}_k \right) -2k^2\left( \omega - \bar{\omega} \right)\bar{a}_k - \mu ik \left( a_k - \bar{a}_k \right) + \frac{\mu ik}{3} \left( \left( a*a*a \right)_k - \left(\bar{a}*\bar{a}*\bar{a}\right)_k \right) \right] c_0
  \end{split}
\end{equation*}
と書ける.そして,第2項は,
\begin{equation*}
  \begin{split}
    &\left[ \left( \partial_a f(b) - \partial_a f(\bar{x}) \right)_{c_1} \right]_k \\
    =& \left( -k^2\omega^2 - \mu ik \omega + 1 \right) \left( c_1 \right)_k + \mu ik \omega (a * a * c_1)_k - \left( -k^2\bar{\omega}^2 - \mu ik \bar{\omega} + 1 \right) \left( c_1 \right)_k \\
    &+ \mu ik \bar{\omega} (\bar{a} * \bar{a} * c_1)_k \\
    =& \left[ -k^2\left(\omega + \bar{\omega}\right) \left(\omega - \bar{\omega}\right) - \mu ik \left(\omega - \bar{\omega}\right) \right] \left(c_1\right)_k + \mu ik \omega \left(\left(a+\bar{a}\right)*\left(a-\bar{a}\right)*c_1\right)_k \\
    &+ \mu ik \left(\omega - \bar{\omega}\right) \left( \bar{a} * \bar{a} * c_1 \right)_k
  \end{split}
\end{equation*}
と書ける.$(Az)_0,(Az)_1$は,
\begin{align*}
  (Az)_0 & = A_{a,0}^{(N)} + z_1^{(N)} \\
  (Az)_1 & = A_{a,1} + z_1
\end{align*}
より,
\begin{equation*}
  \|Az\|_X = \max\left\{ \left| A_{a,0}^{(N)} z_1^{(N)} \right|,\ \left\| A_a,1 z_1 \right\| \right\}
\end{equation*}
となる.

$\left| A_{a,0}^{(N)} z_1^{(N)} \right|$を上から評価する.はじめに,$\tilde{A}_{a,0}, \tilde{B}_{a,0}$を以下のように定義する.
\begin{align*}
  \tilde{A}_{a,0} & := \left( \left| k \right| \left( A_{a,0}^{(N)} \right)_k \right)_{|k|<N} \\
  \tilde{B}_{a,0} & := \left( k^2 \left( A_{a,0}^{(N)} \right)_k \right)_{|k|<N}
\end{align*}
すると,
\begin{equation*}
  \begin{split}
    \left| A_{a,0}^{(N)} z_1^{(N)} \right| &\leq 2\left(\tilde{\omega} + r\right) \left\| \tilde{B}_{a,0} \right\|_\omega r + 2 \left\| \tilde{B}_{a,0} \right\|_\omega \left\| \tilde{a} \right\|_\omega r + \mu \left\| \tilde{A}_{a,0} \right\|_\omega r \\
    &+ \frac{\mu}{3} \left\| \tilde{A}_{a,0} \right\|_\omega \left( r^2 * 3 \left\| \bar{a} \right\|_\omega r + 3 \left\| \bar{a} \right\|^2_\omega \right) r + \left\| \tilde{B}_{a,0} \right\|_\omega \left( 2\bar{\omega} + r\right)r + \mu \left\| \tilde{A}_{a,0} \right\|_\omega r \\
    &+ \mu\left(\bar{\omega}+r\right)\left\| \tilde{A}_{a,0} \right\|_\omega \left( 2 \left\| \bar{a} \right\|_{\omega} + r\right)r + \mu \left\| \tilde{A}_{a,0} \right\|_\omega \left\| a \right\|_\omega^2 r\\
    &= Z_2^{(4,0)} r^3 + Z_2^{(3,0)} r^2 + Z_2^{(2,0)} r
  \end{split}
\end{equation*}
となる.同様に,$\left\| A_{a,1} z_1 \right\|_\omega$を上から評価する.$\tilde{A}_{a,1}, \tilde{B}_{a,1}$を以下のように定義する.
\begin{align*}
  \tilde{A}_{a,1} & := \left( \left| j \right| \left( A_{a,1} \right)_{k,j} \right)_{k,j\in\mathbb{Z}} \\
  \tilde{B}_{a,1} & := \left( j^2 \left( A_{a,1} \right)_{k,j} \right)_{k,j\in\mathbb{Z}}
\end{align*}
すると,
\begin{equation*}
  \begin{split}
    \left\| A_{a,1} z_1 \right\|_{\omega} &\leq 2\left(\tilde{\omega} + r\right) \left\| \tilde{B}_{a,1} \right\|_{\mathcal{L}\left(l_{\nu}^1\right)} r + 2 \left\| \tilde{B}_{a,1} \right\|_{\mathcal{L}\left(l_{\nu}^1\right)} \left\| \tilde{a} \right\|_\omega r + \mu \left\| \tilde{A}_{a,1} \right\|_{\mathcal{L}\left(l_{\nu}^1\right)} r \\
    &+ \frac{\mu}{3} \left\| \tilde{A}_{a,1} \right\|_{\mathcal{L}\left(l_{\nu}^1\right)} \left( r^2 + 3 \left\| \bar{a} \right\|_\omega r + 3 \left\| \bar{a} \right\|^2_\omega \right) r + \left\| \tilde{B}_{a,1} \right\|_{\mathcal{L}\left(l_{\nu}^1\right)} \left( 2\bar{\omega} + r\right)r + \mu \left\| \tilde{A}_{a,1} \right\|_{\mathcal{L}\left(l_{\nu}^1\right)} r \\
    &+ \mu\left(\bar{\omega}+r\right)\left\| \tilde{A}_{a,1} \right\|_{\mathcal{L}\left(l_{\nu}^1\right)} \left( 2 \left\| \bar{a} \right\|_{\omega} + r\right)r + \mu \left\| \tilde{A}_{a,1} \right\|_{\mathcal{L}\left(l_{\nu}^1\right)} \left\| a \right\|_\omega^2 r\\
    &= Z_2^{(4,1)} r^3 + Z_2^{(3,1)} r^2 + Z_2^{(2,1)} r
  \end{split}
\end{equation*}
と書ける.$Z_2^{(4,1)},Z_2^{(3,1)},Z_2^{(2,1)}$は,先ほどの$\tilde{A}_{a,0},\tilde{B}_{a,0}を\tilde{A}_{a,1},\tilde{B}_{a,1}$に置き換えたものになる.$j=2,3,4$で
\begin{equation*}
  Z_2^{(j)} := \max \left\{ Z_2^{(j,0)}, Z_2^{(j,1)} \right\}
\end{equation*}
とすれば,
\begin{equation*}
  Z_2(r) := Z_2^{(4,1)} r^2 + Z_2^{(3,1)} r + Z_2^{(2,1)}
\end{equation*}
となる.

\begin{comment}
  \section{radii polynomialの零点探索の精度保証}
以上で,$Y_0,\cdots,Z_2$の評価を区間演算で求めた.これらの評価を用いて$p(r_0)<0$となる$r_0$を求める.精度保証の方法は,Newton法を反復させることで,$p(r_0)=0$となる$r_0$の近似会を求め,これをKrawczyk法で検証する.

\section{Krawczyk(クラフチック)法}
\begin{thm}[Krawczyk法\cite{}]
  $X \subset \mathbb{R}^n$を区間ベクトル,$c=\mathrm{mid}(X),\ R\simeq Df(c)^{-1} = j(c)^{-1}, E$を単位行列とし,
  \begin{equation*}
    K(X) = c- Rf(c) + (E-RDf(X))(X-c)
  \end{equation*}
  としたとき,$K(X) \subset \mathrm{int}(X)$($\mathrm{int}(X):X$の内部)ならば$Xにf(x)=0$の解が唯一存在する.
\end{thm}
\end{comment}


%======
%
%=====


\chapter{無限次元ガウスの消去法\cite{}}
\section{射影}

\begin{dfn}[代数的直和]
  Banach空間$X$とする.また,$X_1,X_2$をXの線形部分空間とする.ただし,$X_1$と$X_2$のノルムは,$X$のノルムと同一とする.そのとき,$X$が$X_1$と$X_2$の代数的直和であるとは
  \begin{itemize}
    \item $X=X_1+X_2:=\{x_1+x_2 | \  \forall x_1 \in X_1, \ \forall x_2 \in X_2 \}$
    \item $X_1 \cap X_2 = \{0\}$
  \end{itemize}
  が成立することをいう.
\end{dfn}

\begin{dfn}[射影]
  $X$をノルム空間とする.定義域を$X$とした$X$上の線形作用素$P$が
  \begin{equation*}
    P^2=P
  \end{equation*}
  となるとき,線形作用素,あるいは,単に射影と呼ぶ.
\end{dfn}

\begin{thm}
  Banach空間$X$がその線形部分空間$X_1,X_2$の代数的直和であるとする.そのとき,$x\in X$について
  \begin{equation*}
    x=x_1+x_2,\ x_1 \in X_1,\ x_2 \in X_2
  \end{equation*}
とし,$\mathcal{D}(P)=X$となる$X$上の線形作用素$P$を
\begin{equation*}
  Px=x_1,\ (I-P)x=x_2
\end{equation*}
とすると,線形作用素$P$と$I-P$は射影になる.
\end{thm}

\begin{proof}
  まず,$P{x_2}=0$を証明する.$P{x_2} \neq 0$ となる$x_2 \in X_2$が存在すると仮定し,矛盾を示す.

  定義より,
  \begin{equation*}
    x = x_1 + x_2 = x_1 + P{x_2} + (I-P)x_2
  \end{equation*}

  に対して,$P$の閾値$\mathcal{R}(P)=X_1$であることと,$X_1$が線形空間であることに注意すると,$x_1+Px_2 \in x_2 $である.しかし,これは代数的直和の定義より$x$の分解の一意性に矛盾する.よって$P{x_2}=0$となる.

  そのうえで,
  \begin{equation*}
    Px_2 = P(I-P)x = Px - P^2x = 0
  \end{equation*}
  より
  \begin{equation*}
    Px = P^2x
  \end{equation*}
  となるため,$P$は射影となる.

  また,
  \begin{equation*}
    (I-P)(I-P)x = (I-2P+P^2)x = (I - 2P + P)x = (I-P)x
  \end{equation*}
  となるため,$I-P$も射影となる.
\end{proof}

\begin{thm}
  Banach空間$X$が$X_1$と$X_2$の代数的直和であるとする.$x \in X$について
  \begin{equation*}
    x = x_1 + x_2,\ x_1 \in X_1,\ x_2\in X_2
  \end{equation*}
  とし,$\mathcal{D}(P)=X$となる$X$となる$X$上の線形作用素$P$を
  \begin{equation*}
    Px = x_1, (I-P)x = x_2
  \end{equation*}
  とすると以下が成立する.
  \begin{equation*}
    Pが連続 \Leftrightarrow X_1, X_2 がともに\text{Banach}空間
  \end{equation*}
\end{thm}

\begin{proof}
  「Pが連続 $\Rightarrow X_1, X_2$ がともにBanach空間の証明」
  まず,$X_1$がBanach空間であることを示す.$X$が$X_1, X_2$の代数的直和であることから,$X_1$は$X$の線形部分空間である.そのため,$X_1$が閉空間であになることを示せば,定理\ref{dfn:完備}より,$X_1$がBanach空間であることを示せる.よって,$X_1$の任意の点列$(x_n) \subset X_1$に対し,極限$x^*$が$X_1$に属することを示せばよい.$x_n$は$X_1$に属することから$Px_n = x_n$になる.そのうえで,$P$が連続であることから,$Px_n = x_n$の極限は$Px^* = x^*$となる.よって,$Px^*=x^* \in X_1$から,$X_1$は閉集合となるため,Banach空間となる.

  次に,$X_2$がBanach空間であることを示す.そのために,まず,$I-P$が連続になることを示す.$P$が連続であることから,$x_n \rightarrow x^*$となる任意の$x_n$に対して,$Px_n \rightarrow Px^*$となることに注意すると
  \begin{equation*}
    (I-P)x_n = x_n - Px_n \rightarrow x^* - P x^* = (I-P)x^*
  \end{equation*}
  となるため,$I-P$も連続となる.あとは,$P$と$X_1$のときと同様の議論をすればよい.すると,$(I-P)x^*=x^* \in X_2$から,$X_2$は閉集合となるため,Banach空間となる.


  「Pが連続 $\Leftarrow X_1, X_2$ がともにBanach空間の証明」

  Banach空間$X$の$x_n \rightarrow x^*$となる任意の点列$(x_n)$に対し,$Px_n \rightarrow y$としたとき,$y=Px^*$となることで示せば良い.上記の点列を用いて$X_1$の点列$(Px_n)$を作成すると,$X_1$がBanach空間であることから,閉集合であるため点列$(Px_n)$の極限$y$は$X_1$にも属する.よって,
  \begin{equation*}
    y=Py
  \end{equation*}
  となる.

  さらに,同様に$X_2$の点列$((I-P)x_n)$を作成すると,その極限は
  \begin{equation*}
    (I-P)x_n=x_n-Px_n \rightarrow x^*-y
  \end{equation*}
  となり,$X_2$もBanach空間であることがら,閉集合であるため点列$((I_P)x_n)$の極限$x^*-y$は$X^2$にも属する.そのうえで,Banach空間$X$が$X_1$と$X_2$の代数的直和であることから,$X_1 \cup X_2={0}$に注意すると$P(x^*-y)=0$となる.よって,
  \begin{equation*}
    0=P(x^*-y)=Px^*-Py=Px^*-y
  \end{equation*}
  となるため,
  \begin{equation*}
    Px^*=y
  \end{equation*}
  となる.
\end{proof}

\begin{dfn}[位相的直和]
  Banach空間$X$が$X_1$と$X_2$の代数的直和であるとする.その上,$X_1$と$X_2$が共 にBanach空間であるとき,Banach空間$X$が$X_1$と$X_2$の位相的直和であるという.
\end{dfn}

\section{射影を用いたBanach空間上の無限次元ガウスの消去法}
\label{sct:無限次元ガウスの消去法}

  初めに,$X,X_1,X_2,L,g,P,\phi_1,\phi_2,T,B,C,D$を定義する.

  Banach空間$X$とし,有界な線形作用素$L \in \mathcal{B}(X)$と$g\in X$に対して問題
  \begin{equation*}
    \text{Find}\ \phi \in X \ \text{s.t.}\ L\phi = g
  \end{equation*}
  の解を求める方法を考える.

  Banach空間$X$が$X_1$と$X_2$の代数的直和であるとし,$x \in X$について
  \begin{equation*}
    x=x_1+x_2, x_1 \in X_1, x_2 \in X_2
  \end{equation*}
  とし,$P$を$\mathcal{D}(P)=X$
  \begin{equation*}
    Px=x_1, (I-P)x=x_2
  \end{equation*}
  を満たす射影をする.

  上記の式より,
  \begin{equation*}
    L \phi = g \Leftrightarrow
    \left\{ \,
    \begin{aligned}
      PL\phi &= Pg \\
      (I-P)L\phi &= (I-P)g
    \end{aligned}
    \right.
  \end{equation*}

  さらに,解$\phi$も射影を使って分解する.
  \begin{equation*}
    \phi = \phi_1 + \phi_2,\ \phi_1:=P\phi,\ \phi_2:=(I-P)\phi
  \end{equation*}

  この分解した解を利用すると
  \begin{equation*}
    \left\{ \,
    \begin{aligned}
      PL(\phi_1+\phi_2) &= Pg \\
      (I-P)L\phi &= (I-P)g
    \end{aligned}
    \right.
  \end{equation*}
  となる.ここで,それぞれの線形作用素を
  \begin{equation*}
    \begin{array}{ll}
      T:=PL \mid_{X_1} : X_1 \rightarrow X_1, & B:=PL \mid_{X_2} : X_2 \rightarrow X_1 \\
      C:=(I-P)L \mid_{X_1} : X_1 \rightarrow X_2, & D:=(I-P)L \mid_{X_2} : X_2 \rightarrow X_2
   \end{array}
  \end{equation*}
  と定義すると,次のように変形できる.
  \begin{equation*}
    \left\{ \,
    \begin{aligned}
      T\phi_1+B\phi_2 &= Pg \\
      C\phi_1+D\phi_2 &= (I-P)g
    \end{aligned}
    \right.
  \end{equation*}

  ここで,
  \begin{equation*}
    \left(
    \begin{array}{cc}
      T & B \\
      C & D
    \end{array}
    \right)
    :X_1 \times X_2 \rightarrow  X_1 \times X_2
  \end{equation*}
  を定義すると
  \begin{equation*}
    \left(
    \begin{array}{cc}
      T & B \\
      C & D
    \end{array}
    \right)
    \begin{pmatrix}
      \phi_1 \\
      \phi_2
    \end{pmatrix}
    =
    \begin{pmatrix}
      Pg \\
      (I-P)g
    \end{pmatrix}
  \end{equation*}
  と変形できる.

\begin{thm}
  \label{thm:6.2.1-L全単射}
  $X,X_1,X_2,L,g,P,\phi_1,\phi_2,T,B,C,D$は,本節で定義したものとする.

  線形作用素$T$を全単射であると仮定する,作用素$S$を
  \begin{equation*}
    S:=D-CT^{-1}B : X_2 \rightarrow X_2
  \end{equation*}
  とする.もし,$S$が全単射ならば,
  \begin{equation*}
    \begin{pmatrix}
      \phi_1 \\
      \phi_2
    \end{pmatrix}
    =
    \begin{pmatrix}
      T^{-1}+T^{-1}BS^{-1}CT^{-1} & -T^{-1}BS^{-1} \\
      -S^{-1}CT^{-1} & S^{-1}
    \end{pmatrix}
    \begin{pmatrix}
      Pg \\
      (I-P)g
    \end{pmatrix}
  \end{equation*}
  となり,有界線形作用素$L$は全単射である.
\end{thm}

\begin{proof}
  方程式
  \begin{equation*}
    \begin{pmatrix}
      T & B \\
      C & D
    \end{pmatrix}
    \begin{pmatrix}
      \phi_1 \\
      \phi_2
    \end{pmatrix}
    =
    \begin{pmatrix}
      Pg \\
      (I-P)g
    \end{pmatrix}
  \end{equation*}
  に対して,$T$が全単射であることから左から
  \begin{equation*}
    \begin{pmatrix}
      T^{-1} & 0 \\
      0 & I_{X_2}
    \end{pmatrix}
  \end{equation*}
  を掛けると
  \begin{equation*}
    \begin{pmatrix}
      I_{X_1} & T^{-1}B \\
      C & D
    \end{pmatrix}
    \begin{pmatrix}
      \phi_1 \\
      \phi_2
    \end{pmatrix}
    =
    \begin{pmatrix}
      T^{-1} & 0 \\
      0 & I_{X_2}
    \end{pmatrix}
    \begin{pmatrix}
      Pg \\
      (I-P)g
    \end{pmatrix}
  \end{equation*}
  となる.次に,左から
  \begin{equation*}
    \begin{pmatrix}
      I_{X_1} & 0 \\
      -C & I_{X_2}
    \end{pmatrix}
  \end{equation*}
  を掛けると
  \begin{equation}
    \label{eq:6-4}
    \begin{split}
      \begin{pmatrix}
        I_{X_1} & T^{-1}B \\
        0 & S
      \end{pmatrix}
      \begin{pmatrix}
        \phi_1 \\
        \phi_2
      \end{pmatrix}
      &=
      \begin{pmatrix}
        I_{X_1} & 0 \\
        -C & I_{X_2}
      \end{pmatrix}
      \begin{pmatrix}
        T^{-1} & 0 \\
        0 & I_{X_2}
      \end{pmatrix}
      \begin{pmatrix}
        Pg \\
        (I-P)g
      \end{pmatrix}\\
      &=
      \begin{pmatrix}
        T^{-1} & 0 \\
        -CT^{-1} & I_{X_2}
      \end{pmatrix}
      \begin{pmatrix}
        Pg \\
        (I-P)g
      \end{pmatrix}
    \end{split}
  \end{equation}
  となる.次に,$S$が全単射であることから左から
  \begin{equation*}
    \begin{pmatrix}
      I_{X_1} & 0 \\
      0 & S^{-1}
    \end{pmatrix}
  \end{equation*}
  を掛けると
  \begin{equation*}
    \begin{split}
      \begin{pmatrix}
        I_{X_1} & T^{-1}B \\
        0 & S
      \end{pmatrix}
      \begin{pmatrix}
        \phi_1 \\
        \phi_2
      \end{pmatrix}
      &=
      \begin{pmatrix}
        T^{-1} & 0 \\
        -S^{-1}CT^{-1} & S^{-1}
      \end{pmatrix}
      \begin{pmatrix}
        Pg \\
        (I-P)g
      \end{pmatrix}
    \end{split}
  \end{equation*}
  となる.最後に,
  \begin{equation*}
    \begin{pmatrix}
      I_{X_1} & -T^{-1}B \\
      0 & I_{X_2}
    \end{pmatrix}
  \end{equation*}
  を左から掛けると
  \begin{equation*}
    \begin{split}
      \begin{pmatrix}
        \phi_1 \\
        \phi_2
      \end{pmatrix}
      &=
      \begin{pmatrix}
        I_{X_1} & -T^{-1}B \\
        0 & I_{X_2}
      \end{pmatrix}
      \begin{pmatrix}
        T^{-1} & 0 \\
        -S^{-1}CT^{-1} & S^{-1}
      \end{pmatrix}
      \begin{pmatrix}
        Pg \\
        (I-P)g
      \end{pmatrix}\\
      &=
      \begin{pmatrix}
        T^{-1}+T^{-1}BS^{-1}CT^{-1} & -T^{-1}BS^{-1} \\
        -S^{-1}CT^{-1} & S^{-1}
      \end{pmatrix}
      \begin{pmatrix}
        Pg \\
        (I-P)g
      \end{pmatrix}
    \end{split}
  \end{equation*}
  が得られる.これは,
  \begin{equation*}
    \begin{pmatrix}
      T^{-1}+T^{-1}BS^{-1}CT^{-1} & -T^{-1}BS^{-1} \\
      -S^{-1}CT^{-1} & S^{-1}
    \end{pmatrix}
    \begin{pmatrix}
      T & B \\
      C & D
    \end{pmatrix}
    =
    \begin{pmatrix}
      I_{X_1} & 0 \\
      0 & I_{X_2}
    \end{pmatrix}
  \end{equation*}
  を意味する.逆に
  \begin{equation*}
    \begin{split}
      &\begin{pmatrix}
        T & B \\
        C & D
      \end{pmatrix}
      \begin{pmatrix}
        T^{-1}+T^{-1}BS^{-1}CT^{-1} & -T^{-1}BS^{-1} \\
        -S^{-1}CT^{-1} & S^{-1}
      \end{pmatrix}
      \\=&
      \begin{pmatrix}
        I_{X_1} & T^{-1}B + T^{-1}BS^{-1}CT^{-1}B - T^{-1}BS^{-1}D \\
        0 & -S^{-1}CT^{-1}B+S^{-1}D
      \end{pmatrix}
      \\=&
      \begin{pmatrix}
        I_{X_1} & T^{-1}B - T^{-1}BS^{-1}(D-CT^{-1}B) \\
        0 & -S^{-1}(D-CT^{-1}B)
      \end{pmatrix}
      =
      \begin{pmatrix}
        I_{X_1} & 0 \\
        0 & I_{X_2}
      \end{pmatrix}
    \end{split}
  \end{equation*}
  となるため,定義\ref{dfn:逆作用素}と定理\ref{thm:単射と逆作用素の関係}から
  \begin{equation*}
    \begin{pmatrix}
      T & B \\
      C & D
    \end{pmatrix}
  \end{equation*}
  は単射である.また,任意の$g_1 \in X_1$と$g_2 \in X_2$に対して,
  \begin{equation*}
    \begin{pmatrix}
      T & B \\
      C & D
    \end{pmatrix}
    \begin{pmatrix}
      \phi_1 \\
      \phi_2
    \end{pmatrix}
    =
    \begin{pmatrix}
      g_1 \\
      g_2
    \end{pmatrix}
  \end{equation*}
  は
  \begin{equation*}
    \begin{split}
      \begin{pmatrix}
        \phi_1 \\
        \phi_2
      \end{pmatrix}
      =&
      \begin{pmatrix}
        I_{X_1} & -T^{-1}B \\
        0 & I_{X_2}
      \end{pmatrix}
      \begin{pmatrix}
        T^{-1} & 0 \\
        -S^{-1}CT^{-1} & S^{-1}
      \end{pmatrix}
      \begin{pmatrix}
        Pg \\
        (I-P)g
      \end{pmatrix}
      \\=&
      \begin{pmatrix}
        T^{-1}+T^{-1}BS^{-1}CT^{-1} & -T^{-1}BS^{-1} \\
        -S^{-1}CT^{-1} & S^{-1}
      \end{pmatrix}
      \begin{pmatrix}
        g_1 \\
        g_2
      \end{pmatrix}
    \end{split}
  \end{equation*}
  となる解
  \begin{equation*}
    \begin{pmatrix}
      \phi_1 \\
      \phi_2
    \end{pmatrix}
    \in X_1 \times X_2
  \end{equation*}
  を持つため,
  \begin{equation*}
    \begin{pmatrix}
      T & B \\
      C & D
    \end{pmatrix}
    : X_1 \times X_2 \rightarrow X_1 \times X_2
  \end{equation*}
  は全射でもある.

  続いて,$L$が単射であることを示す.定理\ref{dfn:単射}から$L\phi =0$において,解が$\phi=0$だけであることを示せば良い.$X$が$X_1$と$X_2$の代数的直和であることから,解$\phi$は$P\phi$と$(I-P)\phi$に一意に分解できる.その上,
  \begin{equation*}
    \begin{pmatrix}
      T & B \\
      C & D
    \end{pmatrix}
    \begin{pmatrix}
      P\phi\\
      (I-P)\phi
    \end{pmatrix}
    =
    \begin{pmatrix}
      0 \\
      0
    \end{pmatrix}
  \end{equation*}
  となる.ここで,
  \begin{equation*}
    \begin{pmatrix}
      T & B \\
      C & D
    \end{pmatrix}
  \end{equation*}
  が全単射であることから,解は$P\phi = 0, (I-P)\phi=0$のみである.よって,$L \phi = 0$において,解は$\phi=0$のみである.

  最後に,$L$が全射であることを示す.任意の$g \in X$に対して$L \phi = g$を満たす解$\phi \in X$が存在すれば,定義\ref{dfn:全射}から$L$が全射であることがいえる.$X$が$X_1$と$X_2$の代数的直和であることから,解$\phi$は$P\phi$と$(I-P)\phi$に一意に分解できる.その上,
  \begin{equation*}
    \begin{pmatrix}
      T & B \\
      C & D
    \end{pmatrix}
    \begin{pmatrix}
      P\phi \\
      (I-P)\phi
    \end{pmatrix}
    =
    \begin{pmatrix}
      Pg \\
      (I-P)g
    \end{pmatrix}
  \end{equation*}
  となる.ここで,
  \begin{equation*}
    \begin{pmatrix}
      T & B \\
      C & D
    \end{pmatrix}
  \end{equation*}
  が全単射であることから,解$(P\phi,\ (I-P)\phi)\in X_1 \times X_2$は常に存在する.よって,$\phi = P \phi + (I-P)\phi \in X$であるため,任意の$g \in X$に対して解$\phi \in X$は存在する.
\end{proof}

\begin{thm}
  \label{thm:Sが全単射->Lが全単射}
  $X,X_1,X_2,L,g,P,\phi_1,\phi_2,T,B,C,D$はすべて本節で定義したものとする.線形作用素$T$を全単射であると仮定する.そのとき,
  \begin{equation*}
    Sが全単射 \Leftrightarrow Lが全単射
  \end{equation*}
\end{thm}

\begin{proof}
  「$Sが全単射 \Rightarrow Lが全単射$」定理\ref{thm:6.2.1-L全単射}でいえる.

  「$Sが全単射 \Leftarrow Lが全単射$」を示す.$L$が全単射であることから,$L\phi=0$を満たす解は$\phi = 0$のみである.また,$X$が$X_1$と$X_2$の代数的直和であることから,解$\phi=0$は$P\phi=0$と$(I-P)\phi$に一意に分解できる.その上,$T$が全単射であることを利用すると式(\ref{eq:6-4})が得られるため,以下のようになる:
  \begin{equation*}
    \begin{split}
      L\phi=0 &\Leftrightarrow
      \begin{pmatrix}
        T & B \\
        C & D
      \end{pmatrix}
      \begin{pmatrix}
        P \phi \\
        (I-P) \phi
      \end{pmatrix}
      =
      \begin{pmatrix}
        0 \\
        0
      \end{pmatrix}
      \\&\Leftrightarrow
      \begin{pmatrix}
        I_{X_1} & T^{-1}B \\
        0 & S
      \end{pmatrix}
      \begin{pmatrix}
        P \phi \\
        (I-P) \phi
      \end{pmatrix}
      =
      \begin{pmatrix}
        0 \\
        0
      \end{pmatrix}
    \end{split}
  \end{equation*}
  上記の方程式を満たす解が,$P\phi=0$と$(I-P)\phi=0$のみのため,
  \begin{equation*}
    \begin{pmatrix}
      I_{X_1} & T^{-1}B \\
      0 & S
    \end{pmatrix}
  \end{equation*}
  は単射である.ここで,$S$が単射でないと仮定して,矛盾を示す.$S$が単射でないことから,
  \begin{equation*}
    S\phi_2=0
  \end{equation*}
  を満たす$0$以外の解$\phi_2 \in X_2$が存在する.そのうえで,
  \begin{equation*}
    \phi_1 = -T^{-1}B\phi_2
  \end{equation*}
  とすると$(\phi_1, \phi_2)$は方程式
  \begin{equation*}
    \begin{pmatrix}
      I_{X_2} & T^{-1}B \\
      0 & S
    \end{pmatrix}
    \begin{pmatrix}
      \phi_1 \\
      \phi_2
    \end{pmatrix}
    =
    \begin{pmatrix}
      0 \\
      0
    \end{pmatrix}
  \end{equation*}
  の解となる.しかし,$L\phi=0$を満たす解は$\phi=0$のみであり,
  \begin{equation*}
    \begin{pmatrix}
      I_{X_2} & T^{-1}B \\
      0 & S
    \end{pmatrix}
    \begin{pmatrix}
      \phi_1 \\
      \phi_2
    \end{pmatrix}
    =
    \begin{pmatrix}
      0 \\
      0
    \end{pmatrix}
  \end{equation*}
  を満たす解は$0$のみであるため,矛盾する.よって,$S$は単射である.

  続いて,$L$が全単射のとき,$S$が全単射であることを示す.$L$が全単射であるため,任意の$g \in X$に対して
  \begin{equation*}
    \phi = L^{-1}g
  \end{equation*}
  となる解$\phi \in X$が一意に存在する.よって,代数的直和による分解の一意性から$P\phi \in X_1$と$(I-P)\phi \in X_2$が一意に存在する.そのうえで,
  \begin{equation*}
    \begin{split}
      L\phi=0 &\Leftrightarrow
      \begin{pmatrix}
        T & B \\
        C & D
      \end{pmatrix}
      \begin{pmatrix}
        P \phi \\
        (I-P) \phi
      \end{pmatrix}
      =
      \begin{pmatrix}
        Pg \\
        (I-P)g
      \end{pmatrix}
      \\&\Leftrightarrow
      \begin{pmatrix}
        I_{X_1} & T^{-1}B \\
        0 & S
      \end{pmatrix}
      \begin{pmatrix}
        P \phi \\
        (I-P) \phi
      \end{pmatrix}
      =
      \begin{pmatrix}
        T^{-1} & 0 \\
        -CT^{-1} & I_{X_2}
      \end{pmatrix}
      \begin{pmatrix}
        Pg \\
        (I-P)g
      \end{pmatrix}
    \end{split}
  \end{equation*}
  となる.さらに,第二式
  \begin{equation*}
    S(I-P)\phi = -CT^{-1}Pg + (I-P)g
  \end{equation*}
  となる.ここで,任意の$g \in X$に対して$L\phi=g$となる解$\phi \in X$が存在し,$\phi = P\phi + (I-P)\phi$のように一意に分解できることに注意すると,任意の$g_2 \in X_2$に対しても$L\phi=g_2$となる解$\phi \in X$が存在し,$\phi = P\phi + (I-P)\phi$となる$(I-P)\phi \in X_2$が存在する.そのうえで,
  \begin{equation*}
    S(I-P)\phi = -CT^{-1}Pg_2 + (I-P)g_2=g_2
  \end{equation*}
  となる解$(I-P)\phi \in X_2$が存在するため,$S$は全射である.
\end{proof}


\chapter{提案手法}
\label{cpt:提案手法}

\section{概要}

\rad{}が,$l^1$空間で定義したBanach空間上で適用可能である条件は,$DF(\bar{x})$が全単射なことである.

そのために,無限次元ガウスの消去法を用いて,$DF(\bar{x})$が全単射であることを確かめる.

なお,提案手法で用いるBanach空間は,以下のように定義する.

\begin{dfn}[許容重みなしBanach空間$X$]
  \label{dfn:Banach-l1}
  Banach空間$X$を次のように定める.はじめに$l^1$空間を次のように定める.
  \begin{equation}
    l^1 := \left\{ a=(a_k)_{k\in \mathbb{Z}} : a_k \in \mathbb{C}, \|a\| := \sum_{k \in \mathbb{Z}} |a_k| < \infty \right\}
  \end{equation}
  そして,検証に用いる関数空間$X$は
  \begin{equation}
    X:= \mathbb{C} \times l^1, x = (\omega, a), \omega \in \mathbb{C}, a \in l^1
  \end{equation}
  と定め,そのノルムを
  \begin{equation}
    \| x \|_X := \max \{ |\omega| , \|a\| \}
  \end{equation}
  として定義する.このとき,$X$はBanach空間となる.
\end{dfn}

\section{無限次元ガウスの消去法}
$\left \| DF(\bar{x})^{-1} F(\bar{x}) \right \|$について,無限次元ガウスの消去法(節\ref{sct:無限次元ガウスの消去法})を用いる.

\begin{equation*}
  \phi:=DF(\bar{x})^{-1} F(\bar{x})
\end{equation*}
とおくと,
\begin{equation*}
  DF(\bar{x}) \phi = F(\bar{x})
\end{equation*}
と変形できる.$\Pi_N$を射影演算子とすると,
\begin{equation}
    \begin{split}
    \begin{pmatrix}
      \Pi_N DF(\bar{x}) \Pi_N & \Pi_N DF(\bar{x}) (I-\Pi_N) \\
      (I-\Pi_N) DF(\bar{x}) \Pi_N & (I-\Pi_N) DF(\bar{x}) (I-\Pi_N)
    \end{pmatrix}
    \begin{pmatrix}
      \Pi_N \phi \\
      (I-\Pi_N) \phi
    \end{pmatrix}
    \\=
    \begin{pmatrix}
      \Pi_N F(\bar{x}) \\
      (I - \Pi_N) F(\bar{x})
    \end{pmatrix}
  \end{split}
\end{equation}
となり,両辺に$A_M$を掛けて
\begin{equation}
  \label{eq:y0-1}
  \begin{split}
  A_M
    \begin{pmatrix}
      \Pi_N DF(\bar{x}) \Pi_N & \Pi_N DF(\bar{x}) (I-\Pi_N) \\
      (I-\Pi_N) DF(\bar{x}) \Pi_N & (I-\Pi_N) DF(\bar{x}) (I-\Pi_N)
    \end{pmatrix}
    \begin{pmatrix}
      \Pi_N \phi \\
      (I-\Pi_N) \phi
    \end{pmatrix}
    \\=A_M
    \begin{pmatrix}
      \Pi_N F(\bar{x}) \\
      (I - \Pi_N) F(\bar{x})
    \end{pmatrix}
  \end{split}
\end{equation}
となる.

ここで,線形作用素$T,B,C,E$それぞれに対し,
\begin{equation*}
    \label{eq:def_lo}
    \begin{split}
    T:= \Pi_N A_M DF(\bar{x}) \mid _{X_1}:X_1 \rightarrow X_1,\quad &
    B:= \Pi_N A_M DF(\bar{x}) \mid _{X_2}:X_2 \rightarrow X_1, \\
    C:= (I-\Pi_N) A_M DF(\bar{x}) \mid _{X_1}:X_1 \rightarrow X_2,\quad &
    E:= (I-\Pi_N) A_M DF(\bar{x}) \mid _{X_2}:X_2 \rightarrow X_2
  \end{split}
\end{equation*}
と定義でき,式\eqref{eq:y0-1}を次のように変形できる.
\begin{equation}
  \begin{pmatrix}
    T & B \\
    C & E
  \end{pmatrix}
  \begin{pmatrix}
    \Pi_N \phi \\
    (I -\Pi_N) \phi
  \end{pmatrix}
  =
  \begin{pmatrix}
    \Pi_N A_M F(\bar{x}) \\
    (I - \Pi_N) A_M F(\bar{x})
  \end{pmatrix}
\end{equation}

定理\ref{thm:6.2.1-L全単射}より,線形作用素$S$を
\begin{equation*}
  S:=E-CT^{-1}B:X_2 \rightarrow X_2
\end{equation*}
とする.もし,$S$が全単射ならば,
\begin{equation*}
  \begin{pmatrix}
    \Pi_N \phi \\
    (I-\Pi_N)\phi
  \end{pmatrix}
  =
  \begin{pmatrix}
    T^{-1}+T^{-1}BST^{-1} & -T^{-1}BS^{-1} \\
    -S^{-1}CT^{-1} & S^{-1}
  \end{pmatrix}
  \begin{pmatrix}
    Pg \\
    (I-P)g
  \end{pmatrix}
\end{equation*}
となり,有界線形作用素$DF(x)$は全単射となる.

ここで,
\begin{equation}
  \label{eq:I-S_1}
  \| I_{X_2} -S \| < 1
\end{equation}
であれば,$S$は全単射となる.

なお,$DF(x), A_M$の定義は以下の通りである.
\begin{dfn}[拡張したヤコビ行列$DF(x)$]
  \label{dfn:拡張したヤコビ行列}
  まず,近似解$x$について,ある整数$M$を定め,
  \begin{equation}
    x = (\omega,\ \underbrace{0,\cdots,0}_{M},\ a,\ \underbrace{0,\cdots,0}_{M}) \in \mathbb{C}^{2N+2M}
  \end{equation}
  と定めると,$F(x)$のヤコビ行列は
  \begin{equation}
    DF(x) = \left[
      \begin{array}{c|ccc}
        0 & \cdots & 1 & \cdots \\ \hline
        \vdots & & \vdots &  \\
        \partial_w f_k & \cdots & \partial_{a_j}f_k & \cdots \\
        \vdots & & \vdots &
      \end{array}
    \right] %\in \mathbb{C}^{(2N+2M) \times (2N+2M)}
  \end{equation}
  と定義できる.
\end{dfn}
\begin{dfn}[作用素$A_M$の定義]
  \label{dfn:作用素AM}
  はじめに,ある整数$M$を定め,
  \begin{equation}
    \bar{x} = (\omega,\ \underbrace{0,\cdots,0}_{M},\ a,\ \underbrace{0,\cdots,0}_{M}) \in \mathbb{C}^{2N+2M}
  \end{equation}
  と定めると,$A_M$の定義は,以下のように定義できる.
  \begin{equation*}
    A_M = \left(
    \begin{array}{c|ccc}
      DF(\bar{x})^{-1} & 0 & \cdots & \cdots \\ \hline
      0 & \lambda_N^{-1} &  & 0 \\
      \vdots &  & \lambda_{N+1}^{-1} &  \\
      \vdots & 0 &   & \ddots
    \end{array}
    \right)
  \end{equation*}
\end{dfn}


\section{ノルム値の計算}

定義した作用素$T,B,C,E$より,$S$は以下のように変形できる.
\begin{equation}
  \begin{split}
    S =& E-CT^{-1}B \\
    =& \left( \left( I-\Pi_N \right) A_M DF ( \bar{x} ) \right)\\
    &- (\left( I-\Pi_N \right) A_M DF( \bar{x} )) (\Pi_N A_M DF(\bar{x}))^{-1} ((I-\Pi_N)A_M DF(\bar{x}))
  \end{split}
  \label{eq:s-def}
\end{equation}

また,作用素$T,B,C,E$を使って,式\eqref{eq:I-S_1}を変形する.
\begin{equation}
  \label{eq:calc_norm}
  \begin{split}
    \|I_{X_2} - S \| &= \| I_{X_2} - (E-CT^{-1}B)  \| = \| I_{X_2} - E + CT^{-1}B \| \\
    & \leq \| I_{X_2} - E \| + \| C \| \| T^{-1} \| \| B \| \\
    & < 1
  \end{split}
\end{equation}

ここで,各作用素$A_M,DF(\bar{x})$を同じ大きさのブロックに分割する.
\begin{equation}
  A_M = \left[
    \begin{array}{c|ccc}
      DF(\bar{x})^{-1} & 0 & \cdots & \cdots \\ \hline
      0 & \lambda_N^{-1} &  & 0 \\
      \vdots &  & \lambda_{N+1}^{-1} &  \\
      \vdots & 0 &   & \ddots
    \end{array}
    \right]
    = \left[ \begin{array}{c|c}
        T^{-1} & 0 \\ \hline
        0 & \Lambda
    \end{array}\right]
\end{equation}

\begin{equation}
  DF(\bar{x}) = \left[
    \begin{array}{ccc|ccc}
      0 & 1 & \cdots  & 1 & \cdots \\
      \partial_w f_{k} &  \partial_{a_j}f_k & \cdots & \partial_{a_j}f_k & \cdots \\
      \vdots & \vdots & \ddots & \vdots  & \ddots\\ \hline
      \partial_w f_k & \partial_{a_j}f_k & \cdots & \partial_{a_j}f_k & \cdots \\
      \vdots & \vdots & \ddots & \vdots & \ddots
    \end{array}
  \right]
  = \left[ \begin{array}{c|c}
    T & M_{01} \\ \hline
    M_{10} & M_{11}
\end{array} \right]
\end{equation}

このとき,$A_M DF(\bar{x})$はブロック行列を用いて以下のように表せる.なお,ブロック行列のサイズは$\Pi_N$により定まる.

\begin{equation}
  \begin{split}
    A_M DF(\bar{x}) &= \left[ \begin{array}{c|c}
      T^{-1} & 0 \\ \hline
      0 & \Lambda
    \end{array}\right]
    \left[ \begin{array}{c|c}
      T & M_{11} \\ \hline
      M_{21} & M_{22}
    \end{array} \right]
    = \left[ \begin{array}{c|c}
      T^{-1} T & T^{-1}M_{01} \\ \hline
      \Lambda M_{10} & \Lambda M_{11}
    \end{array} \right] \\
    &= \left[ \begin{array}{c|c}
      I & B \\ \hline
      C & E
    \end{array} \right]
  \end{split}
\end{equation}

式\eqref{eq:calc_norm}の各項を,ブロック行列と定義\ref{dfn:拡張したヤコビ行列},\ref{dfn:作用素AM},節\ref{sec:フーリエスペクトル法}を用いて簡単にする.

\begin{enumerate}
  \item[1. $\|I_{X_2} - E \|$]

\begin{equation}
  \tiny
  \begin{split}
    &M_{11}\\
    &=\begin{bmatrix}
      \lambda_{N} & \mu i N \omega (a*a)_{-1} & \cdots & \mu i N \omega (a*a)_{-2N+2} & 0 & \cdots \\
      \mu i (N+1) \omega (a*a)_{1} & \lambda_{N+1} & \cdots & \mu i (N+1) \omega (a*a)_{-2N+1} & \mu i (N+1) \omega (a*a)_{-2N+2} & \cdots \\
      \vdots & \ddots & \ddots & \ddots & \ddots & \ddots \\
      \mu i (3N-2) \omega (a*a)_{2N-2} & \mu i (3N-2) \omega (a*a)_{2N-1} & \ddots & \lambda_{3N-2} & \mu i (3N-2) \omega (a*a)_{-1} & \cdots \\
      0 & \mu i (3N-1) \omega (a*a)_{2N-2} & \ddots & \mu i (3N-1) \omega (a*a)_{1} & \ddots & \ddots \\
      \vdots & \vdots & \vdots & \vdots & \vdots & \ddots \\
    \end{bmatrix}
  \end{split}
\end{equation}
より,

\begin{equation}
  \scriptsize
  \begin{split}
    \|I_{X_2} - E \| =& \| I - \Lambda M_{11}\| \\
    =& \left\| I -
    \begin{bmatrix}
      \lambda^{-1}_N & & \mathbf{0} \\
      & \lambda^{-1}_{N+1} &\\
      \mathbf{0} &  & \ddots
    \end{bmatrix}
    \begin{bmatrix}
      \partial_{a_{N}} f_{N} & \partial_{a_{N+1}} f_{N}  & \cdots \\
      \partial_{a_{N}} f_{N+1} & \partial_{a_{N+1}} f_{N+1}  & \cdots \\
      \vdots & \vdots & \ddots \\
    \end{bmatrix}
    \right\|\\
    =& \left \|
    I - \begin{bmatrix}
      1 & \partial_{a_{N+1}} f_{N}  & \cdots & \lambda^{-1}_{N}\partial_{a_{3N-2}}f_N & 0 & \cdots \\
      \lambda^{-1}_{N+1}\partial_{a_{N}} f_{N+1} & 1  & \ddots & \ddots & \lambda^{-1}_{N+1}\partial_{a_{3N-3}}f_{N+1} & \ddots \\
      \vdots & \ddots & \ddots & \ddots & \ddots & \ddots \\
      \lambda^{-1}_{3N-2}\partial_{a_{N}} f_{3N-2} & \ddots & \ddots & 1 & \ddots & \ddots\\
      0 & \lambda^{-1}_{3N-1}\partial_{a_{N+1}} f_{3N-1}& \ddots & \ddots & 1 & \ddots\\
      \vdots & \ddots & \ddots & \ddots & \ddots & \ddots\\
    \end{bmatrix}
    \right \|\\
    =&\left\|
    \begin{bmatrix}
      0 & \partial_{a_{N+1}} f_{N}  & \cdots & \lambda^{-1}_{N}\partial_{a_{3N-2}}f_N & 0 & \cdots \\
      \lambda^{-1}_{N+1}\partial_{a_{N}} f_{N+1} & 0  & \ddots & \ddots & \lambda^{-1}_{N+1}\partial_{a_{3N-3}}f_{N+1} & \ddots \\
      \vdots & \ddots & \ddots & \ddots & \ddots & \ddots \\
      \lambda^{-1}_{3N-2}\partial_{a_{N}} f_{3N-2} & \ddots & \ddots & 0 & \ddots & \ddots\\
      0 & \lambda^{-1}_{3N-1}\partial_{a_{N+1}} f_{3N-1}& \ddots & \ddots & 0 & \ddots\\
      \vdots & \ddots & \ddots & \ddots & \ddots & \ddots\\
    \end{bmatrix}
    \right \|
  \end{split}
\end{equation}

\begin{equation}
  \begin{split}
    =& \sup \left\{ \right. \\
    &\quad \|\lambda^{-1}_{N+1}\partial_{a_{N}} f_{N+1}\| + \cdots + \|\lambda^{-1}_{3N-2}\partial_{a_{N}} f_{3N-2}\|, \\
    &\quad \|\lambda^{-1}_{N} \partial_{a_{N+1}} f_{N}\| + \cdots + \|\lambda^{-1}_{3N-1}\partial_{a_{N+1}} f_{3N-1}\|,\\
    &\quad \cdots\\
    \left.\right\}\\
    =& \sup \left\{ \right. \\
    &\quad \underbrace{0+ \|\lambda^{-1}_{N+1} \mu i (N+1) \omega (a*a)_{1}\| + \cdots + \|\lambda^{-1}_{3N-2} \mu i (3N-2) \omega (a*a)_{2N-2}\|}_{2N-1}, \\
    &\quad \underbrace{\|\lambda^{-1}_{N}  \mu i N \omega (a*a)_{-1}\| + \cdots + \|\lambda^{-1}_{3N-1} \mu i (3N-1) \omega (a*a)_{2N-2}\|}_{2N},\\
    &\quad \cdots, \\
    &\quad \underbrace{\|\lambda^{-1}_{N}  \mu i N \omega (a*a)_{-2N+2}\| + \cdots + \|\lambda^{-1}_{4N-3} \mu i (4N-3) \omega (a*a)_{2N-2}\|}_{4N-3},\\
    &\quad \underbrace{\|\lambda^{-1}_{N+1}  \mu i (N+1) \omega (a*a)_{-2N+2}\| + \cdots + \|\lambda^{-1}_{4N-2} \mu i (4N-2) \omega (a*a)_{2N-2}\|}_{4N-3},\\
    &\quad \cdots \\
    \left.\right\}
  \end{split}
\end{equation}

上記の上限値は,無限個の要素を比較しなければならない.コンピュータで計算するためには,要素の個数を有限まで小さくする必要がある.そこで,以下の2つの要素
\begin{align}
  &\|\lambda^{-1}_{k}  \mu i k \omega (a*a)_{-2N+2}\| + \cdots + \|\lambda^{-1}_{4k-3} \mu i (4k-3) \omega (a*a)_{2N-2}\|,\\
  &\|\lambda^{-1}_{k+1}  \mu i (k+1) \omega (a*a)_{-2N+2}\| + \cdots + \|\lambda^{-1}_{4k-2} \mu i (4k-2) \omega (a*a)_{2N-2}\|
\end{align}
を比較する.$k\geq N$として,
\begin{equation}
  \begin{split}
    &\|\lambda^{-1}_{k}  \mu i k \omega (a*a)_{-2N+2}\| + \cdots + \|\lambda^{-1}_{4k-3} \mu i (4k-3) \omega (a*a)_{2N-2}\|\\
    &-\|\lambda^{-1}_{k+1}  \mu i (k+1) \omega (a*a)_{-2N+2}\| + \cdots + \|\lambda^{-1}_{4k-2} \mu i (4k-2) \omega (a*a)_{2N-2}\|\\
    =& \left\| \lambda^{-1}_{k}k - \lambda^{-1}_{k+1}(k+1) \right\| \| \mu i \omega (a*a)_{-2N+2}\|+ \cdots \\
    &+ \left\| \lambda^{-1}_{4k-3}(4k-3) - \lambda^{-1}_{4k-2}(4k-2) \right\| \| \mu i \omega (a*a)_{2N-2}\|
  \end{split}
\end{equation}

ここで $ \|\lambda^{-1}_{k}k - \lambda^{-1}_{k+1}(k+1) \|$を計算する.
\begin{equation}
  \begin{split}
    &\|\lambda^{-1}_{k}k - \lambda^{-1}_{k+1}(k+1) \|\\
    =& \left\|\frac{k}{-k^2\omega^2 + \mu i k \omega+ 1} - \frac{k+1}{-(k+1)^2\omega^2 + \mu i (k+1) \omega+ 1}\right\|\\
    =& \left\|\frac{1}{-k\omega^2 + \mu i  \omega+ k^{-1}} - \frac{1}{-(k+1)\omega^2 + \mu i \omega+ (k+1)^{-1}}\right\|\\
    >& 0
  \end{split}
\end{equation}
よって,
\begin{equation}
  \begin{split}
    &\|\lambda^{-1}_{k}  \mu i k \omega (a*a)_{-2N+2}\| + \cdots + \|\lambda^{-1}_{4k-3} \mu i (4k-3) \omega (a*a)_{2N-2}\|\\
    &-\|\lambda^{-1}_{k+1}  \mu i (k+1) \omega (a*a)_{-2N+2}\| + \cdots + \|\lambda^{-1}_{4k-2} \mu i (4k-2) \omega (a*a)_{2N-2}\|\\
    &> 0
  \end{split}
\end{equation}

以上より,要素を省略できて,

\begin{equation}
  \begin{split}
    &\|I_{X_2}-S\|= \sup \left\{ \right. \\
    &\quad \underbrace{0+ \|\lambda^{-1}_{N+1} \mu i (N+1) \omega (a*a)_{1}\| + \cdots + \|\lambda^{-1}_{3N-2} \mu i (3N-2) \omega (a*a)_{2N-2}\|}_{2N-1}, \\
    &\quad \underbrace{\|\lambda^{-1}_{N}  \mu i N \omega (a*a)_{-1}\| + \cdots + \|\lambda^{-1}_{3N-1} \mu i (3N-1) \omega (a*a)_{2N-2}\|}_{2N},\\
    &\quad \cdots, \\
    &\quad \underbrace{\|\lambda^{-1}_{N}  \mu i N \omega (a*a)_{-2N+2}\| + \cdots + \|\lambda^{-1}_{4N-3} \mu i (4N-3) \omega (a*a)_{2N-2}\|}_{4N-3}\\
    &\left.\right\}
  \end{split}
\end{equation}

\item[$\|C\|$]

\begin{equation}
  \tiny
  \begin{split}
    &M_{10}\\
    &= \begin{bmatrix}
      3^{-1}\mu i N (a*a*a)_{N} & 0 & \mu i N \omega (a*a)_{2N-2} & \mu i N \omega (a*a)_{2N-1} & \cdots & \mu i N \omega (a*a)_1 &\\
      3^{-1}\mu i (N+1) (a*a*a)_{N+1} & 0 & 0 & \mu i (N+1) \omega (a*a)_{2N-2} & \cdots & \mu i (N+1) \omega (a*a)_2 &\\
      \vdots & \vdots & \vdots & \ddots & \ddots & \vdots\\
      3^{-1}\mu i (3N-3) (a*a*a)_{3N-3} & 0 & 0 & 0 & \cdots & 0\\
      0 & 0 & 0 & 0 & \cdots & 0 \\
      \vdots & \vdots & \vdots & \vdots & \ddots & \vdots\\
    \end{bmatrix}
  \end{split}
\end{equation}
より,
% lambda M10
\begin{equation}
  \tiny
  \begin{split}
    &\|\Lambda M_{10}\| \\
    &= \left\| \begin{bmatrix}
      \lambda^{-1}_N & & \mathbf{0} \\
      & \lambda^{-1}_{N+1} & \\
      \mathbf{0} &  & \ddots &
    \end{bmatrix}
    \begin{bmatrix}
      \partial_{\omega} f_{N} & \partial_{a_{-N+1}} f_{N} & \cdots \\
      \partial_{\omega} f_{N+1} & \partial_{a_{-N+1}} f_{N+1}  & \cdots \\
      \vdots & \vdots & \ddots \\
    \end{bmatrix} \right\| \\
    &= \left\|  \begin{bmatrix}
      \lambda^{-1}_N & & \mathbf{0} \\
      & \lambda^{-1}_{N+1} & \\
      \mathbf{0} &  & \ddots &
    \end{bmatrix}\right.\\
    & \left.\begin{bmatrix}
      3^{-1}\mu i N (a*a*a)_{N} & 0 & \mu i N \omega (a*a)_{2N-2} & \mu i N \omega (a*a)_{2N-1} & \cdots & \mu i N \omega (a*a)_1 &\\
      3^{-1}\mu i (N+1) (a*a*a)_{N+1} & 0 & 0 & \mu i (N+1) \omega (a*a)_{2N-2} & \cdots & \mu i (N+1) \omega (a*a)_2 &\\
      \vdots & \vdots & \vdots & \ddots & \ddots & \vdots\\
      3^{-1}\mu i (3N-3) (a*a*a)_{3N-3} & 0 & 0 & 0 & \cdots & 0\\
      0 & 0 & 0 & 0 & \cdots & 0 \\
      \vdots & \vdots & \vdots & \vdots & \ddots & \vdots\\
    \end{bmatrix}\right\| \\
  \end{split}
\end{equation}
\begin{equation}
  \begin{split}
    =& \sup \left\{ \right.\\
    &\quad \lambda_N^{-1} 3^{-1}\mu i N (a*a*a)_{N} + 0 + \lambda_{N+2}^{-1} \mu i N \omega (a*a)_{2N-2} + \cdots + \lambda_{N+2}^{-1} \mu i N \omega (a*a)_{},\\
    &\quad \cdots, \\
    &\quad \lambda_{3N-3}^{-1} 3^{-1}\mu i (3N-3) (a*a*a)_{3N-3} \\
    &\left. \right\}
  \end{split}
\end{equation}

\item[$\|T^{-1}\|$]
\begin{equation}
  \begin{split}
    \| T^{-1}\| = \| DF(\bar{x})^{-1}\|
  \end{split}
\end{equation}

\item[$\|B\|$]
% T^{-1}M_{01}q
\begin{equation}
  \begin{split}
    &\|T^{-1}M_{01}\| \\
    =& \left \| \begin{bmatrix}
      T^{-1}_{0\ 0} & \cdots & T^{-1}_{0\ 2N} \\
      \vdots & \ddots & \vdots \\
      T^{-1}_{2N\ 0} & \cdots & T^{-1}_{2N\ 2N} \\
    \end{bmatrix}
    \begin{bmatrix}
      1 & \cdots & \cdots \\
      \partial_{a_N} f_{-N+1} & \partial_{a_{N+1}} f_{-N+1}  & \cdots \\
      \vdots & \vdots & \vdots \\
      \partial_{a_N} f_{N-1} & \partial_{a_{N+1}} f_{N-1}  & \cdots
    \end{bmatrix}\right\| \\
    =& \left\| \begin{bmatrix}
      T^{-1}_{0\ 0} & \cdots & T^{-1}_{0\ 2N} \\
      \vdots & \ddots & \vdots \\
      T^{-1}_{2N\ 0} & \cdots & T^{-1}_{2N\ 2N} \\
    \end{bmatrix}\right.\\
    &\left. \begin{bmatrix}
      1 & \cdots & \cdots  & \cdots & 1 & \cdots \\
      0 & \cdots & \cdots  & \cdots & 0 & \cdots \\
      \mu i (-N+1) \omega (a*a)_{-2N+2} & 0 & \cdots  & \cdots & 0 & \cdots\\
      \mu i (-N+2) \omega (a*a)_{-2N+1} & \mu i (-N+2) \omega (a*a)_{-2N+2} & 0 & \cdots & 0 & \cdots \\
      \vdots & \ddots & \ddots & \ddots & \ddots & \ddots \\
      \mu i (N-1) \omega (a*a)_{-1} & \mu i (N-1) \omega (a*a)_{-2}  & \cdots & \cdots & 0 & \cdots
    \end{bmatrix}\right\|\\
  \end{split}
\end{equation}
\end{enumerate}

上記のノルム値より,$\|I_{X_2} - S\|$をコンピュータで計算する.


\chapter{実験結果}
\section{実験手法}
4章に示した\vdp{}方程式を問題の対象とし,フーリエ・スペクトル法を用いて近似解を計算する.この近似解と\vdp{}方程式を用いて,提案手法で示したノルム値を計算し,$DF(\bar{x})$が全単射であるか確かめる.このとき,フーリエ・スペクトル法におけるフーリエ係数の次数を50,100,150,200と変更し,ノルム値を計算する.

\section{実験環境}
本研究を行った実験環境を表\ref{tab:環境},使用したパッケージを表\ref{tab:パッケージ}を示す.

\begin{table}[h]
  \centering
  \caption{実験環境}
  \label{tab:環境}
  \begin{tabular}{c||c}
    環境 & 詳細 \\ \hline
    CPU & 12th Gen Intel(R) CORE(TM) i7-12700 \\
    OS & Linux(x86_64-linyc-gnu)\\
    Julia & Version 1.11.2
  \end{tabular}
\end{table}

\begin{table}[h]
  \centering
  \caption{使用パッケージ}
  \label{tab:パッケージ}
  \begin{tabular}{c||l}
    パッケージ & バージョン \\  \hline
    DifferentialEquations & v7.10.0 \\
    FFMPEG & v0.4.1 \\
    FFTW & v1.7.1 \\
    ForwardDiff & v0.10.36 \\
    GenericFFT & v0.1.6 \\
    IntervalArithmetic & v0.20.9 \\
    Plots & v1.39.0
  \end{tabular}
\end{table}

\section{実験結果}
実験結果の数値を表\ref{tab:norm-num}に示す.

結果より,フーリエ係数の次数が大きくなるにつれてノルム値が小さくなることがわかる.
また,いかなるフーリエ係数のサイズであっても,ノルム値が$1$を超えることはなかったことがわかる.これは,フーリエ係数のサイズが大きくなるにつれて,計算における$\lambda^{-1}$が小さくなるため,ノルム値が小さくなるためであると考えられる.

この結果より,$DF(\bar{x})$が全単射であることが確かめられた.

\begin{table}[htbp]
  \centering
  \caption{フーリエ係数の次数の変更による$\|I_{X_2}\|$の比較}
  \label{tab:norm-num}
  \begin{tabular}{c||c}
    フーリエサイズ & $\| I_{X_2}-S \|$ \\ \hline
    50 & 0.22815114629236252 \\
    100&0.11455533660051737\\
    150&0.07655718822651922\\
    200&0.05749210273025131
\end{tabular}
\end{table}

\begin{comment}
\begin{figure}[htbp]
  \centering
  \includegraphics[scale=0.5]{img/scatter.png}
  \caption{フーリエ係数の次数の変更による$\|I_{X_2}\|$の比較}
  \label{fig:norm-gra}
\end{figure}
\end{comment}


\chapter{おわりに}

本研究では,\rad{}を用いた精度保証付き数値計算の問題適用範囲の拡大を目的とした.\rad{}における作用素の計算に,無限次元ガウスの消去法を用いることで,$l_\nu^1$を$l_1$と定義し,許容重みを定義に付加せずに解を導出できるかを検証した.プログラムによる検証により,無限次元ガウスの消去法を用いることで\rad{}の精度の向上が可能であることがわかった.

今後の課題として,本研究では触れなかった評価値$Z_0, Z_1, Z_2(r)r$に対して,無限次元ガウスの消去法を用いて計算ができるか,また無限次元ガウスの消去法を用いた計算手法を導出することについて,検討をする.

\chapter*{謝辞}
本研究を進めるに際して, 千葉工業大学の関根晃太准教授には多くのご指導, ご助言を頂きました
こと深く感謝いたします。そして最後に、本研究に対しご助力頂いたすべての皆様に感謝し、お礼申し上げることで謝辞とさせて頂きます。


\nocite{*}
%\bibliographystyle{junsrt}
%\bibliography{2131701}

%\begin{comment}
\begin{thebibliography}{99}
  \bibitem{b1} 大石進一, 精度保証付き数値計算, コロナ社 (2018)

  \bibitem{github}
  舩越康太, 井藤佳奈子, 大谷俊輔, 近藤慎佑, 高橋和暉, 瀬戸翔太, 二平泰知, 高安亮紀
  \newblock ``Julia 言語を使った精度保証付き数値計算のチュートリアル''
  \newblock \verb|https://github.com/tak-lab/rigorous_numerics_tutorial_julia,|,(2024/11/29)

  \bibitem{r1}関根晃太, 中尾充宏, 大石進一, Numerical verification methods for a system of elliptic PDEs,
  and their software library

  \bibitem{r2}  Kouta Sekine, Mitsuhiro T. Nakao, and Shin'ichi Oishi: “A new formulation using the Schur complement for the numerical existence proof of solutions to elliptic problems: without direct estimation for an inverse of the linearized operator”, Numer. Math., 146, 907–926, Oct., 2020.

\end{thebibliography}
%\end{comment}



\begin{lstlisting}[caption=calc_solve.jl, label=calc_solve,]
  using LinearAlgebra, DifferentialEquations, FFTW
include("FourierChebyshev.jl")

# van der Pol方程式
function vanderpol(du, u, μ, t)
  x, y = u
  du[1] = y
  du[2] = μ * (1 - x^2) * y - x
end

# F^(N)(x_n)
function F_fourier(x, μ, η₀)
  N = length(x) / 2
  ω = x[1]
  a = x[2:end]
  (a³, ~) = powerconvfourier(a, 3)
  eta = sum(a) - η₀

  k = -(N - 1):(N-1)
  f = (-k .^ 2 * ω^2 - μ * im * k * ω .+ 1) .* a + μ * im * k * ω .* a³ / 3

  return [eta; f]
end

function DF_fourier(x, μ)
  N = Int((length(x)) / 2)
  ω = x[1]
  a = x[2:end]
  k = (-N+1):(N-1)
  (a³, ~) = powerconvfourier(a, 3)

  DF = zeros(ComplexF64, 2N, 2N)

  DF[1, 2:end] .= 1
  DF[2:end, 1] = (-2 * ω * k .^ 2 - μ * im * k) .* a + μ * im * k .* a³ / 3

  (~, a2) = powerconvfourier(a, 2)

  M = zeros(ComplexF64, 2 * N - 1, 2 * N - 1)

  for j = (-N+1):(N-1)
    M[k.+N, j+N] = μ * im * k * ω .* a2[k.-j.+(2*N-1)]
  end

  L = diagm(-k .^ 2 * ω^2 - μ * im * k * ω .+ 1)

  DF[2:end, 2:end] = L + M
  return DF
end

# a function of  fourier coeffs
function odefouriercoeffs(f, N, I, n=1)
  a = I[1]
  b = I[2]
  # x_j: equidistance node points
  h = (b - a) / (2N - 1)
  j = 0:2N-2
  xⱼ = a .+ j * h
  # f_j: function values on node points
  fⱼ = f(xⱼ)[n, :]
  return (fftshift(fft(fⱼ))) / (2 * N - 1)
end

# たぶん,初期値設定
u_0 = [0.0; 2.0]
tspan = (0.0, 300)
μ = 1.0
prob = ODEProblem(vanderpol, u_0, tspan, μ)
sol = solve(prob, Tsit5(), reltol=1e-8, abstol=1e-8)
u = hcat(sol.u...)
ind = floor(Int, length(sol.t) / 2)

# おおよその周期
#a = 30
#b = 36.55
a = 30
app_period = 6.55
timestep = 0.1

f_tmp = sol(a+app_period/2:timestep:a+3*app_period/2)
find_period = abs.(f_tmp .- sol(a))
(~, ind) = findmin(find_period[1, :])
b = a + app_period / 2 + timestep * (ind - 1)
#calc fouriercoeffs

N = 200 # size of Fourier

println("size of Fourier = $N")
a_0 = odefouriercoeffs(sol, N, [a, b])

include("IntervalFunctions.jl")
# Initial value of Newton method
η_0 = 0.0
x = [2 * pi / (b - a); a_0]

# Newton iteration
tol = 5e-12
F = F_fourier(x, μ, η_0)
println("Before step #1, ||F||_1 = $(norm(F,1))")
num_itr = 0

while num_itr ≤ 100
  global x = x - DF_fourier(x, μ) \ F
  global num_itr += 1
  global F = F_fourier(x, μ, η_0)
  # println("After step #$(num_itr), ||F||_1 = $(norm(F,1))")
  if norm(F, 1) < tol
    break
  end
end

# A^(N)
#ix = map(interval, x)
#iω̄ = map(interval, real(x[1]))
#iā = map(interval, x[2:end])

ix = x
iω̄ = real(x[1])
iā = x[2:end]

function DF_fourier(x::Vector{Complex{Interval{T}}}, μ) where {T}
  N = Int((length(x)) / 2)
  ω = x[1]
  a = x[2:end]
  k = (-N+1):(N-1)

  println("pcf Input: ", a, size(a), typeof(a))

  (a³, ~) = powerconvfourier(a, 3)
  DF = zeros(Complex{Interval{T}}, 2N, 2N)
  DF[1, 2:end] .= 1
  DF[2:end, 1] = (-2 * ω * k .^ 2 - μ * im * k) .* a + μ * im * k .* a³ / 3
  (~, a2) = powerconvfourier(a, 2)
  M = zeros(Complex{Interval{T}}, 2 * N - 1, 2 * N - 1)
  for j = (-N+1):(N-1)
    M[k.+N, j+N] = μ * im * k * ω .* a2[k.-j.+(2*N-1)]
  end
  L = diagm(-k .^ 2 * ω^2 - μ * im * k * ω .+ 1)
  DF[2:end, 2:end] = L + M
  return DF
end

iDF = DF_fourier(ix, μ);
iA = inv(iDF) # map(Interval,inv(mid.(iDF)))

## =======================
## I get a and omega by x.
## =======================
"""
println("x = $x")
println("μ = $μ")
println("N = $N")

μ
"""

omega = x[1]
a = x[2:end]
mu = μ

extend_size1 = (3 * N + 1)
extend_size2 = extend_size1 * 2 + 1
topleft_size = extend_size1

# define DF[]
zero_padding = zeros(ComplexF64, Int(extend_size2))
extend_x = vcat(omega, zero_padding, a, zero_padding)
extend_DF2 = DF_fourier(extend_x, mu)


# define A
zero_padding = zeros(ComplexF64, Int(extend_size1))
extend_x = vcat(omega, zero_padding, a, zero_padding)
extend_DF1 = DF_fourier(extend_x, mu)
T_inv = inv(extend_DF1)

## lambda 行列
topleft_size = size(T_inv)[1]
DF_bottomright = extend_DF2[topleft_size+1:end, topleft_size+1:end]
inv_lambda = inv(diagm(diag(DF_bottomright)))

# norm_D
D = inv_lambda * DF_bottomright
I_minus_D = 1.0I(size(D)[1]) - D
normI_minus_D = maximum(sum(abs.(I_minus_D), dims=1))

# norm_C
DF_bottomleft = extend_DF2[topleft_size+1:end, 1:topleft_size]
C = inv_lambda * DF_bottomleft
normC = maximum(sum(abs.(C), dims=1))

# norm_B
DF_topright = extend_DF2[1:topleft_size, topleft_size+1:end]
B = T_inv * DF_topright
normB = maximum(sum(abs.(B), dims=1))
normDF_topright = maximum(sum(abs.(DF_topright), dims=1))
normT_inv = maximum(sum(abs.(T_inv), dims=1))

println("||I - D|| = ", normI_minus_D)
println("||C|| = ", normC)
println("||T^-1|| = ", normT_inv)
println("||B|| = ", normB)

norm_result = normI_minus_D + normC * normT_inv * normB

println("||I - D|| + ||C|| ||T^-1|| ||B|| = ", norm_result)


\end{lstlisting}

%% 定義 dfn
%% 定理 thm
%% 証明 proof
%% 補題 lmm

%---------------

%%文章のあとに,\cite{label1, label2},\cite{lite:xxx}とつける
%隙間ができたら,\newlineを入れ込む
%% rpa = radii polynomial approach

\end{document}
