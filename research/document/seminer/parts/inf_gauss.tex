%ctrl+alt+b -> build, ctrl+alt+v => preview at new tab
%ctrl+click -> highlight line

%\documentclass[titlepage]{jsarticle}
\documentclass[11pt,a4paper]{jsarticle}
%\documentclass[11pt,a4paper]{jsreport}

\usepackage{algorithmic}
\usepackage{amsmath,amssymb,amsfonts,amsthm,,mathtools}
\usepackage{ascmac}
\usepackage{bm}
\usepackage{caption}
\usepackage{cite}
\usepackage{comment}
\usepackage[dvipdfmx]{color}
\usepackage{colortbl}
\usepackage{float}
\usepackage[dvipdfmx]{graphicx}
\usepackage{multicol}
\usepackage{latexsym}
\usepackage{listings}
\usepackage[dvipdfmx]{pict2e}
\usepackage[ipaex]{pxchfon}
\usepackage{tabularx}
\usepackage{textcomp}
\usepackage{underscore}
\usepackage{ulem}
\usepackage{url}
\usepackage{wrapfig}
\usepackage{xcolor}

\captionsetup[figure]{labelsep=space}
\captionsetup[table]{labelsep=space}

\theoremstyle{definition}
\newtheorem{dfn}{定義}
\newtheorem{thm}{定理}
\newtheorem{lmm}{補題}

\renewcommand{\qedsymbol}{$\blacksquare$}
\renewcommand{\proofname}{\textbf{証明}}
%============

%\newtheo

%============
\begin{document}

\textcolor{blue}{(とりあえず)}

\begin{align*}
   & A \simeq DF^{-1} \rightarrow A = DF^{-1} \\
   & \phi = DF^{-1}F(\tilde{x})               \\
   & DF\phi = F(\tilde{x})                    \\
   & \left\{ \,
  \begin{aligned}
     & \Pi_N DF ( \Pi_N \phi + (I-\Pi_N) \phi ) = \Pi_N F(\tilde{x})             \\
     & (I-\Pi_N) DF (\Pi_N DF (\Pi_N + (I-\Pi_N) \phi)) = (I-\Pi_N) F(\tilde{x})
  \end{aligned}
  \right.                                     \\
   & ||DF^-1(x)F(\tilde{x})||_X \leq Y_0
   & DF \phi = F(\tilde{x})
\end{align*}

\section{無限次元ガウスの消去法}

\begin{dfn}[代数的直和]
  Banach空間$X$とする.また,$X_1,X_2$をXの線形部分空間とする.ただし,$X_1$と$X_2$のノルムは,$X$のノルムと同一とする.そのとき,$X$が$X_1$と$X_2$の代数的直和であるとは
  \begin{itemize}
    \item $X=X_1+X_2:=\{x_1+x_2 | \  \forall x_1 \in X_1, \ \forall x_2 \in X_2 \}$
    \item $X_1 \cap X_2 = \{0\}$
  \end{itemize}
  が成立することをいう.
\end{dfn}

\begin{dfn}[射影]
  $X$をノルム空間とする.定義域を$X$とした$X$上の線形作用素$P$が
  \begin{equation}
    P^2=P
  \end{equation}
  となるとき,線形作用素,あるいは,単に射影と呼ぶ.
\end{dfn}

\end{document}
