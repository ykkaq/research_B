  %ctrl+alt+b -> build, ctrl+alt+v => preview at new tab
%ctrl+click -> highlight line

%\documentclass[titlepage]{jsarticle}
\documentclass[11pt,a4paper]{jsarticle}
%\documentclass[11pt,a4paper]{jsreport}

\usepackage{algorithmic}
\usepackage{amsmath,amssymb,amsfonts,amsthm,,mathtools}
\usepackage{ascmac}
\usepackage{bm}
\usepackage{caption}
\usepackage{cite}
\usepackage{comment}
\usepackage[dvipdfmx]{color}
\usepackage{colortbl}
\usepackage{float}
\usepackage[dvipdfmx]{graphicx}
\usepackage{multicol}
\usepackage{latexsym}
\usepackage{listings}
\usepackage[dvipdfmx]{pict2e}
\usepackage[ipaex]{pxchfon}
\usepackage{tabularx}
\usepackage{textcomp}
\usepackage{underscore}
\usepackage{ulem}
\usepackage{url}
\usepackage{wrapfig}
\usepackage{xcolor}

\captionsetup[figure]{labelsep=space}
\captionsetup[table]{labelsep=space}

\theoremstyle{definition}
\newtheorem{dfn}{定義}
\newtheorem{thm}{定理}
\newtheorem{lmm}{補題}

\renewcommand{\qedsymbol}{$\blacksquare$}
\renewcommand{\proofname}{\textbf{証明}}
%============

%\newtheo

%============
\begin{document}

\textcolor{blue}{(とりあえず)}

\begin{align*}
   & A \simeq DF^{-1} \rightarrow A = DF^{-1} \\
   & \phi = DF^{-1}F(\tilde{x})               \\
   & DF\phi = F(\tilde{x})                    \\
   & \left\{ \,
  \begin{aligned}
     & \Pi_N DF ( \Pi_N \phi + (I-\Pi_N) \phi ) = \Pi_N F(\tilde{x})             \\
     & (I-\Pi_N) DF (\Pi_N DF (\Pi_N + (I-\Pi_N) \phi)) = (I-\Pi_N) F(\tilde{x})
  \end{aligned}
  \right.                                     \\
   & ||DF^-1(x)F(\tilde{x})||_X \leq Y_0\\
   & DF \phi = F(\tilde{x})
\end{align*}

\section{無限次元ガウスの消去法}
\subsection{知識}

\begin{dfn}[代数的直和]
  Banach空間$X$とする.また,$X_1,X_2$をXの線形部分空間とする.ただし,$X_1$と$X_2$のノルムは,$X$のノルムと同一とする.そのとき,$X$が$X_1$と$X_2$の代数的直和であるとは
  \begin{itemize}
    \item $X=X_1+X_2:=\{x_1+x_2 | \  \forall x_1 \in X_1, \ \forall x_2 \in X_2 \}$
    \item $X_1 \cap X_2 = \{0\}$
  \end{itemize}
  が成立することをいう.
\end{dfn}

\begin{dfn}[射影]
  $X$をノルム空間とする.定義域を$X$とした$X$上の線形作用素$P$が
  \begin{equation*}
    P^2=P
  \end{equation*}
  となるとき,線形作用素,あるいは,単に射影と呼ぶ.
\end{dfn}

\begin{thm}
  Banach空間$X$がその線形部分空間$X_1,X_2$の代数的直和であるとする.そのとき,$x\in X$について
  \begin{equation*}
    x=x_1+x_2,\ x_1 \in X_1,\ x_2 \in X_2
  \end{equation*}
\end{thm}
とし,$\mathcal{D}(P)=X$となる$X$上の線形作用素$P$を
\begin{equation*}
  Px=x_1,\ (I-P)x=x_2
\end{equation*}
とすると,線形作用素$P$と$I-P$は射影になる.

\textcolor{blue}{>}

\begin{thm}
  $X$をBanach空間,$X_1,X_2$を$X$の代数的直和となる線形部分空間,$\mathcal{L}(x,y)$を有界な線形作用素集合とする.$L\in\mathcal{L}(x,y)$と
  L g P phi_1 phi_2 T B C E
\end{thm}

\textcolor{blue}{>>}



\subsection{計算}
radii-polynomal approachの$Y_0$の評価式に,無限次元ガウスの消去法を用いる.

$X,Y$をBanach空間,$\mathcal{L}(X,Y)$を$X$から$Y$への有界線形作用素,$A^\dagger \in \mathcal{L}(X,Y),\ A \in \mathcal{L}(Y,X)$を考え,作用素$F:X \rightarrow Y$がC$^1$-Fr\'{e}chet微分可能とする.ここで,$\tilde{x}\in X$に対して,正定数$Y_0$が存在して,次の式を考える.
\begin{equation}
  \left\| AF(\bar{x}) \right\|_X \leq Y_0
  \label{eq:y0}
\end{equation}

$A=DF(\bar{x})$とすると,式(\ref{eq:y0})より,
\begin{equation}
  \left\| AF(\bar{x}) \right\|_X = \left\| DF(\bar{x}) F(\bar{x}) \right\| \leq Y_0
  \label{eq:y1}
\end{equation}

となる.$\phi:=DF(\bar{x}) F(\bar{x})$とし,これを変形する.$\Pi_N$を射影演算子とすると,$DF(\bar{x}) \phi = F(\bar{x})$  より,
\begin{equation}
    \begin{pmatrix}
      \Pi_N DF(\bar{x}) \Pi_N & \Pi_N DF(\bar{x}) (I-\Pi_N) \\
      (I-\Pi_N) DF(\bar{x}) \Pi_N & (I-\Pi_N) DF(\bar{x}) (I-\Pi_N)
    \end{pmatrix}
    \begin{pmatrix}
      \Pi_N \phi \\
      (I-\Pi_N) \phi
    \end{pmatrix}
    =
    \begin{pmatrix}
      \Pi_N F(\bar{x}) \\
      (I - \Pi_N) F(\bar{x})
    \end{pmatrix}
\end{equation}

となり,両辺に$A$をかけて
\begin{equation}
  A
  \begin{pmatrix}
    \Pi_N DF(\bar{x}) \Pi_N & \Pi_N DF(\bar{x}) (I-\Pi_N) \\
    (I-\Pi_N) DF(\bar{x}) \Pi_N & (I-\Pi_N) DF(\bar{x}) (I-\Pi_N)
  \end{pmatrix}
  \begin{pmatrix}
    \Pi_N \phi \\
    (I-\Pi_N) \phi
  \end{pmatrix}
  =A
  \begin{pmatrix}
    \Pi_N F(\bar{x}) \\
    (I - \Pi_N) F(\bar{x})
  \end{pmatrix}
\end{equation}

となる.ここで,線形作用素$T,B,C,E$それぞれに対し,
\begin{equation}
  \begin{split}
    T:= \Pi_N ADF(\bar{x}) \mid _{X_1}:X_1 \rightarrow X_1,\quad & B:= \Pi_N ADF(\bar{x}) \mid _{X_2}:X_2 \rightarrow X_2, \\
    C:= (I-\Pi_N) ADF(\bar{x}) \mid _{X_1}:X_1 \rightarrow X_2,\quad & E:= (I-\Pi_N) ADF(\bar{x}) \mid _{X_2}:X_2 \rightarrow X_1
  \end{split}
\end{equation}

と定義すると,式(3)は以下のように変形できる.

\begin{equation}
  \begin{pmatrix}
    T & B \\
    C & E
  \end{pmatrix}
  \begin{pmatrix}
    \Pi_N \phi \\
    (I -\Pi_N) \phi
  \end{pmatrix}
  =
  \begin{pmatrix}
    \Pi_N F(\bar{x}) \\
    (I - \Pi_N) F(\bar{x})
  \end{pmatrix}
\end{equation}

定理\ref{thm:}より,線形作用素$S$を

\begin{equation}
  S := E-CT^{-1}B:X_2\rightarrow X_2
\end{equation}

とする.もし,$S$が全単射ならば,
\begin{equation*}
  \begin{pmatrix}
    \Pi_N \phi \\
    (I -\Pi_N) \phi
  \end{pmatrix}
  =
  \begin{pmatrix}
    T^{-1}+T^{-1}BS^{-1}CT^{-1} & -T^{-1}BS^{-1} \\
    -S^{-1}CT^{-1} S^{-1} & S^{-1}
  \end{pmatrix}
  \begin{pmatrix}
    \Pi_N F(\bar{x}) \\
    (I - \Pi_N) F(\bar{x})
  \end{pmatrix}
\end{equation*}

となり,有限線形作用素$ADF(\bar{x})$は全単射となる.



\end{document}
