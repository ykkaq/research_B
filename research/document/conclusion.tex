
\chapter{おわりに}

本研究では,\rad{}を用いた精度保証付き数値計算の問題適用範囲の拡大を目的とした.\rad{}における作用素の計算に,無限次元ガウスの消去法を用いることで,$l_\nu^1$を$l_1$と定義し,許容重みを定義に付加せずに解を導出できるかを検証した.プログラムによる検証により,無限次元ガウスの消去法を用いることで\rad{}の精度の向上が可能であることがわかった.

今後の課題として,本研究では触れなかった評価値$Z_0, Z_1, Z_2(r)r$に対して,無限次元ガウスの消去法を用いて計算ができるか,また無限次元ガウスの消去法を用いた計算手法を導出することについて,検討をする.

\chapter*{謝辞}
本研究を進めるに際して, 千葉工業大学の関根晃太准教授には多くのご指導, ご助言を頂きました
こと深く感謝いたします。そして最後に、本研究に対しご助力頂いたすべての皆様に感謝し、お礼申し上げることで謝辞とさせて頂きます。
