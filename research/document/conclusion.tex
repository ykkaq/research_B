
\chapter{おわりに}

本研究では,\rad{}を用いた精度保証付き数値計算の問題条件を緩和可能であるか検証するために,$l_1$空間を用いたBanach空間上で,$DF(\bar{x})$が全単射であるかどうかを確かめた.$DF(\bar{x})$が全単射であるかを確かめるために,無限次元ガウスの消去法を用いた.プログラムによる検証により,$DF(\bar{x})$が全単射であることが確かめられた.この結果より,無限次元ガウスの消去法を用いることで条件が緩和でき,\rad{}の精度の向上が可能であることがわかった.

今後の課題として,本研究では触れなかった評価値$Z_0, Z_1, Z_2(r)r$に対して,無限次元ガウスの消去法を用いて計算ができるか,また無限次元ガウスの消去法を用いた計算手法を導出することについて,検討をする.

\chapter*{謝辞}
本研究を進めるに際して, 千葉工業大学の関根晃太准教授には多くのご指導, ご助言を頂きました
こと深く感謝いたします。そして最後に、本研究に対しご助力頂いたすべての皆様に感謝し、お礼申し上げることで謝辞とさせて頂きます。
