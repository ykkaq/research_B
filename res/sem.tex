%ctrl+alt+b -> build, ctrl+alt+v => preview at new tab
%ctrl+click -> highlight line

%\documentclass[titlepage]{jsarticle}
\documentclass[11pt,a4paper]{jsarticle}
%\documentclass[11pt,a4paper]{jsreport}

\usepackage{algorithmic}
\usepackage{amsmath,amssymb,amsfonts,amsthm}
\usepackage{ascmac}
\usepackage{bm}
\usepackage{caption}
\usepackage{cite}
\usepackage{comment}
\usepackage[dvipdfmx]{color}
\usepackage{colortbl}
\usepackage{float}
\usepackage[dvipdfmx]{graphicx}
\usepackage{multicol}
\usepackage{latexsym}
\usepackage{listings}
\usepackage[dvipdfmx]{pict2e}
\usepackage[ipaex]{pxchfon}
\usepackage{tabularx}
\usepackage{textcomp}
\usepackage{underscore}
\usepackage{ulem}
\usepackage{url}
\usepackage{wrapfig}
\usepackage{xcolor}

\captionsetup[figure]{labelsep=space}
\captionsetup[table]{labelsep=space}

\theoremstyle{difinition}
\newtheorem{dfn}{定義}
\newtheorem{thm}{定理}
\newtheorem{prf}{証明}

\renewcommand{\qedsymbol}{$\blacksquare$}
\renewcommand{\proofname}{\textbf{証明}}
%============

%\newtheo

%============
\begin{document}
\pagestyle{plain}
\title{タイトル}
\author{2131701 齋藤悠希}
\date{}\maketitle

\section{Preparation}
\subsection{Banach Space}
\begin{dfn}[線形空間の公理]
  \label{dfn:線形空間の公理}
  空でない集合$X$が,係数体$\mathbb{K}$上の線形空間であるとは,任意の$u+v \in X$とスカラー$\alpha \in \mathbb{K}$に対して,加法$u+v \in X$とスカラー乗法$\alpha u \in X$が定義されていて,任意の$u,v,w \in X$とスカラー$\alpha, \beta \in \mathbb{K}$に対して次のことが成り立つことである.
  \begin{enumerate}
    \item $(u+v)+w=u+(v+w)$
    \item $u+v=v+u$
    \item $u+0=u$となる$0 \in X$が一意に存在
    \item $u+(-u)=0$となる$-u\in X$が一意に存在
    \item $\alpha(u+v)=\alpha u+\alpha v$
    \item $(\alpha +\beta)u = \alpha u+\beta u$
    \item $(\alpha \beta)u = \alpha (\beta u)$
    \item $1u=u, 1 \in \mathbb{K}$
  \end{enumerate}
\end{dfn}

\begin{dfn}[ノルムとノルム空間の定義]
  $X$を係数体$\mathbb{K}$上の線形空間とする.$X$で定義された関数$||\cdot||:X\rightarrow \mathbb{K}$上で定義された関数が$X$のノルムであるとは
  \begin{enumerate}
    \item $||u||\geq 0, \quad u \in X$
    \item $||u||=0 \Leftrightarrow u=0$
    \item $||\alpha u||=|\alpha| ||u||, \quad (\alpha \in \mathbb{K}, u \in X)$
    \item $||u+v||\leq ||u||+||v||$
  \end{enumerate}
  が成立することである.さらに$X$に1つのノルムが指定されているとき,$X$はノルム空間という.
\end{dfn}

\begin{dfn}[ノルム空間の収束と極限]$X$をノルム空間とする.$X$の点列$(u_n)\subset X$は
  \begin{equation*}
    \forall\epsilon >0, \ \exists N\in \mathbb{N}, \space \forall N \geq Nに対して||u_n-u||<\epsilon
  \end{equation*}
  のとき,点$u\in X$に収束するといい,
  \begin{equation*}
    ||u_n-u||\rightarrow 0, \ \left(n\rightarrow\infty\right)
  \end{equation*}
  と表す.このとき,$u$を$u_n$の極限といい,
  \begin{equation*}
    u_n-u, \ (n\rightarrow \infty)
  \end{equation*}
  と表す.
\end{dfn}

\begin{dfn}[Cauchy列]
  $X$をノルム空間とする.そのとき$X$がCauchy列であるとは
  \begin{equation*}
    u_n-u_m\rightarrow 0, \ \left(n,m\rightarrow\infty \right)
  \end{equation*}
  が成立することである.即ち
  \begin{equation*}
    ||u_n-u_m||\rightarrow 0, \ \left(n,m\rightarrow \infty\right)
  \end{equation*}
  が成立することである.
\end{dfn}

\begin{dfn}[完備]
  $X$をノルム空間とする.$X$が完備であるとは,任意のCauchy列$(u_n)$が$X$の中で極限をもつことである.すなわち,任意のCauchy列$(u_n\subset X)$が
  \begin{equation*}
    \|u_n-u\|\rightarrow 0\ ,\left(n\rightarrow 0\right)
  \end{equation*}
  となる極限$u$を$X$内に持つことである.
\end{dfn}

\begin{dfn}[Banach空間]
  ノルム空間$X$がBanach空間であるとは,$X$が完備であることである.
\end{dfn}

\begin{thm}[逆三角不等式]
  \label{thm1}
  $X$をノルム空間とする.任意の$u,v\in X$について次の不等式を満たす.
  \begin{equation*}
    |\|u\|-\|v\||\leq\|u-v\|
  \end{equation*}
\end{thm}

\begin{proof}
  任意の$u,v\in X$について
  \begin{align*}
    \|u\| & =\|u-v+v\|\leq\|u-v\|+\|v\|               \\
    \|v\| & =\|v-u+u\|\leq\|v-u\|+\|u\|=\|u-v\|+\|u\|
  \end{align*}
  となる.よって
  \begin{equation*}
    \|u\|-\|v\|\leq\|u-v\| \\
    \|v\|-\|u\|\leq\|u-v\|
  \end{equation*}
  となるため,
  \begin{equation*}
    |\|u\|-\|v\||\leq\|u-v\|
  \end{equation*}
  を持つ.
\end{proof}

\begin{dfn}[有界列]
  $X$をノルム空間とする.そのとき$X$の点列$(u_n)$が有界列とは任意の$n\in\mathbb{N}$に対して
  \begin{equation*}
    \|u_n\|\leq M
  \end{equation*}
  となる定数$M>0$が存在することである.
\end{dfn}

\begin{thm}[Cauchy列ならば有界列]
  $X$をノルム空間とする.そのとき$X$の点列$(u_n)$がCauchy列ならば有界列でもある.
\end{thm}

\begin{proof}
  $X$の点列$(u_n)$がCauchy列であるために,$\epsilon -N$論法を用いた表記で
  \begin{equation*}
    \forall\epsilon >0, \ \exists N\in \mathbb{N}, \ \forall n,m \geq Nに対して||u_n-u_m||<\epsilon
  \end{equation*}
  を満たす.$\epsilon=1$としても,それに対応した$N$が存在し,任意の$n\geq N$に対して
  \begin{equation*}
    \|u_n-u_N\|<1
  \end{equation*}
  を満たす.

  任意の$n\geq N$に対して$\|u_n\|$が$\|u_N\|$で評価できることを示す.逆三角不等式である定理\ref{thm1}を用いると
  \begin{equation*}
    |\|u_n\|-\|u_N\|\leq \|u_n-u_N\|<1
  \end{equation*}
  となる.絶対値の性質より$|\|u_n-u_N\||<1$は
  \begin{equation*}
    \|u_N\|-1\leq\|u_n\|<\|u_N\|+1
  \end{equation*}
  となる.よって
  \begin{equation*}
    M=\max\{\|u_1\|,\|u_2\|,\cdots,\|u_{N-1}\|,\|u_N\|+1\}
  \end{equation*}
  とすると,任意の$n\in \mathbb{N}について$
  \begin{equation*}
    \|u_n\|\leq M
  \end{equation*}
  が成り立つため,点列$(u_N)$は有界列である.
\end{proof}

\begin{dfn}[線形部分空間]
  線形空間$X$の空でない集合$M$が任意の元$u,v\in M$と任意の係数体$\alpha\in\mathbb{K}$に対して
  \begin{align*}
    u+v\in M \\
    \alpha u \in M
  \end{align*}
  を満たすとき,$M$は線形空間$X$の線形部分空間と呼ぶ.
\end{dfn}

\begin{dfn}[ノルム空間の開球]
  $X$をノルム空間とする.$x\in X$とし,$r>0$を正実数とする.そのとき,集合
  \begin{equation*}
    B_X(x,r):=\{y\in X \mid \|x-y\|_X<r\}
  \end{equation*}
  を中心$x$,半径$r$の開球という.$X$が明らかな場合は$B_X(x,r)$を省略して$B(x,r)$と表記する.
\end{dfn}

\begin{dfn}[ノルム空間の開集合]
  $X$をノルム空間とし,$M$を$X$の部分集合とする.任意の$x\in M$に対して,$B_X(x,r)\subset M$となる$r>0$が存在する場合,$M$が開集合であるという.
\end{dfn}

\begin{dfn}[ノルム空間の閉集合]
  Xをノルム空間とし,MをXの部分集合とする.Mが閉集合であるとは,Mの任意の点列$(u_n)$の極限$u\in X$がMにも属することである.すなわち,点列$(u_n)\subset M$について
  \begin{equation*}
    u_n\rightarrow u, \quad \left(n\rightarrow \infty\right) \Rightarrow u\in M
  \end{equation*}
  であるとき,Mは閉集合であるという.
\end{dfn}

\begin{dfn}[閉部分空間]
  Xをノルム空間とし,MをXの線形部分空間が閉集合であるとき,Mを閉部分空間である.
\end{dfn}

\subsection{Operator}
\begin{dfn}[作用素]
  ある線形空間Xから別の線形空間Yへの作用素Aとは,
  \begin{equation*}
    \mathcal{D}(A) := \{u \in X \mid Au \in Y\}
  \end{equation*}
  としたとき,$\mathcal{D}(A)$のどんな元に対しても,それぞれ集合Yのただ一つの元を指定する規則のことである.また,$\mathcal{D}(A)$はAの定義域と呼ばれ
  \begin{equation*}
    \mathcal{R}(A) := \{Au \in Y \mid u\in\mathcal{D}(A)\}
  \end{equation*}
  を値域と呼ぶ
\end{dfn}

\begin{dfn}[単射]
  \label{injection}
  線形空間Xから線形空間Yへの作用素Aが
  \begin{equation*}
    u_1 \neq u_2, \quad \forall u_1,u_2 \in \mathcal{D}(A) \Rightarrow A(u_1)\neq A(u_2)
  \end{equation*}
\end{dfn}

\begin{dfn}[全射]
  線形空間Xから線形空間Yへの作用素Aが
  \begin{equation*}
    Y=\mathcal{R}(A)
  \end{equation*}
  を満たすときに作用素Aは全単射であるという.
\end{dfn}

\begin{dfn}[全射]
  線形空間Xから線形空間Yへの作用素Aとし,その定義域を$\mathcal{D}(A)\subset X$,値域を$\mathcal{R}(A)\subset Y$とする.そのとき,
  \begin{align*}
    A^{-1}\left( A\left( u \right) \right) & =u, \  u\in\mathcal{D}\left( A \right) \\
    A(A^{-1}(u))                           & =u, \  u\in\mathcal{R}(A)
  \end{align*}
  かつ
  \begin{align*}
    \mathcal{D}(A^{-1}) & =\mathcal{R}(A) \\
    \mathcal{R}(A^{-1}) & =\mathcal{D}(A)
  \end{align*}
  となるYからXへの作用素$A^{-1}$をAの逆作用素と呼ぶ.
\end{dfn}

\begin{thm}[単射と逆作用素の環境]
  線形空間Xから線形空間Yへの作用素Aとすると.
  \begin{equation*}
    Aが逆作用素を持つ \Leftrightarrow Aが単射である
  \end{equation*}
\end{thm}

\begin{proof}
  「$Aが逆作用素を持つ \Rightarrow Aが単射である$」の証明

  単射の定義\ref{injection}の待遇「$任意のu_1,u_2\in\mathcal{D}(A)に対しA(u_1)=A(u_2) \Rightarrow u_1=u_2$」を満たすことを確かめる.Aの逆作用素を$A^{-1}$とすると,任意の$u_1,u_2\in\mathcal{D}(A)$に対し
  \begin{align*}
                & A(u_1) = A(u_2)               \\
    \Rightarrow & A^{-1}(A(u_1))=A^{-1}(A(u_2)) \\
    \Rightarrow & u_1=u_2
  \end{align*}

  $「Aが単射である\Rightarrow Aが逆作用素A^{-1}をもつ」の証明$

  $A$の値域の定義$\mathcal{R}(A)=\{A(u)\in Y \mid u\in\mathcal{D}(A)\}$より,任意の$v\in\mathcal{R}(A)$に対し,
  \begin{equation*}
    A(u)=v
  \end{equation*}
  となる$u\in\mathcal{D}(A)$が存在する.その上,Aが単射であるため,単射の定義の対偶より$u\in\mathcal{D}(A)$はどんな$u\in\mathcal{R}(A)$に対してもただ一つの元である.そのため,作用素の定義より,上記の$u\in\mathcal{R}(A)$に対してただ一つの元$u\in\mathcal{D}(A)$を指定する規則として
  \begin{equation*}
    B(v)=u
  \end{equation*}
  となる定義域$\mathcal{D}(B)=\mathcal{R}(A)$と値域$\mathcal{R}(B)=\mathcal{D}(A)$となるYからXへの作用素Bが定義できる.その上,$B(v)=u$の$v=A(u)$を代入すると
  \begin{equation*}
    B(A(u))=u
  \end{equation*}
  となる.同様に,$A(u)=v$の$u$に$u=B(v)$を代入すると
  \begin{equation*}
    A(B(v))=v
  \end{equation*}
  となる.よって,定義域$\mathcal{D}(B)=\mathcal{R}(A)$と値域$\mathcal{R}(B)=\mathcal{D}(A)$となるYからXへの作用素BはAの逆作用素であるため,Aは逆作用素を持つ.
\end{proof}

\begin{dfn}[作用素の等号]
  線形空間Xから線形空間Yへの作用素AとBが等しいとは
  \begin{equation*}
    \mathcal{D}(A) = \mathcal{D}(B)
  \end{equation*}
  かつ
  \begin{equation*}
    Au=Bu, \  \forall u\in\mathcal{D}(A)=\mathcal{D}(B)
  \end{equation*}
  が成立することであり,
  \begin{equation*}
    A=B
  \end{equation*}
  と表記する.
\end{dfn}

\begin{dfn}[作用素の連続]
  ノルム空間Xからノルム空間Yへの作用素Aが$u\in\mathcal{D}(A)$で連続であるとは
  \begin{equation*}
    u_n \rightarrow u, \ (n\rightarrow \infty)
  \end{equation*}
  となる任意の$u_n\in\mathcal{D}(A)\subset X$に対して
  \begin{equation*}
    Au_n\rightarrow Au, \ (n\rightarrow \infty)
  \end{equation*}
  を満たすときである.さらに,Aが任意の$u\in\mathcal{D}(A)$において連続であるとき,Aは連続であるという.
\end{dfn}

\begin{dfn}[線形作用素]
  線形空間Xから線形空間Yへの作用素Aが,任意の$u,v\in\mathcal{D}(A)\subset X$と$\alpha\in\mathbb{K}$に対し,
  \begin{align*}
     & \mathcal{D}(A)がXの線形部分空間 \\
     & A(u+v)=Au+Av                    \\
     & A(\alpha u)=\alpha Au
  \end{align*}
  を満たすとき,Aを作用素と呼ぶ.
\end{dfn}

\begin{dfn}[線形作用素の加法]
  \label{dfn:線形作用素の加法}
  線形空間Xから線形空間Yへの線形作用素AとBの和を
  \begin{equation*}
    (A+B)u := Au+Bu, \ u\in\mathcal{D}(A)\cup\mathcal{D}(B)
  \end{equation*}
  と定義する.このとき,XからYへの作用素$A+B$の定義域は
  \begin{equation*}
    \mathcal{D}(A+B) = \mathcal{D}(A)\cup\mathcal{D}(B)
  \end{equation*}
  とする.
\end{dfn}

\begin{dfn}[線形作用素のスカラー乗法]
  \label{dfn:線形作用素のスカラー乗法}
  線形空間Xから線形空間Yへの線形作用素Aの$\alpha\in\mathbb{K}$によるスカラー倍を
  \begin{equation*}
    (\alpha A)u := \alpha Au, \ u\in\mathcal{D}(A)
  \end{equation*}
  と定義する.このとき,XからYへの作用素$\alpha A$の定義域は
  \begin{equation*}
    \mathcal{D}(\alpha A):= \mathcal{D}(A)
  \end{equation*}
  とする.
\end{dfn}

\begin{dfn}[合成作用素]
  X,Y,Zを線形空間とする.AをYからZへの線形作用素とし,BをXからYへの線形作用素とする.そのとき,AとBの合成作用素ABは
  \begin{equation*}
    (AB)u:=A(Bu),\ u\in\{v\in\mathcal{D}(B)\mid Bv\in\mathcal{D}(A)\}
  \end{equation*}
  と定義する.このとき,XからZへの合成作用素ABの定義域は
  \begin{equation*}
    \mathcal{D}(AB):=\{v\in\mathcal{D}(B)\mid Bv\in\mathcal{D}(A)\}
  \end{equation*}
  とする.
\end{dfn}

\begin{thm}[線形作用素に対する単射性(1)]
  線形空間Xから線形空間Yへの線形作用素Aにおいて以下は同値である.
  \begin{enumerate}
    \item 線形作用素がAの単射である.
    \item $Au=0,\ u\in\mathcal{D}(A)\Rightarrow u=0$
  \end{enumerate}
\end{thm}

\begin{proof}
  単射の定義の対偶は
  \begin{equation*}
    Au_1 = Au_2,\ \forall u_1,u_2\in\mathcal{D}(A)\Rightarrow u_1=u_2
  \end{equation*}
  となる.その上,Aは線形作用素であるため,
  \begin{equation*}
    Au_1=Au_2\Leftrightarrow A(u_1-u_2)=0
  \end{equation*}
  となる.$u_1-u_2\in\mathcal{D}(A)$を$u\in\mathcal{D}(A)$とおきなおせば,$1\Rightarrow2$は証明された.また,証明を逆に追うことで$2\Rightarrow1$も示せる.
\end{proof}

\begin{thm}[線形作用素に対する単射性(2)]
  ノルム空間Xからノルム空間Yへの線形作用素Aとする.不等式
  \begin{equation*}
    \|u\|_X \leq K\|Au\|_Y,\ u\in\mathcal{D}(A)
  \end{equation*}
  を満たす定数$K>0$が存在するならば,線形作用素Aは単射である.
\end{thm}

\begin{proof}
  Aが線形作用素であるため,$Au=0,\ u\in\mathcal{D}(A)\Rightarrow u=0$を使って証明する.ノルムの定義より
  \begin{equation*}
    Au=0,\ \forall u\in\mathcal{D}(A)\Leftrightarrow\|Au\|_Y=0
  \end{equation*}
  となる.さらに,$Au=0$ならば,
  \begin{equation*}
    \|u\|_X \leq K\|Au\|_Y=0,\ u\in\mathcal{D}(A)
  \end{equation*}
  より$\|u\|_X=0$となる.よって,再びノルムの定義より
  \begin{equation*}
    \|u\|_X=0,\ \forall u\in\mathcal{D}(A)\Leftrightarrow u=0
  \end{equation*}
  より,$Au=0$ならば$u=0$となる.
\end{proof}

\begin{dfn}[有界な線形作用素]
  ノルム空間Xからノルム空間Yへの作用素Aに対し,
  \begin{equation*}
    \|Au\|_Y\leq K\|u\|_X,\ \mathcal{D}(A)
  \end{equation*}
  を満たす正の定数Kが存在する時,線形作用素Aを有界な作用素と呼ぶ.
\end{dfn}

\begin{thm}[有界な線形作用素と連続な線形作用素]
  ノルム空間Xからノルム空間Yへの作用素Aに対し,
  \begin{equation*}
    Aが有界\Leftrightarrow Aが連続
  \end{equation*}
\end{thm}

\begin{proof}
  「$Aが有界\Rightarrow Aが連続$」の証明

  連続性の定義より,$u_n\rightarrow u$となる任意の$u_n\in\mathcal{D}(A)$に対して$Au_n\rightarrow Au$となることを確かめる.$u_n\rightarrow u$となる任意の$u_n\in \mathcal{D}(A)$から$\|u_n-u\|_X\rightarrow 0,\ (n\rightarrow \infty)$を持つ.その上,Aは有界であることから
  \begin{equation*}
    \|Au_n-Au\|_Y\leq M\|u_n-u\|_X\rightarrow 0,\ (n\rightarrow\infty)
  \end{equation*}
  よって,$u_n\rightarrow u,\ (n\rightarrow \infty)$ならば,$Au_n\rightarrow Au$であるため,Aは連続である.

  \vskip\baselineskip

  「$Aが連続\Rightarrow Aが有界$」の証明

  背理法によって証明する.すなわち,$A$が連続ならば,任意の$M_2>0$に対して
  \begin{equation*}
    \|Au\|_Y>M_2\|u\|_x
  \end{equation*}
  を満たす$u\in\mathcal{D}$が存在すると仮定して矛盾を見つける.この仮定より自然数$n$に対して,
  \begin{equation*}
    \|Au_n\|_Y>n\|u_n\|_X
  \end{equation*}
  を満たす$u_n\in\mathcal{D}(A)$が存在する.このとき,$\|u_n\|_X\neq 0$であることに注意する.ノルム空間$X$はノルム空間全体の定義より線形空間であるため,ゼロ元$0\subset X$を持つ.その上,線形作用素の定義より$\mathcal{D}(A)$は$X$の部分空間であるため,ゼロ元$0\in\mathcal{D}(A)\subset X$を持つ.その上,$A$が連続であるため,$A$は$0\in\mathcal{D}(A)$でも連続である.$\epsilon-\sigma$論法による$A$の$0\in\mathcal{D}(A)\subset X$における連続の定義を記述すると
  \begin{equation*}
    \forall\epsilon>0,\ \exists\delta>0,\ \|u_n\|_X<\delta となる\forall u_n\in Xに対して\|Au_n\|_Y<\epsilon
  \end{equation*}
  となる.その上,$\epsilon$を$n\|u_n\|_X$とすると,$\delta_n>0$が存在し,$\|u_n\|_X<\delta$となる任意の$u_n\in\mathcal{D}(A)$に対して,
  \begin{equation*}
    \|Au_n\|_Y<n\|u_n\|_X
  \end{equation*}
  となる.有界ではないという仮定と組み合わせると
  \begin{equation*}
    n\|u_n\|_X<\|Au_n\|_Y<n\|u_n\|_X
  \end{equation*}
  となるため矛盾する.
\end{proof}

\begin{dfn}[定義域が$X$の全体となる有界な線形作用素全体の集合$\mathcal{L}(X,Y)$]
  定義域が\textrm{Banach}空間X全体となるXからYへの有界線形作用素全体を
  \begin{equation*}
    \mathcal{L}(X,Y)
  \end{equation*}
  とする.
\end{dfn}

\begin{thm}[$\mathcal{L}(X,Y)$はBanach空間]
  Xをノルム空間とし,Yを\textrm{Banach}空間とする.定義域がX全体となるXからYへの有界な線形作用素全体の集合$\mathcal{L}(X,Y)$のノルムを
  \begin{equation*}
    \|A\|_{\mathcal{L}(X,Y)}:=\sup_{u\in X\backslash\{0\}} \frac{\|Au\|_Y}{\|X\|_X}, \ A\in\mathcal{L}(X,Y)
  \end{equation*}
  とすると,$\mathcal{L}(X,Y)$はBanach空間となる.
\end{thm}

\begin{proof}
  作用素の加法(\ref{dfn:線形作用素の加法})と作用素のスカラー乗法(\ref{dfn:線形作用素のスカラー乗法})の定義をもとに線形空間の公理(\ref{dfn:線形空間の公理})が満たされていることが導かれる.ただし,$\mathcal{L}(X,Y)$のゼロ元は任意の$u\in X$を$0\in Y$へ写す作用素であることに注意が必要である.

  「ノルム空間」
  $\|A\|_{\mathcal{L}(X,Y)}$がノルムの定義を満たすことを示せばよい.ノルム空間$X$とBanach空間$Y$であるため$\|\cdot\|_X\geq 0$と$\|\cdot\|_Y\geq 0$であることから
  \begin{align*}
    \frac{\|Au\|_Y}{\|u\|_X} \geq 0
  \end{align*}
  となるため,$\|A\|_{\mathcal{L}(X,Y)}\geq 0$となり,ノルムの定義(\ref{dfn:線形空間の公理})はいえる.

  次に,$A=0$ならば$\|Au\|_Y=0$であるため,
  \begin{equation*}
    \|A\|_{\mathbb{B}(X,Y)}=\sup_{u\in X\backslash \{0\}} 
  \end{equation*}
\end{proof}

%% 定義 dfn
%% 定理 thm
%% 証明 proof

%\begin{thebibliography}{9}
%  \bibitem{ラベル} 著者名, ``題名'', 出版社, pp.0-1.
%  \bibitem{lit:name} 著者名, ``題名'',\linebreak \verb|http://www.kisarazu.ac.jp/|
%\end{thebibliography}
%%文章のあとに,\cite{label1, label2},\cite{lite:xxx}とつける
%隙間ができたら,\newlineを入れ込む
\end{document}