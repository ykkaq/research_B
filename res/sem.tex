%ctrl+alt+b -> build, ctrl+alt+v => preview at new tab
%ctrl+click -> highlight line

%\documentclass[titlepage]{jsarticle}
\documentclass[11pt,a4paper]{jsarticle}
%\documentclass[11pt,a4paper]{jsreport}

\usepackage{algorithmic}
\usepackage{amsmath,amssymb,amsfonts,amsthm,,mathtools}
\usepackage{ascmac}
\usepackage{bm}
\usepackage{caption}
\usepackage{cite}
\usepackage{comment}
\usepackage[dvipdfmx]{color}
\usepackage{colortbl}
\usepackage{float}
\usepackage[dvipdfmx]{graphicx}
\usepackage{multicol}
\usepackage{latexsym}
\usepackage{listings}
\usepackage[dvipdfmx]{pict2e}
\usepackage[ipaex]{pxchfon}
\usepackage{tabularx}
\usepackage{textcomp}
\usepackage{underscore}
\usepackage{ulem}
\usepackage{url}
\usepackage{wrapfig}
\usepackage{xcolor}

\captionsetup[figure]{labelsep=space}
\captionsetup[table]{labelsep=space}

\theoremstyle{definition}
\newtheorem{dfn}{定義}
\newtheorem{thm}{定理}

\renewcommand{\qedsymbol}{$\blacksquare$}
\renewcommand{\proofname}{\textbf{証明}}
%============

%\newtheo

%============
\begin{document}
\pagestyle{plain}
\title{タイトル}
\author{2131701 齋藤悠希}
\date{}\maketitle

\tableofcontents
\clearpage

\section{Preparation}
\subsection{Banach Space}
\begin{dfn}[線形空間の公理]
  \label{dfn:線形空間の公理}
  空でない集合$X$が,係数体$\mathbb{K}$上の線形空間であるとは,任意の$u+v \in X$とスカラー$\alpha \in \mathbb{K}$に対して,加法$u+v \in X$とスカラー乗法$\alpha u \in X$が定義されていて,任意の$u,v,w \in X$とスカラー$\alpha, \beta \in \mathbb{K}$に対して次のことが成り立つことである.
  \begin{enumerate}
    \item $(u+v)+w=u+(v+w)$
    \item $u+v=v+u$
    \item $u+0=u$となる$0 \in X$が一意に存在
    \item $u+(-u)=0$となる$-u\in X$が一意に存在
    \item $\alpha(u+v)=\alpha u+\alpha v$
    \item $(\alpha +\beta)u = \alpha u+\beta u$
    \item $(\alpha \beta)u = \alpha (\beta u)$
    \item $1u=u, 1 \in \mathbb{K}$
  \end{enumerate}
\end{dfn}

\begin{dfn}[ノルムとノルム空間の定義]
  $X$を係数体$\mathbb{K}$上の線形空間とする.$X$で定義された関数$||\cdot||:X\rightarrow \mathbb{K}$上で定義された関数が$X$のノルムであるとは
  \begin{enumerate}
    \item $||u||\geq 0, \quad u \in X$
    \item $||u||=0 \Leftrightarrow u=0$
    \item $||\alpha u||=|\alpha| ||u||, \quad (\alpha \in \mathbb{K}, u \in X)$
    \item $||u+v||\leq ||u||+||v||$
  \end{enumerate}
  が成立することである.さらに$X$に1つのノルムが指定されているとき,$X$はノルム空間という.
\end{dfn}

\begin{dfn}[ノルム空間の収束と極限]$X$をノルム空間とする.$X$の点列$(u_n)\subset X$は
  \begin{equation*}
    \forall\epsilon >0, \ \exists N\in \mathbb{N}, \space \forall N \geq Nに対して||u_n-u||<\epsilon
  \end{equation*}
  のとき,点$u\in X$に収束するといい,
  \begin{equation*}
    ||u_n-u||\rightarrow 0, \ \left(n\rightarrow\infty\right)
  \end{equation*}
  と表す.このとき,$u$を$u_n$の極限といい,
  \begin{equation*}
    u_n-u, \ (n\rightarrow \infty)
  \end{equation*}
  と表す.
\end{dfn}

\begin{dfn}[Cauchy列]
  \label{dfn:cauchy}
  $X$をノルム空間とする.そのとき$X$がCauchy列であるとは
  \begin{equation*}
    u_n-u_m\rightarrow 0, \ \left(n,m\rightarrow\infty \right)
  \end{equation*}
  が成立することである.即ち
  \begin{equation*}
    ||u_n-u_m||\rightarrow 0, \ \left(n,m\rightarrow \infty\right)
  \end{equation*}
  が成立することである.
\end{dfn}

\begin{dfn}[完備]
  $X$をノルム空間とする.$X$が完備であるとは,任意のCauchy列$(u_n)$が$X$の中で極限をもつことである.すなわち,任意のCauchy列$(u_n\subset X)$が
  \begin{equation*}
    \|u_n-u\|\rightarrow 0\ ,\left(n\rightarrow 0\right)
  \end{equation*}
  となる極限$u$を$X$内に持つことである.
\end{dfn}

\begin{dfn}[Banach空間]
  ノルム空間$X$がBanach空間であるとは,$X$が完備であることである.
\end{dfn}

\begin{thm}[逆三角不等式]
  \label{thm1}
  $X$をノルム空間とする.任意の$u,v\in X$について次の不等式を満たす.
  \begin{equation*}
    |\|u\|-\|v\||\leq\|u-v\|
  \end{equation*}
\end{thm}

\begin{proof}
  任意の$u,v\in X$について
  \begin{align*}
    \|u\| & =\|u-v+v\|\leq\|u-v\|+\|v\|               \\
    \|v\| & =\|v-u+u\|\leq\|v-u\|+\|u\|=\|u-v\|+\|u\|
  \end{align*}
  となる.よって
  \begin{equation*}
    \|u\|-\|v\|\leq\|u-v\| \\
    \|v\|-\|u\|\leq\|u-v\|
  \end{equation*}
  となるため,
  \begin{equation*}
    |\|u\|-\|v\||\leq\|u-v\|
  \end{equation*}
  を持つ.
\end{proof}

\begin{dfn}[有界列]
  $X$をノルム空間とする.そのとき$X$の点列$(u_n)$が有界列とは任意の$n\in\mathbb{N}$に対して
  \begin{equation*}
    \|u_n\|\leq M
  \end{equation*}
  となる定数$M>0$が存在することである.
\end{dfn}

\begin{thm}[Cauchy列ならば有界列]
  $X$をノルム空間とする.そのとき$X$の点列$(u_n)$がCauchy列ならば有界列でもある.
\end{thm}

\begin{proof}
  $X$の点列$(u_n)$がCauchy列であるために,$\epsilon -N$論法を用いた表記で
  \begin{equation*}
    \forall\epsilon >0, \ \exists N\in \mathbb{N}, \ \forall n,m \geq Nに対して||u_n-u_m||<\epsilon
  \end{equation*}
  を満たす.$\epsilon=1$としても,それに対応した$N$が存在し,任意の$n\geq N$に対して
  \begin{equation*}
    \|u_n-u_N\|<1
  \end{equation*}
  を満たす.

  任意の$n\geq N$に対して$\|u_n\|$が$\|u_N\|$で評価できることを示す.逆三角不等式である定理\ref{thm1}を用いると
  \begin{equation*}
    |\|u_n\|-\|u_N\|\leq \|u_n-u_N\|<1
  \end{equation*}
  となる.絶対値の性質より$|\|u_n-u_N\||<1$は
  \begin{equation*}
    \|u_N\|-1\leq\|u_n\|<\|u_N\|+1
  \end{equation*}
  となる.よって
  \begin{equation*}
    M=\max\{\|u_1\|,\|u_2\|,\cdots,\|u_{N-1}\|,\|u_N\|+1\}
  \end{equation*}
  とすると,任意の$n\in \mathbb{N}について$
  \begin{equation*}
    \|u_n\|\leq M
  \end{equation*}
  が成り立つため,点列$(u_N)$は有界列である.
\end{proof}

\begin{dfn}[線形部分空間]
  線形空間$X$の空でない集合$M$が任意の元$u,v\in M$と任意の係数体$\alpha\in\mathbb{K}$に対して
  \begin{align*}
    u+v\in M \\
    \alpha u \in M
  \end{align*}
  を満たすとき,$M$は線形空間$X$の線形部分空間と呼ぶ.
\end{dfn}

\begin{dfn}[ノルム空間の開球]
  $X$をノルム空間とする.$x\in X$とし,$r>0$を正実数とする.そのとき,集合
  \begin{equation*}
    B_X(x,r):=\{y\in X \mid \|x-y\|_X<r\}
  \end{equation*}
  を中心$x$,半径$r$の開球という.$X$が明らかな場合は$B_X(x,r)$を省略して$B(x,r)$と表記する.
\end{dfn}

\begin{dfn}[ノルム空間の開集合]
  $X$をノルム空間とし,$M$を$X$の部分集合とする.任意の$x\in M$に対して,$B_X(x,r)\subset M$となる$r>0$が存在する場合,$M$が開集合であるという.
\end{dfn}

\begin{dfn}[ノルム空間の閉集合]
  Xをノルム空間とし,MをXの部分集合とする.Mが閉集合であるとは,Mの任意の点列$(u_n)$の極限$u\in X$がMにも属することである.すなわち,点列$(u_n)\subset M$について
  \begin{equation*}
    u_n\rightarrow u, \quad \left(n\rightarrow \infty\right) \Rightarrow u\in M
  \end{equation*}
  であるとき,Mは閉集合であるという.
\end{dfn}

\begin{dfn}[閉部分空間]
  Xをノルム空間とし,MをXの線形部分空間が閉集合であるとき,Mを閉部分空間である.
\end{dfn}

\subsection{Operator}
\begin{dfn}[作用素]
  ある線形空間Xから別の線形空間Yへの作用素Aとは,
  \begin{equation*}
    \mathcal{D}(A) := \{u \in X \mid Au \in Y\}
  \end{equation*}
  としたとき,$\mathcal{D}(A)$のどんな元に対しても,それぞれ集合Yのただ一つの元を指定する規則のことである.また,$\mathcal{D}(A)$はAの定義域と呼ばれ
  \begin{equation*}
    \mathcal{R}(A) := \{Au \in Y \mid u\in\mathcal{D}(A)\}
  \end{equation*}
  を値域と呼ぶ
\end{dfn}

\begin{dfn}[単射]
  \label{injection}
  線形空間Xから線形空間Yへの作用素Aが
  \begin{equation*}
    u_1 \neq u_2, \quad \forall u_1,u_2 \in \mathcal{D}(A) \Rightarrow A(u_1)\neq A(u_2)
  \end{equation*}
\end{dfn}

\begin{dfn}[全射]
  線形空間Xから線形空間Yへの作用素Aが
  \begin{equation*}
    Y=\mathcal{R}(A)
  \end{equation*}
  を満たすときに作用素Aは全単射であるという.
\end{dfn}

\begin{dfn}[全射]
  線形空間Xから線形空間Yへの作用素Aとし,その定義域を$\mathcal{D}(A)\subset X$,値域を$\mathcal{R}(A)\subset Y$とする.そのとき,
  \begin{align*}
    A^{-1}\left( A\left( u \right) \right) & =u, \  u\in\mathcal{D}\left( A \right) \\
    A(A^{-1}(u))                           & =u, \  u\in\mathcal{R}(A)
  \end{align*}
  かつ
  \begin{align*}
    \mathcal{D}(A^{-1}) & =\mathcal{R}(A) \\
    \mathcal{R}(A^{-1}) & =\mathcal{D}(A)
  \end{align*}
  となるYからXへの作用素$A^{-1}$をAの逆作用素と呼ぶ.
\end{dfn}

\begin{thm}[単射と逆作用素の環境]
  線形空間Xから線形空間Yへの作用素Aとすると.
  \begin{equation*}
    Aが逆作用素を持つ \Leftrightarrow Aが単射である
  \end{equation*}
\end{thm}

\begin{proof}
  「$Aが逆作用素を持つ \Rightarrow Aが単射である$」の証明

  単射の定義\ref{injection}の待遇「$任意のu_1,u_2\in\mathcal{D}(A)に対しA(u_1)=A(u_2) \Rightarrow u_1=u_2$」を満たすことを確かめる.Aの逆作用素を$A^{-1}$とすると,任意の$u_1,u_2\in\mathcal{D}(A)$に対し
  \begin{align*}
                & A(u_1) = A(u_2)               \\
    \Rightarrow & A^{-1}(A(u_1))=A^{-1}(A(u_2)) \\
    \Rightarrow & u_1=u_2
  \end{align*}

  $「Aが単射である\Rightarrow Aが逆作用素A^{-1}をもつ」の証明$

  $A$の値域の定義$\mathcal{R}(A)=\{A(u)\in Y \mid u\in\mathcal{D}(A)\}$より,任意の$v\in\mathcal{R}(A)$に対し,
  \begin{equation*}
    A(u)=v
  \end{equation*}
  となる$u\in\mathcal{D}(A)$が存在する.その上,Aが単射であるため,単射の定義の対偶より$u\in\mathcal{D}(A)$はどんな$u\in\mathcal{R}(A)$に対してもただ一つの元である.そのため,作用素の定義より,上記の$u\in\mathcal{R}(A)$に対してただ一つの元$u\in\mathcal{D}(A)$を指定する規則として
  \begin{equation*}
    B(v)=u
  \end{equation*}
  となる定義域$\mathcal{D}(B)=\mathcal{R}(A)$と値域$\mathcal{R}(B)=\mathcal{D}(A)$となるYからXへの作用素Bが定義できる.その上,$B(v)=u$の$v=A(u)$を代入すると
  \begin{equation*}
    B(A(u))=u
  \end{equation*}
  となる.同様に,$A(u)=v$の$u$に$u=B(v)$を代入すると
  \begin{equation*}
    A(B(v))=v
  \end{equation*}
  となる.よって,定義域$\mathcal{D}(B)=\mathcal{R}(A)$と値域$\mathcal{R}(B)=\mathcal{D}(A)$となるYからXへの作用素BはAの逆作用素であるため,Aは逆作用素を持つ.
\end{proof}

\begin{dfn}[作用素の等号]
  線形空間Xから線形空間Yへの作用素AとBが等しいとは
  \begin{equation*}
    \mathcal{D}(A) = \mathcal{D}(B)
  \end{equation*}
  かつ
  \begin{equation*}
    Au=Bu, \  \forall u\in\mathcal{D}(A)=\mathcal{D}(B)
  \end{equation*}
  が成立することであり,
  \begin{equation*}
    A=B
  \end{equation*}
  と表記する.
\end{dfn}

\begin{dfn}[作用素の連続]
  ノルム空間Xからノルム空間Yへの作用素Aが$u\in\mathcal{D}(A)$で連続であるとは
  \begin{equation*}
    u_n \rightarrow u, \ (n\rightarrow \infty)
  \end{equation*}
  となる任意の$u_n\in\mathcal{D}(A)\subset X$に対して
  \begin{equation*}
    Au_n\rightarrow Au, \ (n\rightarrow \infty)
  \end{equation*}
  を満たすときである.さらに,Aが任意の$u\in\mathcal{D}(A)$において連続であるとき,Aは連続であるという.
\end{dfn}

\begin{dfn}[線形作用素]
  線形空間Xから線形空間Yへの作用素Aが,任意の$u,v\in\mathcal{D}(A)\subset X$と$\alpha\in\mathbb{K}$に対し,
  \begin{align*}
     & \mathcal{D}(A)がXの線形部分空間 \\
     & A(u+v)=Au+Av                    \\
     & A(\alpha u)=\alpha Au
  \end{align*}
  を満たすとき,Aを作用素と呼ぶ.
\end{dfn}

\begin{dfn}[線形作用素の加法]
  \label{dfn:線形作用素の加法}
  線形空間Xから線形空間Yへの線形作用素AとBの和を
  \begin{equation*}
    (A+B)u := Au+Bu, \ u\in\mathcal{D}(A)\cup\mathcal{D}(B)
  \end{equation*}
  と定義する.このとき,XからYへの作用素$A+B$の定義域は
  \begin{equation*}
    \mathcal{D}(A+B) = \mathcal{D}(A)\cup\mathcal{D}(B)
  \end{equation*}
  とする.
\end{dfn}

\begin{dfn}[線形作用素のスカラー乗法]
  \label{dfn:線形作用素のスカラー乗法}
  線形空間Xから線形空間Yへの線形作用素Aの$\alpha\in\mathbb{K}$によるスカラー倍を
  \begin{equation*}
    (\alpha A)u := \alpha Au, \ u\in\mathcal{D}(A)
  \end{equation*}
  と定義する.このとき,XからYへの作用素$\alpha A$の定義域は
  \begin{equation*}
    \mathcal{D}(\alpha A):= \mathcal{D}(A)
  \end{equation*}
  とする.
\end{dfn}

\begin{dfn}[合成作用素]
  \label{dfn:合成作用素}
  X,Y,Zを線形空間とする.AをYからZへの線形作用素とし,BをXからYへの線形作用素とする.そのとき,AとBの合成作用素ABは
  \begin{equation*}
    (AB)u:=A(Bu),\ u\in\{v\in\mathcal{D}(B)\mid Bv\in\mathcal{D}(A)\}
  \end{equation*}
  と定義する.このとき,XからZへの合成作用素ABの定義域は
  \begin{equation*}
    \mathcal{D}(AB):=\{v\in\mathcal{D}(B)\mid Bv\in\mathcal{D}(A)\}
  \end{equation*}
  とする.
\end{dfn}

\begin{thm}[線形作用素に対する単射性(1)]
  \label{thm:線形作用素に対する単射性(1)}
  線形空間Xから線形空間Yへの線形作用素Aにおいて以下は同値である.
  \begin{enumerate}
    \item 線形作用素がAの単射である.
    \item $Au=0,\ u\in\mathcal{D}(A)\Rightarrow u=0$
  \end{enumerate}
\end{thm}

\begin{proof}
  単射の定義の対偶は
  \begin{equation*}
    Au_1 = Au_2,\ \forall u_1,u_2\in\mathcal{D}(A)\Rightarrow u_1=u_2
  \end{equation*}
  となる.その上,Aは線形作用素であるため,
  \begin{equation*}
    Au_1=Au_2\Leftrightarrow A(u_1-u_2)=0
  \end{equation*}
  となる.$u_1-u_2\in\mathcal{D}(A)$を$u\in\mathcal{D}(A)$とおきなおせば,$1\Rightarrow2$は証明された.また,証明を逆に追うことで$2\Rightarrow1$も示せる.
\end{proof}

\begin{thm}[線形作用素に対する単射性(2)]
  ノルム空間Xからノルム空間Yへの線形作用素Aとする.不等式
  \begin{equation*}
    \|u\|_X \leq K\|Au\|_Y,\ u\in\mathcal{D}(A)
  \end{equation*}
  を満たす定数$K>0$が存在するならば,線形作用素Aは単射である.
\end{thm}

\begin{proof}
  Aが線形作用素であるため,$Au=0,\ u\in\mathcal{D}(A)\Rightarrow u=0$を使って証明する.ノルムの定義より
  \begin{equation*}
    Au=0,\ \forall u\in\mathcal{D}(A)\Leftrightarrow\|Au\|_Y=0
  \end{equation*}
  となる.さらに,$Au=0$ならば,
  \begin{equation*}
    \|u\|_X \leq K\|Au\|_Y=0,\ u\in\mathcal{D}(A)
  \end{equation*}
  より$\|u\|_X=0$となる.よって,再びノルムの定義より
  \begin{equation*}
    \|u\|_X=0,\ \forall u\in\mathcal{D}(A)\Leftrightarrow u=0
  \end{equation*}
  より,$Au=0$ならば$u=0$となる.
\end{proof}

\begin{dfn}[有界な線形作用素]
  ノルム空間Xからノルム空間Yへの作用素Aに対し,
  \begin{equation*}
    \|Au\|_Y\leq K\|u\|_X,\ \mathcal{D}(A)
  \end{equation*}
  を満たす正の定数Kが存在する時,線形作用素Aを有界な作用素と呼ぶ.
\end{dfn}

\begin{thm}[有界な線形作用素と連続な線形作用素]
  ノルム空間Xからノルム空間Yへの作用素Aに対し,
  \begin{equation*}
    Aが有界\Leftrightarrow Aが連続
  \end{equation*}
\end{thm}

\begin{proof}
  「$Aが有界\Rightarrow Aが連続$」の証明

  連続性の定義より,$u_n\rightarrow u$となる任意の$u_n\in\mathcal{D}(A)$に対して$Au_n\rightarrow Au$となることを確かめる.$u_n\rightarrow u$となる任意の$u_n\in \mathcal{D}(A)$から$\|u_n-u\|_X\rightarrow 0,\ (n\rightarrow \infty)$を持つ.その上,Aは有界であることから
  \begin{equation*}
    \|Au_n-Au\|_Y\leq M\|u_n-u\|_X\rightarrow 0,\ (n\rightarrow\infty)
  \end{equation*}
  よって,$u_n\rightarrow u,\ (n\rightarrow \infty)$ならば,$Au_n\rightarrow Au$であるため,Aは連続である.

  \vskip\baselineskip

  「$Aが連続\Rightarrow Aが有界$」の証明

  背理法によって証明する.すなわち,$A$が連続ならば,任意の$M_2>0$に対して
  \begin{equation*}
    \|Au\|_Y>M_2\|u\|_x
  \end{equation*}
  を満たす$u\in\mathcal{D}$が存在すると仮定して矛盾を見つける.この仮定より自然数$n$に対して,
  \begin{equation*}
    \|Au_n\|_Y>n\|u_n\|_X
  \end{equation*}
  を満たす$u_n\in\mathcal{D}(A)$が存在する.このとき,$\|u_n\|_X\neq 0$であることに注意する.ノルム空間$X$はノルム空間全体の定義より線形空間であるため,ゼロ元$0\subset X$を持つ.その上,線形作用素の定義より$\mathcal{D}(A)$は$X$の部分空間であるため,ゼロ元$0\in\mathcal{D}(A)\subset X$を持つ.その上,$A$が連続であるため,$A$は$0\in\mathcal{D}(A)$でも連続である.$\epsilon-\sigma$論法による$A$の$0\in\mathcal{D}(A)\subset X$における連続の定義を記述すると
  \begin{equation*}
    \forall\epsilon>0,\ \exists\delta>0,\ \|u_n\|_X<\delta となる\forall u_n\in Xに対して\|Au_n\|_Y<\epsilon
  \end{equation*}
  となる.その上,$\epsilon$を$n\|u_n\|_X$とすると,$\delta_n>0$が存在し,$\|u_n\|_X<\delta$となる任意の$u_n\in\mathcal{D}(A)$に対して,
  \begin{equation*}
    \|Au_n\|_Y<n\|u_n\|_X
  \end{equation*}
  となる.有界ではないという仮定と組み合わせると
  \begin{equation*}
    n\|u_n\|_X<\|Au_n\|_Y<n\|u_n\|_X
  \end{equation*}
  となるため矛盾する.
\end{proof}

\begin{dfn}[定義域が$X$の全体となる有界な線形作用素全体の集合$\mathcal{L}(X,Y)$]
  定義域がBanach空間X全体となるXからYへの有界線形作用素全体を
  \begin{equation*}
    \mathcal{L}(X,Y)
  \end{equation*}
  とする.
\end{dfn}

\begin{thm}[$\mathcal{L}(X,Y)$はBanach空間]
  Xをノルム空間とし,YをBanach空間とする.定義域がX全体となるXからYへの有界な線形作用素全体の集合$\mathcal{L}(X,Y)$のノルムを
  \begin{equation*}
    \|A\|_{\mathcal{L}(X,Y)}:=\sup_{u\in X\backslash\{0\}} \frac{\|Au\|_Y}{\|X\|_X}, \ A\in\mathcal{L}(X,Y)
  \end{equation*}
  とすると,$\mathcal{L}(X,Y)$はBanach空間となる.
\end{thm}

\begin{proof}
  作用素の加法(\ref{dfn:線形作用素の加法})と作用素のスカラー乗法(\ref{dfn:線形作用素のスカラー乗法})の定義をもとに線形空間の公理(\ref{dfn:線形空間の公理})が満たされていることが導かれる.ただし,$\mathcal{L}(X,Y)$のゼロ元は任意の$u\in X$を$0\in Y$へ写す作用素であることに注意が必要である.

  「ノルム空間」
  $\|A\|_{\mathcal{L}(X,Y)}$がノルムの定義を満たすことを示せばよい.ノルム空間$X$とBanach空間$Y$であるため$\|\cdot\|_X\geq 0$と$\|\cdot\|_Y\geq 0$であることから
  \begin{align*}
    \frac{\|Au\|_Y}{\|u\|_X} \geq 0
  \end{align*}
  となるため,$\|A\|_{\mathcal{L}(X,Y)}\geq 0$となり,ノルムの定義(\ref{dfn:線形空間の公理})はいえる.

  次に,$A=0$ならば$\|Au\|_Y=0$であるため,
  \begin{equation*}
    \|A\|_{\mathcal{L}(X,Y)}=\sup_{u\in X\backslash \{0\}} \frac{\|Au\|_Y}{\|u\|_X} = \sup_{u\in X\backslash \{0\}} \frac{0}{\|u\|_X} = 0
  \end{equation*}
  である.さらに,任意の$u\in X\backslash\{0\}$について
  \begin{equation*}
    \frac{\|Au\|_Y}{\|u\|_X} = 0 \Leftrightarrow \|Au\|_Y=0 \Leftrightarrow Au=0
  \end{equation*}
  任意の$u\in X\backslash \{0\}$を$0\in Y$へ写す作用素は$\mathcal{L}(X,Y)$が線形空間により一意に存在し,$A=0$である.よって,ノルムの定義(2)も示された.

  続いて$\alpha\in K$としたとき,$Y$はBanach空間であるため,$\|\cdot\|_Y$はノルムの定義を満たすため,
  \begin{equation*}
    \|\alpha A\|_{\mathcal{L}} = \sup_{u\in X\backslash \{0\}} \frac{\|\alpha Au\|_Y}{\|u\|_X} = |\alpha|\sup_{u\in X\backslash \{0\}} \frac{\| Au\|_Y}{\|u\|_X} = |\alpha|\|A\|_{\mathcal{L}(X,Y)}
  \end{equation*}
  となるため,ノルムの定義(3)も示された.

  最後に,任意の$A,B\in \mathcal{L}(X,Y)$について
  \begin{align*}
    \|A+B\|_{\mathcal{L}(X,Y)} & = \sup_{u\in X\backslash \{0\}} \frac{\|(A+B)u\|_Y}{\|u\|_X}                                                    \\
                               & = \sup_{u\in X\backslash \{0\}} \frac{\|Au+Bu\|_Y}{\|u\|_X}                                                     \\
                               & = \sup_{u\in X\backslash \{0\}} \frac{\|Au\|_Y + \|Bu\|_Y}{|u\|_X}                                              \\
                               & = \sup_{u\in X\backslash \{0\}} \frac{\|Au\|_Y}{|u\|_X} + \sup_{u\in X\backslash \{0\}} \frac{\|Bu\|_Y}{|u\|_X} \\
                               & = \|A\|_{\mathcal{L}(X,Y)} + \|B\|_{\mathcal{L}(X,Y)}
  \end{align*}
  となり,ノルムの定義(4)も示されたため,$\mathcal{L}(X,Y)$はノルム空間である.

  「Banach空間」

  Banach空間であることを証明するには$\mathcal{L}(X,Y)$の任意のCauchy列$(A_n)\subset \mathcal{L}(X,Y)$が極限$T$を$\mathcal{L}(X,Y)$内に持つことを示せばよい.

  まず,極限の候補$\tilde{A}$が定義できるか確認する.任意のCauchy列$(A_n)\subset \mathcal{L}(X,Y)$はCauchy列の定義(\ref{dfn:cauchy})より
  \begin{equation*}
    \|A_n-A_m\|_{\mathcal{L}(X,Y)} \rightarrow 0,\ (n,m\rightarrow \infty)
  \end{equation*}
  となる.任意の$u\in X\backslash \{0\}$に対して,点列$(A_nu) \subset Y$は
  \begin{align*}
    \|A_n u-A_mu\|_Y & = \frac{\|(A_n-A_m)u\|_Y}{\|u\|_X}\|u\|_X                                         \\
                     & \leq \sup_{\phi\in X\backslash\{0\}} \frac{\|(A_n-A_m)\phi\|_Y}{\|\phi\|_X}\|\|_X \\
                     & = \|A_n-A_m\|_{\mathcal{L}(X,Y)}\|u\|_X \rightarrow 0,\ (n,m \rightarrow \infty)
  \end{align*}
  を持つため,点列$(A_nu)\subset Y$はCauchy列になる.その上,$Y$はBanach空間であるため,$Y$の任意のCauchy列は収束し,$Y$内に極限$\tilde{A}u$となるような$X$から$Y$への作用素$\tilde{A}$が存在する.ここで,任意の$u\in X$に対して極限$\tilde{A}u$が定義されることから,$\tilde{A}$の定義域は$\mathcal{D}(\tilde{A})=X$である.これにより,$\mathcal{L}(X,Y)$の任意のCauchy列$(A_n)$の極限の候補$\tilde{A}$が定義できた.

  続いて,定義した極限の候補$\tilde{A}$が$\mathcal{L}(X,Y)$に属しているか確認する.$\tilde{A}$が有界な線形作用素であり,かつ$D(\tilde{A})=X$であることを示せばよい.$\mathcal{L}(X,Y)$の任意のCauchy列$(A_n)$の元$A_n$は線形作用素であるため,線形作用素の定義より任意の$\alpha,\beta \in \mathbb{K}$と$u,v\in X$について

  \begin{equation*}
    A_n(\alpha u+\beta v) = \alpha A_n u + \beta A_n v
  \end{equation*}
  を持つ.よって$n\rightarrow \infty$とすると
  \begin{equation*}
    (\alpha u+\beta v)=\alpha \tilde{A} u + \beta \tilde{A} v
  \end{equation*}
  となり,極限の候補$\tilde{A}$は線形作用素である.次に極限の候補$\tilde{A}$が有界作用素であることを示す.点列$(A_n)$はCauchy列であるため定理(2)より有界列でもある.すなわち,どんな$n\in \mathbb{N}$に対しても
  \begin{equation*}
    \|A_n\|_{\mathcal{L}(X,Y)} \leq M
  \end{equation*}
  となる$n\in\mathbb{N}$に依存しない定数$M$が存在する.この$n\in\mathbb{N}$に依存しない定数$M$は,任意の$n\in X$について
  \begin{equation*}
    \|A_nu\|_Y \leq M\|u\|_X
  \end{equation*}
  も満たす.$\|A_nu\|\rightarrow\tilde{A}_u,\ (n\rightarrow \infty)$であるため,上の不等式に対して$n\rightarrow \infty$とすると$M$が$n$に依存しないため
  \begin{equation*}
    \|\tilde{A}u\|_Y \leq M\|u\|_X
  \end{equation*}
  を得る.よって,点列$(A_n)$の極限の候補$\tilde{A}$は$\mathcal{L}(X,Y)$に属する.

  最後に,点列$(A_n)$の極限が$\tilde{A}$であることを示す.任意の$u\in X$に対して,点列$(A_nu)\subset Y$は$Y$内に極限$\tilde{A}u$を持つこと,すなわち
  \begin{equation*}
    A_nu\rightarrow \tilde{A}u,\ (u\rightarrow \infty)
  \end{equation*}
  を持つことから
  \begin{equation*}
    \|A_n u- A_m u\|_Y\rightarrow \|A_n u-\tilde{A}u\|_Y,\ (m\rightarrow \infty)
  \end{equation*}
  となる.その上,$\mathcal{L}(X,Y)$のノルムの定義と$\tilde{A}\in\mathcal{L}(X,Y)$から
  \begin{equation*}
    \|A_n-A_m\|_{\mathcal{L}(X,Y)}\rightarrow\|A_n-\tilde{A}\|_{\mathcal{L}(X,Y)},\ (m\rightarrow \infty)
  \end{equation*}
  を得る.点列$(A_n)$がCauchy列であるため
  \begin{equation*}
    \forall \epsilon >0,\ \exists N\in\mathbb{N},\ \forall n,m\geq N に対して\|A_n-A_m\|_{\mathcal{L}(X,Y)}<\epsilon
  \end{equation*}
  を満たす.その上,$m\rightarrow \infty$とすると
  \begin{equation*}
    \forall \epsilon >0,\ \exists N\in\mathbb{N},\ \forall n\geq N に対して\|A_n-\tilde{A}\|_{\mathcal{L}(X,Y)}<\epsilon
  \end{equation*}
  となり,$\tilde{A}\in\mathcal{L}(X,Y)$はCauchy列$(A_n)$の極限である.よって,任意のCauchy列は$\mathcal{L}(X,Y)$内に極限を持つため,ノルム空間$\mathcal{L}(X,Y)$はBanach空間である.
\end{proof}

\begin{thm}[$\mathcal{L}(X,Y)$ノルムの性質(1)]
  $X$をノルム空間とし,$Y$をBanach空間とする.そのとき,任意の$u\in X$と任意の$A\in\mathcal{L}(X,Y)$について以下の不等式が成り立つ.
  \begin{equation*}
    \|Au\|_Y\geq\|A\|_{\mathcal{L}(X,Y)}\|u\|_X
  \end{equation*}
\end{thm}

\begin{proof}
  $u=0$の時は明らかに成り立つため,$u\in X\backslash\{0\}$について考える.$u\in X\backslash\{0\}$について
  \begin{equation*}
    \|Au\|_Y = \frac{\|Au\|_Y}{\|u\|_X}\|u\|_X \leq \sup_{\phi\in X\backslash \{0\}} \frac{\|A\phi\|_Y}{\|\phi\|_X}\|u\|_X = \|A\|_{\mathcal{L}(X,Y)}\|u\|_X
  \end{equation*}
  となるため,題意は示された.
\end{proof}

\begin{thm}[$\mathcal{L}(X,Y)$ノルムの性質(2)]
  \label{thm:ノルムの性質2}
  $X$をノルム空間とし,$Y$と$Z$をBanach空間とする.そのとき,任意の$B\in\mathcal{L}(X,Y)$と$A\in\mathcal{L}(X,Y)$の合成作用素$AB$は$\mathcal{L}(X,Y)$に属する.その上,
  \begin{equation*}
    \|AB\|_{\mathcal{L}(X,Z)} \leq \|A\|_{\mathcal{L}(Y,Z)}\|B\|_{\mathcal{L}(X,Y)}
  \end{equation*}
\end{thm}

\begin{proof}
  合成作用素の定義(\ref{dfn:合成作用素})から
  \begin{equation*}
    \mathcal{D}(AB)=\{v\in\mathcal{D}(B)=X \mid Bv\in\mathcal{D}(A)=Y\}
  \end{equation*}
  となるが,$B\in\mathcal{L}(X,Y)$であるため,任意の$v\in X$に対して$Bv$は$Y$に属する.よって,
  \begin{equation*}
    \mathcal{D}(AB)=\mathcal{D}=X
  \end{equation*}
  となる.その上,$A$も$B$も線形作用素であることから,任意の$u,v\in X$と任意の$\alpha,\beta\in\mathbb{K}$に対して
  \begin{equation*}
    AB(\alpha u+\beta v) = A(B\alpha u+B\beta v) = A(\alpha Bu+\beta Bv) = A\alpha Bu + A\beta Bv = \alpha ABu + \beta ABv
  \end{equation*}
  となるため,合成作用素$AB$は定義域が$X$全体となる線形作用素である.よって$AB$は$\mathcal{L}(X,Y)$に属する.その上,
  \begin{align*}
    \|AB\|_{\mathcal{L}(X,Z)} & = \sup_{u\in X\backslash \{0\}} \frac{\|ABu\|_Z}{\|u\|_X}                                                            \\
                              & \leq \sup_{u\in X\backslash \{0\}} \frac{\|A\|_{\mathcal{L}(Y,Z)} \|B\|_{\mathcal{L}(X,Y)} \|u\|_X }{\|u\|_X}\|u\|_X \\
                              & = \|A\|_{\mathcal{L}(Y,Z)} \|B\|_{\mathcal{L}(X,Y)}
  \end{align*}
\end{proof}

\begin{dfn}[X上の恒等作用素]
  $X$をBanach空間とする.任意の$u\in X$に対して
  \begin{equation*}
    Iu=u
  \end{equation*}
  となる$I\in\mathcal{L}(X)$をX上の恒等作用素と呼ぶ.
\end{dfn}

\begin{thm}[Neumann級数]
  $X$をBanach空間とする.$B\in\mathcal{L}(X)$とし,$I\in\mathcal{L}(X)$を$X$上の恒等作用素とする.もし
  \begin{equation*}
    \|I-B\|_{\mathcal{L}(X)}<1
  \end{equation*}
  ならば逆作用素をもち$B^{-1}\in\mathcal{L}(X)$となる.そのうえ,
  \begin{equation*}
    B^{-1} = I + (I-B) + (I-B)^2 + \cdots = \sum^\infty_{i=0}(I-B)^i
  \end{equation*}
  で,かつ
  \begin{equation*}
    \|B^{-1}\|_{\mathcal{L}(X)} \leq \frac{1}{1-\|I-B\|_{\mathcal{L}(X)}}
  \end{equation*}
\end{thm}

\begin{proof}
  \begin{equation*}
    S_n = I + (I-B) + (I-B)^2 + \cdots + (I-B)^n
  \end{equation*}
  とすると,$B$と$I$ともに$\mathcal{L}(X)$に属するため,加法$I-B$や合成作用素$(I-B)(I-B)$なども$\mathcal{L}(X)$に属する.よって$S_n$も$\mathcal{L}(X)$に属する.

  続いて点列$(S_n)\subset\mathcal{L}(X)$が極限$S$を$\mathcal{L}(X)$内に持つか確認する.定理(\ref{thm:ノルムの性質2})より
  \begin{equation*}
    \|(I-B)^i\|_{\mathcal{L}(X)} \leq \|(I-B)\|_{\mathcal{L}(X)}^i,\ i=0,1,\cdots
  \end{equation*}
  となるため,$n>m>0$となる整数に対して
  \begin{equation*}
    \|S_n-S_m\|_{\mathcal{L}(X)} = \left \| \sum^n_{i=m+1}(I-B)^i \right \| _{\mathcal{L}(X)} \leq \sum^n_{i=m+1}\|(I-B)\|^i_{\mathcal{L}(X)}
  \end{equation*}
  となる.定理の仮定より$\|(I-B)\|_{\mathcal{L}(X)}<1$であるため,
  \begin{equation*}
    \sum^n_{i=m+1}\|(I-B)\|^i_{\mathcal{L}(X)} \rightarrow 0,\ (n,m\rightarrow \infty)
  \end{equation*}
  となる.よって
  \begin{equation*}
    \|S_n-S_m\|_{\mathcal{L}(X)} \rightarrow 0,\ (n,m\rightarrow \infty)
  \end{equation*}
  となるため,点列$(S_n)$はCauchy列である.その上,$\mathcal{L}(X)$はBanach空間であるため,任意のCauchy列は極限$\mathcal{L}(X)$に持つため,点列$(S_n)$は
  \begin{equation*}
    \|S_n-S\|_{\mathcal{L}(X)} \rightarrow 0,\ (n,m\rightarrow \infty)
  \end{equation*}
  となる極限$S\in\mathcal{L}(X,Y)$を持つ.

  次に$S$が$B^{-1}$になることを示す.合成作用素の定義(\ref{dfn:合成作用素})にしたがって合成作用素$BS_n$を考える.$X$はBanach空間であり,$S,B_n\in\mathcal{L}(X)$であるため,定理(\ref{thm:ノルムの性質2})より合成作用素$BS_n$は$\mathcal{L}(X)$に属する.その上,点列$(BS_n)\subset\mathcal{L}(X)$は
  \begin{equation*}
    \|BS_n-BS\|_{\mathcal{L}(X)} \leq \|B\|_{\mathcal{L}(X)}\|S_n-S\|_{\mathcal{L}(X)}\rightarrow 0,\ (n\rightarrow \infty)
  \end{equation*}
  となるため,極限$BS$を$\mathcal{L}(X)$内にもつ.一方で,
  \begin{align*}
    BS_n & = (I-(I-B))S_n                                 \\
         & = S_n-(I-B)S_n                                 \\
         & = \sum^n_{i=0}(I-B)^i -\sum^{n+1}_{i=1}(I-B)^i \\
         & = I-(I-B)^{n+1}
  \end{align*}
  となり,定理の仮定より$\|I-B\|_{\mathcal{L}(X)}<1$を持つため
  \begin{equation*}
    \|BS_n-I\|_{\mathcal{L}(X)} = \|(I-B)^{n+1}\|_{\mathcal{L}(X)} \leq \|(I-B)\|^{n+1}_{\mathcal{L}(X)} \rightarrow 0,\ (n\rightarrow 0)
  \end{equation*}
  となるため,点列$(BS_n)$は極限$I$も$\mathcal{L}(X)$内に持つ.よって,極限の一意性より
  \begin{equation*}
    BS=I
  \end{equation*}
  を得る.$\mathcal{R}(I)=X$であるため,$\mathcal{R}(BS)=X$である.その上,$X=\mathcal{R}(BS)\subset \mathcal{R}(B)$と$\mathcal{R}(B)\subset X$となるため,
  \begin{equation*}
    \mathcal{D}(S)=\mathcal{R}(B)=X
  \end{equation*}
  となる.

  同様の議論を$S_nB\in\mathcal{L}(X)$について行うと
  \begin{equation*}
    SB=I
  \end{equation*}
  と
  \begin{equation*}
    \mathcal{D}(B)=\mathcal{R}(S)=X
  \end{equation*}
  が得られる.そのため,$B$は逆作用素を持ち,逆作用素$B^{-1}=S\in\mathcal{L}(X)$である.

  また,
  \begin{equation*}
    S_n=I+(I-B)+(I-B)^2+\cdots+(I-B)^n \rightarrow B^-1,\ (n\rightarrow\infty)
  \end{equation*}
  より
  \begin{equation*}
    B^{-1}=I+(I-B)+(I-B)^2+\cdots = \sum^{\infty}_{i=0}(I-B)^i\\
  \end{equation*}
  となる.

  最後に
  \begin{equation*}
    \|B^{-1}\|_{\mathcal{L}(X)} = \left \| \sum^\infty_{i=0}(I-B)^i \right \| _{\mathcal{L}(X)} \leq \sum^\infty_{i=0}\|(I-B)\|^i
  \end{equation*}
  となり,初項1,公比$\|I-B\|_{\mathcal{L}(X)}<1$の総和より
  \begin{equation*}
    \|B^{-1}\|_{\mathcal{L}(X)} \leq \frac{1}{1-\|I-B\|_{\mathcal{L}(X)}}
  \end{equation*}
\end{proof}

\begin{thm}
  $X$と$Y$をBanach空間とする.$A\in\mathcal{L}(X,Y),R\in\mathcal{L}(Y,X)$とする.もし$RA$が全単射ならば,$A$は単射であり,$R$は全射である.
\end{thm}

\begin{proof}
  「$A$が単射」の証明

  定理(\ref{thm:線形作用素に対する単射性(1)})(線形作用素に対する単射性(1))の(2)を用いて証明する.$u\in X$とし,$RA$が単射であることに注意すると
  \begin{equation*}
    Au=0 \Rightarrow RAu=0 \Rightarrow u=0
  \end{equation*}

  「$R$が全射」の証明

  $RA$が全射であるため任意の$g\in X$に対して,
  \begin{equation*}
    RAu=g
  \end{equation*}
  となる$u\in X$が存在する.その上,$v=Au$とすると任意の$g\in X$に対して
  \begin{equation*}
    Rv=g
  \end{equation*}
  となる$v\in Y$が存在するため,$R$は全射である.
\end{proof}

\begin{dfn}[Fr\'{e}chet微分]
  作用素$F:X\rightarrow Y$が$x_0\in X$でFr\'{e}chet微分可能であるとは,ある有界線形作用素$E:X\rightarrow Y$が存在して
  \begin{equation*}
    \lim_{\|h\|_X\rightarrow 0} \frac{\|F(x_0+h)-F(x_0)-Eh\|_Y}{\|h\|_X} = 0
  \end{equation*}
  が成り立つことをいう.このとき$E$は作用素$F$の$x_0$におけるFr\'{e}chet微分といい,$E=DF(x_0)$とも書く.もしも作用素$F:X\rightarrow Y$がすべての$x\in X$に対してFr\'{e}chet微分可能ならば,$F$は$X$において$C^1\textrm{-Fr\'{e}chet}$微分可能という.
\end{dfn}

%===================%
\subsection{Banachの不動点定理}
\begin{dfn}[不動点]
  $X$を係数帯が$\mathbb{K}$のBanach空間とする.$M$は空でない閉集合で$M\subset X$を満たすとする.$A$を$M$から$M$への写像とする.$x\in M$が$A$の不動点であるとは,$x$が
  \begin{equation*}
    x=Ax
  \end{equation*}
  を満たすことである.
\end{dfn}

\begin{dfn}[距離空間]
  $X$をノルム空間とし,$x,y\in X$に対して実数値を対応させる関数$d(\cdot,\cdot):X\times X\rightarrow \mathbb{R}$が定義され,
  \begin{enumerate}
    \item $d(x,y) \leq 0$かつ$d(x,y)=0 \Leftrightarrow x=y,\ x,y\in X$
    \item $d(x,y)=d(y,x),\ x,y\in X$
    \item $d(x,y)\leq d(x,z) + d(z,y),\ x,y,z\in X$
  \end{enumerate}
  を満たすとき,$d$を距離空間という.距離の備わった集合を距離空間という.
\end{dfn}

\begin{dfn}[縮小写像]
  $X$を係数体が$\mathbb{K}$のBanach空間とする.$M$は空でない閉集合で$M\subset X$を満たすとする.写像$A:M\rightarrow M$が$k$次の縮小写像であるとは,$0\leq k<1$を満たす定数$k$が存在し,$\forall x,y \in M$について
  \begin{equation*}
    \|Ax-Ay\|\leq k\|x-y\|
  \end{equation*}
  を満たすことである.
\end{dfn}

\begin{thm}[Banachの不動点定理]
  $X$を係数体が$\mathbb{K}$のBanach空間とする.$M$は空でない閉集合で$M\subset X$を満たすとする.$A$は$M$から$M$への$k$次の縮小写像とする.そのとき,問題
  \begin{equation}
    \label{equ:Banachの不動点定理}
    \mathrm{Find}\ u\in M\ \mathrm{s.t.} \ u=Au
  \end{equation}
  は真の解$u^*$を$M$内にただ一つ持つ.即ち,写像$A$は$M$上にただ一つ不動点$u^*$を持つ.
\end{thm}

\begin{proof}
  $u_0$を閉集合$M$の元として与えられていると仮定する.点列$(u_n)$は反復法
  \begin{equation}
    \label{equ:反復法による点列}
    u_{n+1}=Au_n,\ n=0,1,\cdots
  \end{equation}
  にとって得られる.そのとき,証明のプロセスは次のように考える.
  \begin{enumerate}
    \item $(u_n)$がCauchy列になること,さらにBanach空間の完備性を使うことで,$u_n\rightarrow u,\ n\rightarrow \infty$となる$u$が$X$内に存在することを示す.
    \item $u$が(\ref{equ:Banachの不動点定理})を満たす真の解$u^*$と一致することを示す(解の存在性).
    \item 真の解$u^*$が$M$内で一意であることを示す.
  \end{enumerate}

  \textbf{1}

  (\ref{equ:反復法による点列})より,
  \begin{equation}
    \|u_n-u_{n+1}\|=\|Au_{n-1}-Au_{n}\|
  \end{equation}
  となる.定理の仮定より$A$は$k$次の縮小写像であるため,
  \begin{equation}
    \|Au_{n-1}-Au_{n}\|\leq k\|u_{n-1}-u_{n}\|
  \end{equation}
  となる定数$k$が存在する.同様に$\|u_{n-1}-u_{n}\|$に(\ref{equ:反復法による点列})と$A$の縮小写像の性質を使うと最終的に
  \begin{equation}
    \label{equ:点列と縮小写像より}
    \|u_{n}-u_{n+1}\|\leq k^n\|u_{0}-u_{1}\|
  \end{equation}
  を得る.

  次に三角不等式より,$n=0,1,2,\cdots,\ m>n$について
  \begin{align}
    \|u_{n}-u_{m}\| & = \|(u_{n}-u_{n+1})+(u_{n+1}-u_{n+2})+\cdots+(u_{m-1}-u_{m})\|      \\
                    & \leq \|u_{n}-u_{n+1}\|+\|u_{n+1}-u_{n+2}\|+\cdots+\|u_{m-1}-u_{m}\|
  \end{align}
  となる.上の式に(\ref{equ:点列と縮小写像より})を適用すると
  \begin{align}
    \|u_{n}-u_{m}\| & \leq \|u_{n}-u_{n+1}\|+\|u_{n+1}-u_{n+2}\|+\cdots+\|u_{m-1}-u_{m}\| \label{equ:-1}    \\
                    & \leq k^n\|(u_{n}-u_{n+1}\|+k^{n+1}\|u_{n+1}-u_{n+2}\|+\cdots+k^{m-1}\|u_{m-1}-u_{m}\| \\
                    & = k^n(1+k+\cdots+k^{m-n-1})\|u_0-u_1\|
  \end{align}
  となる.ここで,$k$は$0<k<1$であるため,$1+k+\cdots+k^{m-n-1}\leq 1+k+\cdots+k^{m-1}$となる.さらに,等比級数から
  \begin{equation}
    1+k+\cdots+k^{m-1} = \frac{1-k^m}{1-k}
  \end{equation}

  であるため,(\ref{equ:-1})は
  \begin{equation}
    \|u_{n}-u_{m}\| \leq \frac{k^n(1-k^m)}{1-k} \|u_0-u_1\|
  \end{equation}
  となる.よって,$k$は$0<k<1$から$k^n\rightarrow 0,\ n\rightarrow \infty$と$k^m\rightarrow 0,\ m\rightarrow \infty$となる.すなわち,
  \begin{equation*}
    \|u_{n}-u_{m}\| \rightarrow 0,\ (n,m\rightarrow \infty)
  \end{equation*}
  となるため,点列$(u_n)$はCauchy列である.さらに$X$はBanach空間であるため,$X$は完備である.すなわち,任意のCauchy列が$X$の中で極限を持つ.よって,点列$(u_n)$は
  \begin{equation*}
    u_n\rightarrow u,\ n\rightarrow \infty
  \end{equation*}
  となる$u \in X$が存在する.

  \textbf{2}

  $t_0$を$M$の元とする.仮定より,$A$は$M$から$M$への写像であるため,$A(M)\subseteq M$となる.すなわち,$u_1=Au_0$が成立する$u_1\in M$が存在する.同様に,$\forall n \in \mathbb{N}$について$u_n\in M$が存在する.さらに,$M$は閉集合であるため,\textbf{1}で存在を証明した$u$は$M$に属する.そのうえで仮定とり$A$は$k$次の縮小写像であるため
  \begin{equation*}
    \|Au_n-Au\|\leq k\|u_n-u\|
  \end{equation*}
  を得る.\textbf{1}より点列$(u_n)$は極限をもつため,$\|u_n-u\|\rightarrow 0,\ n\rightarrow \infty$となる.すなわち
  \begin{equation*}
    \|Au_n-Au\|\rightarrow 0,\ n\rightarrow \infty
  \end{equation*}
  となるため,$Au$は$Au_n$の極限である.よって(\ref{equ:反復法による点列})とすると
  \begin{equation*}
    u=Au
  \end{equation*}
  が成立する.とって$u$は(\ref{equ:Banachの不動点定理})を満たす真の解$u^*$となる.

  \textbf{3}

  $u^*,v^*\in M$をそれぞれ$u^*=Au^*$と$v^*=Av^*$を満たすとする.その時,$A$は$k$次の縮小写像であるため
  \begin{equation*}
    \|u^*-v^*\| \leq \|Au^*-Av^*\| \leq k\|u^*-v^*\|
  \end{equation*}
  を得る.ここで$0\leq k<1$であるため,不等式を満たすものは$u^*=v^*$の場合のみである.すなわち(\ref{equ:Banachの不動点定理})を満たす真の解は一意である.
\end{proof}

\subsection{簡易ニュートン写像}
本節では,$X,Y$をBanach空間とし,写像$F:X\rightarrow Y$に対して
\begin{equation*}
  F(x)=0 \in Y
\end{equation*}
という(非線形)作用素方程式を考える.

このとき,写像$T:X\rightarrow X$を
\begin{equation*}
  T(x):=x-AF(X)
\end{equation*}
で定義する.これを簡易ニュートン写像という.ここで$A:Y\rightarrow X$はある全単射な線形作用素である.いま$\bar{x}$を$F(\bar{x})\approx 0$となる近似解とし,$\bar{x}$の近傍を
\begin{align*}
  B(\bar{x},r)            & :=\{x\in X\:\|x-\bar{x}\|<r\}\ (開球)     \\
  \overline{B(\bar{x},r)} & :=\{x\in X\:\|x-\bar{x}\|\leq r\}\ (閉球)
\end{align*}
で定義する.このとき,もし,$B(\bar{x},r)$上で写像$T$が縮小写像となれば,Banachの不動点定理から$F(\tilde{x})=0$をみたす解$\tilde{x}\in B(\bar{x},r)$がただ一つ存在することになる.このように解の存在を仮定せずに近似解近傍での収束をいう定理をNewton-Kantorovichの定理という.

\subsection{フーリエ級数}
\subsubsection{フーリエ級数の導出}
ある関数$f(x)$が有限の閉空間$[a,b]$で定義されている場合に,次の性質をもつとき区分的に連続であるという.
\begin{enumerate}
  \item $f(x)$は有限個の不連続点を除いて連続である.
  \item $f(x)$の不連続点$c$では$f(c+0)$と$f(c-0)$が存在する.
\end{enumerate}

また,$f(x)$が無限区間で定義されている場合には,$f(x)$が任意の有限区間で区分的に連続であるときに,$f(x)$は区分的に連続であるという.以降は,特に断らないかぎり区分的に連続な関数のみを考えることにする.

関数$f(x),\ (x\in [-\pi,\pi])$を周期$2\pi$の周期関数とする.このような$f(x)$が以下の三角関数で表されるとして,係数$a_0,a_1,a_2,\cdots,b_0,b_1,b_2,\cdots$を求めることを考える.

\begin{align}
  f(x) & = \frac{a_0}{2} + \sum_{\infty}^{n=1} (a_n \cos nx + b_n \sin nx)                       \label{equ:fourie-1} \\
       & = \frac{a_0}{2} + (a_1 \cos x + b_1 \sin nx) + (a_2 \cos 2x + b_2 \sin 2x) + \cdots \notag
\end{align}

そのために,次の公式を用意する.ここで,$m,n\in \mathbb{N}$である.
\begin{align}
  \int_{-\pi}^{\pi} \cos nx \cos mx dx                        & = \begin{cases}
                                                                    0 \quad (n\neq m) \\
                                                                    \pi \quad (n=m)
                                                                  \end{cases} \notag       \\
  \int_{-\pi}^{\pi} \sin nx \sin mx dx                        & = \begin{cases}
                                                                    0 \quad (n\neq m) \\
                                                                    \pi \quad (n=m)   \\
                                                                  \end{cases} \notag       \\
  \int_{-\pi}^{\pi} \sin nx \cos mx dx                        & = 0   \label{equ:fourie-2} \\
  \int_{-\pi}^{\pi} \sin nx dx = \int_{-\pi}^{\pi} \cos mx dx & = 0 \label{equ:fourie-3}
\end{align}

ここで,次のような形式的計算が許されると仮定する.まず,(\ref{equ:fourie-1})の両辺について$-\pi$から$\pi$まで積分し,(\ref{equ:fourie-3})を用いれば
\begin{align*}
  \int_{-\pi}^{\pi} f(x)dx & = \frac{a_0}{2}\int_{-\pi}^{\pi} dx + \sum_{n=1}^{\infty} (a_n \int_{-\pi}^{\pi}\cos nx dx+ \int_{-\pi}^{\pi}b_n \sin nx dx) \\
                           & = \pi a_0
\end{align*}
\begin{equation}
  \label{equ:fourie-a0}
  \therefore a_0 = \frac{1}{\pi}\int_{-\pi}^{\pi} f(x) dx
\end{equation}

次に,(\ref{equ:fourie-1})の両辺に$\cos mx$をかけて,$-\pi$から$\pi$まで積分し,\ref{equ:fourie-2}\ref{equ:fourie-3}を用いれば
\begin{align*}
  \int_{-\pi}^{\pi} f(x) \cos mx dx & = \frac{a_0}{2} \int_{-\pi}^{\pi} \cos mx dx                                                                    \\
                                    & \quad + \sum_{n=1}^{\infty} (a_n \int_{-\pi}^{\pi}\cos nx \cos mx dx + \int_{-\pi}^{\pi}b_n \sin nx \cos mx dx) \\
                                    & = \pi a_0
\end{align*}
\begin{equation}
  \label{equ:fourie-am}
  \therefore a_m = \frac{1}{\pi}\int_{-\pi}^{\pi} f(x) \cos mx dx
\end{equation}

同様に,(\ref{equ:fourie-1})の両辺に$\sin mx$をかけて,$-\pi$から$\pi$まで積分することにより次の式が得られる.
\begin{equation}
  \label{equ:fourie-bm}
  b_m = \frac{1}{\pi}\int_{-\pi}^{\pi} f(x)\sin xm dx
\end{equation}

(\ref{equ:fourie-a0}),(\ref{equ:fourie-am}),(\ref{equ:fourie-bm})をまとめて次のようになる.

\begin{dfn}[フーリエ級数]
  \begin{equation*}
    f(x) \sim \frac{a_0}{2} + \sum_{n=1}^{\infty} a_n \cos nx + \sum_{n=1}^{\infty} b_n \sin nx
  \end{equation*}
  ただし
  \begin{align*}
    a_n = \frac{1}{\pi} \int_{0}^{2\pi} f(x) \cos nx dx,\ (n\geq 0) \\
    a_n = \frac{1}{\pi} \int_{0}^{2\pi} f(x) \cos nx dx,\ (n\geq 1)
  \end{align*}
  この無限級数を$f$のフーリエ級数といい,$a_n,b_n$をフーリエ係数という.
\end{dfn}

\begin{dfn}[複素フーリエ変換]
  また,$$\cos nx = \frac{e^{inx}+e^{-inx}}{2},\ \sin nx = \frac{e^{inx}+e^{-inx}}{2}\quad (iは虚数単位)$$という関係を用いて
  \begin{align*}
    f(x) & = \sum_{k=-\infty}^{\infty} c_k e^{ikx}         \\
    c_k  & = \frac{1}{2\pi} \int_{0}^{2\pi} f(x)e^{ikx} dx
  \end{align*}
  と複素数を用いた形式も考えられる.これを$f$の複素フーリエ級数,$c_k$を複素フーリエ係数という.

  これらには関係式
  \begin{align*}
    c_k=
    \begin{dcases}
      \frac{a_k}{2}            & ,\ k=0 \\
      \frac{a_k-ib_k}{2}       & ,\ k>0 \\
      \frac{a_{-k}-ib_{-k}}{2} & ,\ k<0
    \end{dcases}
  \end{align*}
\end{dfn}
があり,変換可能である.

\subsubsection{フーリエ級数の性質}
\begin{enumerate}
  \item 対称性 \\
        周期関数$f(x)$が偶関数であるとき,$\sin$関数の係数$b_n$は$0$になる.偶関数のフーリエ級数は
        \begin{equation*}
          f(x)=\frac{a_0}{2} + \sum_{n_1}^{\infty} a_n \
        \end{equation*}

  \item bb
\end{enumerate}


\begin{comment}
\mathcal{L}(X,Y)
点列$(A_n)$
\end{comment}

%% 定義 dfn
%% 定理 thm
%% 証明 proof

%\begin{thebibliography}{9}
%  \bibitem{ラベル} 著者名, ``題名'', 出版社, pp.0-1.
%  \bibitem{lit:name} 著者名, ``題名'',\linebreak \verb|http://www.kisarazu.ac.jp/|
%\end{thebibliography}
%%文章のあとに,\cite{label1, label2},\cite{lite:xxx}とつける
%隙間ができたら,\newlineを入れ込む
\end{document}