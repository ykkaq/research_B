%ctrl+alt+b -> build, ctrl+alt+v => preview at new tab
%ctrl+click -> highlight line

%\documentclass[titlepage]{jsarticle}
\documentclass{jsarticle}
%\documentclass[11pt,a4paper]{jsreport}

\usepackage{algorithmic}
\usepackage{amsmath,amssymb,amsfonts,amsthm}
\usepackage{ascmac}
\usepackage{bm}
\usepackage{caption}
\usepackage{cite}
\usepackage{comment}
\usepackage[dvipdfmx]{color}
\usepackage{colortbl}
\usepackage{float}
\usepackage[dvipdfmx]{graphicx}
\usepackage{multicol}
\usepackage{latexsym}
\usepackage{listings}
\usepackage[dvipdfmx]{pict2e}
\usepackage[ipaex]{pxchfon}
\usepackage{tabularx}
\usepackage{textcomp}
\usepackage{underscore}
\usepackage{ulem}
\usepackage{url}
\usepackage{wrapfig}
\usepackage{xcolor}

\usepackage[top=30truemm,bottom=30truemm,left=25truemm,right=25truemm]{geometry}

\captionsetup[figure]{labelsep=space}
\captionsetup[table]{labelsep=space}

\theoremstyle{difinition}
\newtheorem{dfn}{定義}
\newtheorem{thm}{定理}

\renewcommand{\qedsymbol}{$\blacksquare$}
\renewcommand{\proofname}{\textbf{証明}}

%============
\begin{document}
\pagestyle{plain}
\title{タイトル}
\author{2131701 齋藤悠希}
\date{}
\maketitle

\section{Banach Space}
\begin{dfn}[線形空間の公理]
  空でない集合$X$が,係数体$\mathbb{K}$上の線形空間であるとは,任意の$u+v \in X$とスカラー$\alpha \in \mathbb{K}$に対して,加法$u+v \in X$とスカラー乗法$\alpha u \in X$が定義されていて,任意の$u,v,w \in X$とスカラー$\alpha, \beta \in \mathbb{K}$に対して次のことが成り立つことである.
  \begin{enumerate}
    \item $(u+v)+w=u+(v+w)$
    \item $u+v=v+u$
    \item $u+0=u$となる$0 \in X$が一意に存在
    \item $u+(-u)=0$となる$-u\in X$が一意に存在
    \item $\alpha(u+v)=\alpha u+\alpha v$
    \item $(\alpha +\beta)u = \alpha u+\beta u$
    \item $(\alpha \beta)u = \alpha (\beta u)$
    \item $1u=u, 1 \in \mathbb{K}$
  \end{enumerate}
\end{dfn}

\begin{dfn}[ノルムとノルム空間の定義]
  $X$を係数体$\mathbb{K}$上の線形空間とする.$X$で定義された関数$||\cdot||:X\rightarrow \mathbb{K}$上で定義された関数が$X$のノルムであるとは
  \begin{enumerate}
    \item $||u||\geq 0, \quad u \in X$
    \item $||u||=0 \Leftrightarrow u=0$
    \item $||\alpha u||=|\alpha| ||u||, \quad (\alpha \in \mathbb{K}, u \in X)$
    \item $||u+v||\leq ||u||+||v||$
  \end{enumerate}
  が成立することである.さらに$X$に1つのノルムが指定されているとき,$X$はノルム空間という.
\end{dfn}

\begin{dfn}[ノルム空間の収束と極限]$X$をノルム空間とする.$X$の点列$(u_n)\subset X$は
  \begin{align*}
    \forall\epsilon >0, \ \exists N\in \mathbb{N}, \space \forall N \geq Nに対して||u_n-u||<\epsilon
  \end{align*}
  のとき,点$u\in X$に収束するといい,
  \begin{align*}
    ||u_n-u||\rightarrow 0, \ \left(n\rightarrow\infty\right)
  \end{align*}
  と表す.このとき,$u$を$u_n$の極限といい,
  \begin{align*}
    u_n-u, \ (n\rightarrow \infty)
  \end{align*}
  と表す.
\end{dfn}

\begin{dfn}[Cauchy列]
  $X$をノルム空間とする.そのとき$X$がCauchy列であるとは
  \begin{align*}
    u_n-u_m\rightarrow 0, \ \left(n,m\rightarrow\infty \right)
  \end{align*}
  が成立することである.即ち
  \begin{align*}
    ||u_n-u_m||\rightarrow 0, \ \left(n,m\rightarrow \infty\right)
  \end{align*}
  が成立することである.
\end{dfn}

\begin{dfn}[完備]
  $X$をノルム空間とする.$X$が完備であるとは,任意のCauchy列$(u_n)$が$X$の中で極限をもつことである.すなわち,任意のCauchy列$(u_n\subset X)$が
  \begin{align*}
    \|u_n-u\|\rightarrow 0\ ,\left(n\rightarrow 0\right)
  \end{align*}
  となる極限$u$を$X$内に持つことである.
\end{dfn}

\begin{dfn}[Banach空間]
  ノルム空間$X$がBanach空間であるとは,$X$が完備であることである.
\end{dfn}

\begin{thm}[逆三角不等式]
  $X$をノルム空間とする.任意の$u,v\in X$について次の不等式を満たす.
  \begin{align*}
    |\|u\|-\|v\||\geq\|u-v\|
  \end{align*}
\end{thm}

\begin{proof}
  任意の$u,v\in X$について
  \begin{align*}
    \|u\| & =\|u-v+v\|\geq\|u-v\|+\|v\|               \\
    \|v\| & =\|v-u+u\|\geq\|v-u\|+\|u\|=\|u-v\|+\|u\|
  \end{align*}
  となる.よって
  \begin{align*}
    \|u\|-\|v\|\geq\|u-v\| \\
    \|v\|-\|u\|\geq\|u-v\|
  \end{align*}
  となるため,
  \begin{align*}
    |\|u\|-\|v\||\geq\|u-v\|
  \end{align*}
  を持つ.
\end{proof}

\begin{dfn}[有界列]
  $X$をノルム空間とする.そのとき$X$の点列$(u_n)$が有界列とは任意の$n\in\mathbb{N}$に対して
  \begin{align*}
    \|u_n\|\geq M
  \end{align*}
  となる定数$M>0$が存在することである.
\end{dfn}

\begin{thm}[Cauchy列ならば有界列]
  $X$をノルム空間とする.そのとき$X$の点列$(u_n)$がCauchy列ならば有界列でもある.
\end{thm}

\begin{proof}
  $X$の点列$(u_n)$がCauchy列であるために,$\epsilon -N$論法を用いた表記で
  \begin{align*}
    \forall\epsilon >0, \ \exists N\in \mathbb{N}, \ \forall n,m \geq Nに対して||u_n-u_m||<\epsilon
  \end{align*}
  を満たす.$\epsilon=1$としても,それに対応した$N$が存在し,任意の$n\geq N$に対して
  \begin
\end{proof}


%\begin{thebibliography}{9}
%  \bibitem{ラベル} 著者名, ``題名'', 出版社, pp.0-1.
%  \bibitem{lit:name} 著者名, ``題名'',\linebreak \verb|http://www.kisarazu.ac.jp/|
%\end{thebibliography}
%%文章のあとに,\cite{label1, label2},\cite{lite:xxx}とつける
%隙間ができたら,\newlineを入れ込む
\end{document}